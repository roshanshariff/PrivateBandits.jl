\documentclass{article}
\usepackage[utf8]{inputenc}

\title{Differentially Private Contextual Bandits}
\author{Roshan Shariff}

\usepackage[semibold]{libertine} % a bit lighter than Times--no osf in math
\usepackage[T1]{fontenc} % best for Western European languages
\usepackage{textcomp} % required to get special symbols
\usepackage[varqu,varl]{inconsolata} % a typewriter font must be defined
\usepackage{amsmath,mathtools}
\usepackage{amsthm,thmtools,thm-restate}
\usepackage{amssymb} % loads amsfonts
\usepackage{nicefrac}
\usepackage[libertine,cmintegrals,bigdelims,vvarbb]{newtxmath} % replaces some amssymb symbols
\usepackage[scr=rsfso]{mathalfa} % Use rsfso to provide mathscr
\usepackage{dsfont}
\usepackage{bm} % load after all math to give access to bold math

\usepackage{microtype}
\usepackage[inline,shortlabels]{enumitem}
\usepackage{setspace}
\usepackage{tikz-cd}
\usepackage{booktabs}
\usepackage[boxed]{algorithm}
\usepackage{algpseudocode}
\usepackage{fullpage}
%\usepackage{dblfloatfix}

%% Bibliography/References
\usepackage[round,colon]{natbib}

%% Cross-references
\usepackage{varioref}
\usepackage[hidelinks]{hyperref}
\usepackage[capitalise]{cleveref}

%% To-do notes
\usepackage[obeyFinal]{todonotes}

% \newcommand{\tinytodo}[2][]{\todo[size=\tiny]{#2}}
\newcommand{\tinytodo}[2][]{\todo[size=\tiny, #1]{\begin{spacing}{1.0}#2\end{spacing}}}
\newcommand{\RStodo}[2][]{\tinytodo[color=red!20, #1]{R:\@#2}} % Roshan
\newcommand{\OStodo}[2][]{\tinytodo[color=blue!20, #1]{Cs: #2}} % Or
\newcommand{\fix}{\marginpar{FIX}}
\newcommand{\new}{\marginpar{NEW}}

%% Macros

\newcommand{\wildcard}{\mathinner{\,{\cdot}\,}}
\newcommand{\defeq}{\coloneq}
\newcommand{\eqdef}{\eqcolon}
\newcommand{\inv}[1]{#1^{-1}}
\newcommand{\Real}{\mathds{R}}
\newcommand{\Nat}{\mathds{N}}
\newcommand{\Int}{\mathds{Z}}
\newcommand{\mgf}{\mathrm{mgf}}
\newcommand{\UCB}{\operatorname{UCB}}
\renewcommand\mid{\mathinner{\vert}}
\DeclareMathOperator*{\argmin}{arg\,min}
\DeclareMathOperator*{\argmax}{arg\,max}
\DeclareMathOperator{\tr}{tr}
\DeclareMathOperator{\rank}{rank}

\newcommand\given[1][\delimsize]{%
  \providecommand{\delimsize}{}
  \nonscript\:#1\vert\allowbreak\nonscript\:\mathopen{}
}

\DeclarePairedDelimiter{\abs}||
\DeclarePairedDelimiter{\paren}()
\DeclarePairedDelimiter{\brck}{[}{]}
\DeclarePairedDelimiterX{\set}[1]\lbrace\rbrace{#1}
\DeclarePairedDelimiter{\ceil}\lceil\rceil
\DeclarePairedDelimiterX{\innerp}[2]\langle\rangle{#1,#2}
\DeclarePairedDelimiterXPP{\Prob}[1]{\mathds{P}}(){}{#1}
\DeclarePairedDelimiterXPP{\PrSet}[1]{\mathds{P}}\{\}{}{#1}
\DeclarePairedDelimiterXPP{\Ex}[1]{\mathds{E}}{[}{]}{}{#1}
\DeclarePairedDelimiterXPP{\Exx}[2]{\mathds{E}_{#1}}{[}{]}{}{#2}
\DeclarePairedDelimiterXPP{\Var}[1]{\mathrm{Var}}{[}{]}{}{#1}
\DeclarePairedDelimiterXPP{\One}[1]{\mathds{1}}\{\}{}{#1}
\DeclarePairedDelimiterXPP{\norm}[2]{}\Vert\Vert{_{#1}}{#2}

%% Other symbols
\newcommand{\transp}[1]{#1^\top}
\newcommand{\Aset}[1]{\mathcal{A}_{#1}}
\newcommand{\Dset}[1]{\mathcal{D}_{#1}}
\newcommand{\Cset}[1]{\mathcal{C}_{#1}}

% Theorem environments

\declaretheorem[style=definition]{definition}
\declaretheorem[style=definition]{assumption}
\declaretheorem[style=definition]{theorem}
\declaretheorem[style=definition]{lemma}
\declaretheorem[style=definition]{claim}
\declaretheorem[style=definition]{corollary}

\Crefname{assumption}{Assumption}{Assumptions}

%%%%%%%%%%%%%%%%%%%%%G%%%%%%%%%%%%%%%%%%%%%%%%%%%%%%%%%%%%%%%%%%%%%%%%%%%%%%%%%%%

\onehalfspacing

\begin{document}

\maketitle


\section{Introduction}

The multi-armed bandit is a sequential decision-making task in which
an agent (or learner) repeatedly takes an action (or arm) and receives
a corresponding reward.  The agent tries to maximize the total reward
it earns over several rounds.  Since the agent only observes the
reward for actions it takes, this is an online learning task with
partial information.  Learning the rewards associated with all the
arms requires an agent to \emph{explore} them.  Too much exploration,
however, would be counter-productive; eventually the algorithm should
\emph{exploit} the best arms it has found.  The main challenge of
bandit strategies is to balance exploration and exploitation.

If a bandit algorithm is learning from sensitive user data, we might
also want it maintain users' privacy.

\todo[inline]{Consider non-contextual bandits; only the rewards are
  private.  Does this help with the counter-example against optimism
  in Tor's paper?}

\todo[inline]{Looks like a $d^2$ dependence is necessary: consider
  $\theta^*=\paren{\pm 1/d, \dotsc}$ and the actions are $e_i$.  Then
  you need $d^2\log d$ samples?}

\section{Contextual Bandits}

A contextual bandit algorithm receives a \emph{context}
$c_t\in\Cset{}$ at every round $t$, which it uses to select an action
$a_t\in\Aset{}$.  It then receives a reward $r_t\in\Real$
corresponding to the action it selected.  The goal of the algorithm is
to choose actions that maximize its cumulative reward.  We will
represent such an algorithm by a function
$A : (\Cset{}\times\Aset{}\times\Real)^* \times \Cset{} \to
\Delta_{\Aset{}}$ whose input is a history of context-action-reward
triples along with a current context, and whose output is a
probability distribution over actions.  The context usually gives
additional information about the actions, in a sense that we will
expand upon below.

\subsection{Privacy for Contextual Bandits}

In many applications, the context and reward may be considered private
information about the users of the bandit algorithm.  For example, a
search engine might have a context consisting of a user's search
query, identity, interests, and physical location, while the reward
indicates which search result the user clicked on.  The search engine
should, of course, use the context to answer each query; furthermore,
it should learn from the reward to better respond to future queries
from other users.  However, it should also maintain privacy: its
responses to queries should not reveal \emph{too much} information
about the context and rewards it has learned from.  More precisely, we
want algorithms that are \emph{jointly differentially private} in the
following sense:

\begin{definition}
  A randomized contextual bandit algorithm $A$ is
  \emph{$(\varepsilon,\delta)$-differentially private} if the
  following holds for every $t^*\in\set{1,\dotsc,n}$.  Let
  $H = (c_t,a_t,r_t)_{t=1}^n$ and $H' = (c'_t,a'_t,r'_t)_{t=1}^n$ be
  two sequences of context-action-reward triples that differ only on
  round $t^*$; in other words, $c_t=c'_t$, $a_t=a'_t$, $r_t=r'_t$ for
  $t\neq t^*$.  \todo[inline]{Complete this definition.  How to
    construct the probability space?}
\end{definition}

\subsection{Stochastic Linear Contextual Bandits}

We will assume that the context affects the reward in a linear way, in
the following sense.  Suppose there is a known \emph{feature function}
$\phi:\Cset{}\times\Aset{}\to\Real^d$ that maps every context-action
pair to a $d$-dimensional \emph{feature vector}.  We will assume that
the reward is a linear function of the feature vector, with some added
noise: $r_t = \innerp{\theta^*}{\phi(c_t,a_t)} + \eta_t$ for some
unknown vector $\theta^*\in\Real^d$.

Since all the information the algorithm needs is encoded in the
feature vector of each action, we can now formulate the contextual
bandit problem in the following equivalent way.  Define the decision
set
$\Dset{t} \defeq \set{\phi(c_t,a)\given a\in\Aset{}} \subset \Real^d$.
The context is implicitly encoded in the decision set, and choosing
$x_t\in\Dset{t}$ is equivalent to choosing an action $a_t\in\Aset{}$.
In the contextual stochastic linear bandit framework, at each round:
\begin{enumerate}
\item The agent receives a \emph{decision set} $\mathcal{D}_t \subset
  \Real^d$.
\item The agent chooses an \emph{action} $X_t \in \mathcal{D}_t$.
\item The agent receives a \emph{reward} $Y_t = \innerp{X_t}{\theta^*} + \eta_t$.
\end{enumerate}
The vector $\theta^*\in\Real^d$ is an unknown parameter of the
environment which the agent learns so that it can maximize its reward.

\begin{assumption}[Subgaussian noise]\label{assumption:subgaussian-noise}
  We denote by
  $\mathcal{F}_t =
  \sigma(\mathcal{D}_1,X_1,Y_1,\dotsc,\mathcal{D}_{t-1},X_{t-1},Y_{t-1},\mathcal{D}_t,X_t)$
  all the information available just before the noise $\eta_t$ is
  observed.  We assume that $\eta_t$ is \emph{conditionally
    $\sigma^2$-subgaussian:}
  \begin{align*}
    \Ex{\exp(\lambda\eta_t)\given \mathcal{F}_t} &\le \exp(\lambda^2\sigma^2/2),
    &\text{for all } \lambda\in\Real.
  \end{align*}
\end{assumption}


\section{Linear UCB}

We start by analyzing the LinUCB algorithm (also known as OFUL ---
Optimism in the Face of Uncertainty--Linear).  The algorithm is as follows

\begin{algorithm}
  \caption{Linear UCB}\label{alg:linucb}
  \begin{algorithmic}
    \For{$t \in 1,2,\dotsc,n$}
    \State Receive decision set $\Dset{t} \subset \Real^d$
    \State $\Cset{t} \gets \Cset{t}(X_1,Y_1,\dotsc,X_{t-1},Y_{t-1})$
    \State $X_t \gets \argmax_{x\in\Dset{t}}
    \max_{\theta\in\Cset{t}} \innerp{x}{\theta}$
    \State Choose action $X_t$ and receive reward $Y_t$
    \EndFor
  \end{algorithmic}
\end{algorithm}

We will show a generic regret bound that will depend on certain
assumptions about the decision sets $\Dset{t} \subset \Real^d$,
unknown parameter vector $\theta^*\in\Real^d$, and confidence sets
$\Cset{t}$.

\begin{assumption}\label{assumption:linucb}
  We assume that
  \begin{itemize}
  \item The mean reward is bounded: $\abs{\innerp{x}{\theta^*}} \le B$
    for any $x\in\bigcup_t\Dset{t}$.
  \item The actions are bounded: $\norm{}{x} \le L$ for all
    $x\in\bigcup_t\Dset{t}$.
  \item The confidence intervals hold: for all $t\in[n]$,
    $x\in\Dset{t}$,
    \begin{align*}
      \UCB_t(x) &\ge \innerp{x}{\theta^*} \ge \UCB_t(x) - 2\sqrt{\beta_{t-1}}\norm{\inv{V_{t-1}}}{x}
      \shortintertext{where}
      \UCB_t(x) &\defeq \max_{\theta\in\Cset{t}} \innerp{x}{\theta}
    \end{align*}
    and $\beta_0,\dotsc,\beta_{n-1}\in\Real$ is a non-decreasing sequence
    with $\beta_{n-1} \ge 1$ and $0 \prec V_0,\dotsc,V_{n-1} \in
    \Real^{d \times d}$ is a sequence of symmetric positive definite matrices.
  \end{itemize}
\end{assumption}

\begin{lemma}\label{lemma:linucb-regret}
  Suppose \cref{assumption:linucb} holds with some
  ${(\beta_t, V_t)}_{t=0}^{n-1}$.  Then the pseudo-regret of Linear
  UCB satisfies
  \begin{align*}
    \widehat{R}_n &\le \sqrt{4n\beta_{n-1} \sum_{t=1}^n\min\set{B^2, \norm{\inv{V_{t-1}}}{X_t}^2}}.
  \end{align*}

  \begin{proof}
    Let $r_t = \max_{x\in\Dset{t}} \innerp{x-X_t}{\theta^*}$ be the
    immediate pseudo-regret suffered at round $t\in[n]$ and let
    $X_t^* = \argmax_{x\in\Dset{t}}\innerp{x}{\theta^*}$ be an optimal
    action for round $t$.  We know that the confidence interval holds
    for $X_t^*$, and also that we choose the action with highest UCB,
    so
    \begin{align*}
      \innerp{X_t^*}{\theta^*} &\le \UCB_t(X_t^*) \le \UCB_t(X_t).
    \end{align*}
    Since the lower confidence bound also holds by our assumption,
    \begin{align*}
      \innerp{X_t}{\theta^*} &\ge \UCB_t(X_t) - 2\sqrt{\beta_{t-1}}\norm{\inv{V_{t-1}}}{X_t}.
    \end{align*}
    Combining these inequalities we get
    \begin{align*}
      r_t &= \innerp{X_t^*}{\theta^*} - \innerp{X_t}{\theta^*} \\
         &\le \UCB_t(X_t) - \innerp{X_t}{\theta^*} \\
         &\le 2\sqrt{\beta_{t-1}}\norm{\inv{V_{t-1}}}{X_t}.
    \end{align*}
    Since we assumed that the mean absolute reward is bounded by $B$
    we also get that $r_t \le 2B$, and since $\beta_{n-1} \ge \max\set{\beta_t, 1}$
    we have
    \begin{align*}
      r_t &\le \min\set{2B, 2\sqrt{\beta_{t-1}}\norm{\inv{V_{t-1}}}{X_t}}
           \le 2\sqrt{\beta_{n-1}} \min\set{B, \norm{\inv{V_{t-1}}}{X_t}}.
    \end{align*}

    Now we apply Jensen's inequality as follows:
    \begin{align*}
      \widehat{R}_n^2 &= n^2 \paren[\Big]{\sum_{t=1}^n \frac{r_t}{n}}^2
                      \le n^2 \sum_{t=1}^n \frac{r_t^2}{n} = n\sum_{t=1}^nr_t^2 \\
      \widehat{R}_n &\le \sqrt{n\sum_{t=1}^n r_t^2} \\
                    &= \sqrt{4n\beta_{n-1} \sum_{t=1}^n\min\set{B^2, \norm{\inv{V_{t-1}}}{X_t}^2}}.
                      \qedhere
    \end{align*}
  \end{proof}
\end{lemma}

\begin{lemma}[Elliptical Potential]\label{lemma:elliptical-potential}
  Let $x_1,\dotsc,x_n \in \Real^d$ and for all $t\in[n]$,
  $U_t = U_0 + \sum_{s=1}^t x_s \transp{x_s}$ and
  $L \ge \max_t\norm{}{x_t}$. Then for any $B > 0$
  \begin{align*}
    \sum_{t=1}^n\min\set{B^2, \norm{\inv{U_{t-1}}}{x_t}^2}
    &\le \max\set{2, B^2}\log\frac{\det U_n}{\det U_0}
      \le \max\set{2, B^2} d\log\frac{\tr U_0+nL^2}{d\det^{1/d} U_0}.
  \end{align*}

  \begin{proof}
    We use the fact that $u \wedge B \le \max\set{2, B}\log(1+u)$ for
    any $u \ge 0$, so that
    \begin{align*}
      \sum_{t=1}^n\min\set{B^2 \wedge \norm{\inv{U_{t-1}}}{x_t}^2}
      &\le \max\set{2, B^2} \sum_{t=1}^n \log(1 + \norm{\inv{U_{t-1}}}{x_t}^2).
    \end{align*}
    We will show that this last summation is $\log(\det U_n/\det
    U_0)$.  For $t \ge 1$ we have
    \begin{align*}
      U_t &= U_{t-1} + x_t\transp{x_t}
           = U_{t-1}^{1/2} (I + U_{t-1}^{-1/2}x_t\transp{x_t}U_{t-1}^{-1/2}) U_{t-1}^{1/2} \\
      \det U_t &= \det U_{t-1}\det(I + U_{t-1}^{-1/2}x_t\transp{x_t}U_{t-1}^{-1/2}).
    \end{align*}

    Consider the eigenvectors of the matrix $I+y\transp{y}$ for an
    arbitrary vector $y\in\Real^d$.  We know that $y$ itself is an
    eigenvector with eigenvalue $1+\norm{}{y}^2$:
    \begin{align*}
      (I + y\transp{y})y &= y + y\innerp{y}{y} = (1+\norm{}{y}^2)y.
    \end{align*}
    Moreover, since $I+y\transp{y}$ is symmetric, every other
    eigenvector $u$ is orthogonal to $y$, so that
    \begin{align*}
      (I + y\transp{y})u &= u + u\innerp{y}{u} = u.
    \end{align*}
    Therefore the only eigenvalues of $I+y\transp{y}$ are
    $1+\norm{}{y}^2$ (with eigenvector $y$) and 1.

    In our case $y = U_{t-1}^{-1/2}x_t$ and $\norm{}{y}^2 =
    \transp{x_t}\inv{U_{t-1}}x_t = \norm{\inv{U_{t-1}}}{x_t}^2$, so we get our
    first inequality:
    \begin{align*}
      \det U_n &= \det U_0 \prod_{t=1}^n(1 + \norm{\inv{U_{t-1}}}{x_t}^2) \\
      \log\frac{\det U_n}{\det U_0} &= \sum_{t=1}^n\log(1+\norm{\inv{U_{t-1}}}{x_t}^2).
    \end{align*}

    To get the second inequality, we apply the arithmetic-geometric
    mean inequality to the eigenvalues $\lambda_i$ of $U_n$:
    \begin{align*}
      \det U_n &= \prod_{i=1}^d \lambda_i
                \le \paren[\Big]{\frac{1}{d} \sum_{i=1}^d \lambda_i}^d
                = \paren{(1/d)\tr U_n}^d
                \le \paren{(\tr U_0 + nL^2)/d}^d \\
      \log\frac{\det U_n}{\det U_0}
              &\le d\log\frac{\tr U_0 + nL^2}{d\det^{1/d}U_0}
                \qedhere
    \end{align*}
  \end{proof}
\end{lemma}


\begin{theorem}[Generic LinUCB Regret]\label{thm:linucb-regret}
  Suppose \cref{assumption:linucb} holds with some $\beta_t$ and
  $V_t \defeq G_t + H_t$, where
  $G_t \defeq \sum_{s=1}^t X_s\transp{X_s}$ and
  $H_t \succeq H \succ 0$ for some symmetric
  $H_t, H \in \Real^{d\times d}$.  Then the pseudo-regret
  \begin{align*}
    \widehat{R}_n &= \sum_{t=1}^n \max_{x\in\Dset{t}} \innerp{x}{\theta^*} - \innerp{X_t}{\theta*} \\
    \intertext{of Linear UCB satisfies}
    \widehat{R}_n &\le \sqrt{4\max\set{2,B^2}n\beta_{n-1} \log\frac{\det V_n}{\det H}} \\
    \widehat{R}_n &\le \sqrt{4\max\set{2,B^2}dn\beta_{n-1}\log\frac{\tr H + nL^2}{d\det^{1/d} H}}.
  \end{align*}

  \begin{proof}
    From our assumptions and \cref{lemma:linucb-regret}, we know that
    the Linear UCB regret is bounded by
    \begin{align*}
      \widehat{R}_n &\le \sqrt{4n\beta_{n-1} \sum_{t=1}^n\min\set{B^2, \norm{\inv{V_{t-1}}}{X_t}^2}}.
    \end{align*}
    Define $U_t \defeq G_t + H$, so that $U_t \preceq V_t$ and by
    \citet[Theorem~7.8]{ZhangMatrixTheory2011} it follows that
    $\inv{V_t} \preceq \inv{U_t}$, which means $\norm{\inv{V_{t-1}}}{X_t}^2
    \le \norm{\inv{U_{t-1}}}{X_t}^2$.  We now apply
    \cref{lemma:elliptical-potential} to the sequence $U_t$ to get
    \begin{align*}
      \sum_{t=1}^n\min\set{B^2,\norm{\inv{V_{t-1}}}{X_t}}
      &\le \sum_{t=1}^n\min\set{B^2,\norm{\inv{U_{t-1}}}{X_t}} \\
      &\le \max\set{2, B^2}\log\det\frac{U_n}{U_0} \\
      &\le \max\set{2, B^2}d\log\frac{\tr U_0 + nL^2}{d \det^{1/d} U_0}.
    \end{align*}
    To get the required bounds, it only remains to see that $U_0 = H$
    and that $\det U_n \le \det V_n$ \citep[again
    by][Theorem~7.8]{ZhangMatrixTheory2011}.
  \end{proof}

\end{theorem}


\section{Ellipsoidal Confidence Sets}
\label{sec:ellips-conf-bounds}

Throughout this section we will fix some particular round $t \le n$
and suppress the subscripted $t$.

We will consider confidence sets for the value of $\theta^*$ that are
ellipsoidal with centre $\hat{\theta}$, of the form
$\set{\theta\in\Real^d \given \norm{V}{\theta-\hat{\theta}}^2 \le
  \beta}$.  Then the upper confidence bound of an action $x$ is
\begin{align*}
  \UCB(x) &= \max_{\theta: \norm{V}{\theta-\hat\theta}^2 \le \beta} \innerp{x}{\theta}
\end{align*}
We solve this constrained optimization problem by the method of
Lagrange multipliers:
\begin{align*}
  \mathcal{L}(\theta, \lambda) &= \innerp{x}{\theta} - \lambda(\norm{V}{\theta-\hat\theta}^2 - \beta) \\
  \nabla_\theta\mathcal{L}(\theta, \lambda) &= \transp{x} - \lambda\transp{(\theta-\hat\theta)}V = 0 \\
  \theta &= \hat\theta + \frac{{\inv{V}x}}{\lambda} \\
  \shortintertext{To find $\lambda$:}
  \beta &= \norm{V}{\theta-\hat\theta}^2
          = \norm[\bigg]{V}{\frac{\inv{V}x}{\lambda}}^2
          = \frac{1}{\lambda^2}\transp{x}\inv{V}x \\
  \lambda &= \norm{\inv{V}}{x} / \sqrt\beta \\
  \intertext{Which we substitute into the expression above:}
  \UCB(x) &= \innerp[\bigg]{x}{\hat\theta + \frac{\sqrt\beta}{\norm{\inv{V}}{x}}\inv{V}x} \\
          &= \innerp{x}{\hat\theta} + \sqrt\beta \norm{\inv{V}}{x}.
\end{align*}
We have therefore shown that if we construct an ellipsoidal confidence
set with an appropriate $V$ and $\beta$, then the requirements of
\cref{assumption:linucb} will be satisfied with the above $\UCB$
index.

Let $X_s \in \Real^d$ be the action vector selected by the algorithm at
time $s$, so that $X$ is a $t \times d$ matrix.  The \emph{Gram matrix} is given by
$G \defeq \transp{X}X \in \Real^{d \times d}$.  Let $y_s$ be the
reward received at time $s$, so that $y\in\Real^t$.

We will take $\hat{\theta}$ to be a regularized solution of the
least-squares system
\begin{align*}
  X\theta &\approx y,
\end{align*}
regularized by the symmetric positive definite matrix $H \succeq 0$:
\begin{align*}
  \hat{\theta} &= \argmin_\theta \frac{1}{2}\norm{}{X\theta - y}^2 + \frac{1}{2}\norm{H}{\theta}^2 \\
               &= \inv{\paren{\transp{X}X + H}} \transp{X}y \\
               &= \inv{\paren{\transp{X}X + H}} \transp{X} \paren{X\theta^* + \eta} \\
               &= \theta^* + \inv{\paren{\transp{X}X + H}}\paren{\transp{X}\eta - H\theta^*} \\
               &= \theta^* + \inv{V}Z - \inv{V}H\theta^*
\end{align*}
where we used the fact that $y = X\theta^* + \eta$ (and $\eta_s$ is the
noise in the reward at round $s$) and we defined $V \defeq \transp{X}X
+ H$ and $Z\defeq X^T\eta$.

\begin{lemma}\label{lemma:subgaussian-z}
  Under \cref{assumption:subgaussian-noise} and for any
$\lambda\in\Real^d$, the random variable $Z = \transp{X}\eta$ satisfies
  \begin{align*}
    \Ex{\exp(\transp{\lambda}Z)} \le \exp\paren{\norm{G}{\lambda}^2/2}.
  \end{align*}

  \begin{proof}
    We have
    \begin{align*}
      \Ex{\exp(\transp{\lambda}Z)}
      &= \Ex{\exp(\transp{\lambda}\transp{X}\eta)} \\
      &= \Ex[\Big]{\Ex[\Big]{\prod_{s=1}^t \exp\paren{\transp{\lambda}X_s\eta_s} \given \mathcal{F}_t}} \\
      &= \Ex[\Big]{\Ex[\Big]{\exp\paren{\transp{\lambda}X_t\eta_t} \given \mathcal{F}_t} \prod_{s=1}^{t-1} \exp\paren{\transp{\lambda}X_s\eta_s}} \\
      &\le \Ex[\Big]{\exp\paren{{(\transp{\lambda}X_t)}^2/2}
        \prod_{s=1}^{t-1} \exp\paren{\transp{\lambda}X_s\eta_s}}
      &\text{since $\eta_t$ is conditionally 1-subgaussian given $\mathcal{F}_t$}\\
      &\le \Ex[\Big]{\prod_{s=1}^t\exp({(\transp{\lambda}X_s)}^2/2)}
      &\text{similarly conditioning on $\mathcal{F}_{t-1},\mathcal{F}_{t-2},\dotsc$}\\
      &= \Ex[\Big]{\exp\paren[\Big]{\sum_{s=1}^t\frac{\transp{\lambda}X_s\transp{X_s}\lambda}{2}}} \\
      &= \exp\paren[\Big]{\frac{1}{2}\transp{\lambda}\transp{X}X\lambda}
        = \exp(\norm{G}{\lambda}^2/2)
        \qedhere
    \end{align*}
  \end{proof}
\end{lemma}


\begin{lemma}\label{lemma:z-norm-bounded-whp}
  Suppose for any $\lambda\in\Real^d$ the random variable
  $Z=\transp{X}\eta$ satisfies
  $\Ex{\exp(\transp{\lambda}Z)} \le \exp(\norm{G}{\lambda}^2/2)$ (see,
  for example, \cref{lemma:subgaussian-z}).  Let
  $0 \prec H \in \Real^{d\times d}$ be any symmetric positive definite
  matrix.  Then for any $0 < \delta \le 1$, we have
  \begin{align*}
    \Prob[\Bigg]{\norm{\inv{(G+H)}}{Z} \ge \sqrt{2\log\frac{1}{\delta} + \log\frac{\det(G+H)}{\det H}}} &\le \delta.
  \end{align*}

  \begin{proof}
    We define
    \begin{align*}
      M_\lambda &\defeq \exp\paren[\Big]{\transp{\lambda}Z - \frac{1}{2}\transp{\lambda}G\lambda}.
    \end{align*}
    Now consider $\lambda\sim\mathcal{N}(0, \inv{H})$ to be a random
    variable with the density function $h(\lambda)$.  Define
    \begin{align*}
      \bar{M}
      &\defeq \int M_\lambda \,h(\lambda)\,d\lambda \\
      &= \frac{1}{\sqrt{{(2\pi)}^d \det\inv{H}}} \int\exp\paren[\Big]{\transp{\lambda}Z
        - \frac{1}{2}\transp{\lambda}G\lambda
        - \frac{1}{2}\transp{\lambda}H\lambda
        } \,d\lambda.
    \end{align*}
    We complete the square in the integrand:
    \begin{align*}
      \transp{\lambda}Z - \frac{1}{2}\transp{\lambda}G\lambda - \frac{1}{2}\transp{\lambda}H\lambda
      &= \frac{1}{2}\transp{Z}\inv{(G+H)}Z - \frac{1}{2}\transp{(\lambda - \inv{(G+H)}Z)}(G+H)(\lambda-\inv{(G+H)}Z) \\
      &= \frac{1}{2} \norm{\inv{(G+H)}}{Z}^2 - \frac{1}{2}\norm{G+H}{\lambda-\inv{(G+H)}Z}^2
    \end{align*}
    which we substitute into the previous equation to get
    \begin{align*}
      \bar{M}
      &= \frac{\exp\paren[\big]{\frac{1}{2} \norm{\inv{(G+H)}}{Z}^2}}{\sqrt{{(2\pi)}^d \det\inv{H}}}
        \int \exp\paren[\Big]{-\frac{1}{2} \norm{G+H}{\lambda-\inv{(G+H)}Z}^2} \,d\lambda \\
      &= \sqrt{\frac{\det H}{\det(G+H)}} \exp\paren[\Big]{\frac{1}{2}\norm{\inv{(G+H)}}{Z}^2}, \\
      \log\bar{M} &= \frac{1}{2}\norm{\inv{(G+H)}}{Z}^2 + \frac{1}{2}\log\frac{\det H}{\det(G+H)}.
    \end{align*}
    Our assumption gives us
    \begin{align*}
      \Ex{\exp(\transp{\lambda}Z)} &\le \exp\paren[\Big]{\frac{1}{2}\transp{\lambda}G\lambda} \\
      \Ex[\bigg]{\frac{\exp(\transp{\lambda}Z)}{\exp(\frac{1}{2}\transp{\lambda}G\lambda)}}
      &= \Ex{M_\lambda} \le 1, &\text{for all }\lambda\in\Real^d.
    \end{align*}
    Now, by Fubini's theorem we can exchange the expectation with the
    integral over $\lambda$:
    \begin{align*}
      \Ex{\bar{M}} &= \Ex[\Big]{\int M_\lambda\,h(\lambda)\,d\lambda} = \int \Ex{M_\lambda} \,h(\lambda)\,d\lambda \le 1
    \end{align*}
    and use a Chernoff bound:
    \begin{align*}
      \Prob{\log(\bar{M}) \ge u} &\le \exp(-u) \\
      \Prob[\Big]{\frac{1}{2}\norm{\inv{(G+H)}}{Z}^2 \ge u + \frac{1}{2}\log\frac{\det(G+H)}{\det H}} &\le \exp(-u) \\
      \Prob[\Big]{\norm{\inv{(G+H)}}{Z} \ge \sqrt{2u + \log\frac{\det(G+H)}{\det H}}} &\le \exp(-u).
    \end{align*}
    Substituting $u = \log(1/\delta)$ completes the proof.
  \end{proof}

\end{lemma}

%===============================================================================
% \hrule

% We see that
% \begin{align*}
%   \max_{\lambda\in\Real^d} \exp\paren[\Big]{\transp{\lambda}Z - \frac{1}{2}\transp{\lambda}G\lambda}
%   &= \exp\paren[\Big]{\max_{\lambda\in\Real^d} \transp{\lambda}Z - \frac{1}{2}\transp{\lambda}G\lambda}.
% \end{align*}
% The maximizer of this expression is given by taking the gradient and
% setting it to zero:
% \begin{align*}
%   \nabla_\lambda\brck[\Big]{\transp{\lambda}Z - \frac{1}{2}\transp{\lambda}G\lambda}_{\lambda=\lambda^*} = Z - G\lambda^* &= 0 \\
%   \lambda^* &= \inv{G}Z
% \end{align*}
% and thus
% \begin{align*}
%   \max_{\lambda\in\Real^d} \exp\paren[\Big]{\transp{\lambda}Z - \frac{1}{2}\transp{\lambda}G\lambda}
%   &= \exp\paren[\Big]{\transp{Z}\inv{G}Z - \frac{1}{2}\transp{Z}\inv{G}Z} \\
%   &= \exp\paren[\Big]{\frac{1}{2}\norm{\inv{G}}{Z}^2}.
% \end{align*}
% We define
% \begin{align*}
%   M_\lambda \defeq \exp\paren[\Big]{\transp{\lambda}Z - \frac{1}{2}\transp{\lambda}G\lambda}
% \end{align*}
% and use Chernoff's bound:
% \begin{align*}
%   \Pr\paren[\Big]{\frac{1}{2}\norm{\inv{G}}{Z}^2 > u}
%   &= \Pr\paren[\Big]{\max_{\lambda\in\Real^d}M_\lambda > u} \\
%   &\le \exp(-u) \Ex[\Big]{\max_{\lambda\in\Real^d} M_\lambda}.
% \end{align*}
% \hrule
%===============================================================================

\subsection{Ridge Regression}
\label{sec:ridge-regression}

We will now instantiate a concrete algorithm based on \emph{ridge
  regression}, i.e.\ using the regularizer $H=\rho I$ for some
$\rho>0$.  Thus we will have
\begin{align*}
  V_t &\defeq \rho I + \sum_{s=1}^t X_s\transp{X_s}\\
  \shortintertext{and}
  \hat{\theta}_t - \theta^* &= \inv{V_t}Z_t - \inv{V_t}H\theta^* \\
                          &= \inv{V_t}Z_t - \rho\inv{V_t}\theta^* \\
  V_t^{1/2}(\hat{\theta}_t - \theta^*) &= V_t^{-1/2}Z - \rho V_t^{-1/2}\theta^* \\
  \norm{V_t}{\hat{\theta}_t - \theta^*} &= \norm{}{V_t^{-1/2}Z - \rho V_t^{-1/2}\theta^*} \\
  &\le \norm{\inv{V_t}}{Z} + \rho\norm{\inv{V_t}}{\theta^*}.\
                          &\text{Triangle inequality} \\
  &\le \norm{\inv{V_t}}{Z} + \sqrt\rho \norm{}{\theta^*} &\text{Since } V_t \succeq \rho I.
\end{align*}
We now use \cref{lemma:subgaussian-z,lemma:z-norm-bounded-whp} in our
confidence bound:
\begin{align*}
  \Prob[\Bigg]{\norm{V_t}{\hat{\theta}_t - \theta^*} > \sqrt\rho\norm{}{\theta^*} + \sqrt{2\log\frac{1}{\delta} + \log\frac{\det V_t}{\det \rho I}}}
  &\le \delta.
\end{align*}
We assume that $\norm{}{\theta^*} \le S$.  We use the union bound,
replacing $\delta$ with $\delta/n$ to get a bound that holds uniformly
at every round:
\begin{align*}
  \sqrt{\beta_t} &= \sqrt\rho S + \sqrt{2\log\frac{n}{\delta} + \log\frac{\det V_t}{\det \rho I}}.
\end{align*}
Applying \cref{thm:linucb-regret} gives us the regret bound
\begin{align*}
  \widehat{R}_n
  &\le \sqrt{8dn\beta_{n-1}\log\frac{\tr V_0 + nL^2}{d\det^{1/d} V_0}}
    = \sqrt{8dn\beta_{n-1}\log\frac{d\rho + nL^2}{d\rho}} \\
  \shortintertext{where}
  \sqrt{\beta_{n-1}}
  &= \sqrt\rho S + \sqrt{2\log\frac{n}{\delta} + \log\frac{\det V_{n-1}}{\det \rho I}} \\
  &\le \sqrt\rho S + \sqrt{2\log\frac{n}{\delta} + d\log\paren*{1 + \frac{nL^2}{d\rho}}}.
\end{align*}
Choosing $\delta=1/n$ and constant $\rho$ gives $\sqrt{\beta_{n-1}} =
O(d^{1/2}\log^{1/2}(n/d))$ and thus the expected regret of Linear UCB
with ellipsoidal confidence sets satisfies
\begin{align*}
  \widehat{R}_n &= O(\beta_n^{1/2}\sqrt{dn\log(n/d)}) = O(d\log(n/d)\sqrt{n}).
\end{align*}

\appendix

\section{Facts About Random Variables}

\begin{definition}[Subgaussian random variable]\label{def:subG}
  A real-valued random variable $X$ is \emph{$\sigma^2$-subgaussian}
  if its moment-generating function satisfies:
  \begin{align*}
    \mgf_X(s) \defeq \Ex{\exp(sX)} &\le \exp(s^2\sigma^2/2).
  \end{align*}
\end{definition}

\begin{definition}[Subexponential random variable]\label{def:subExp}
  A real-valued random variable $X$ is \emph{$\lambda$-subexponential}
  if its moment-generating function satisfies:
  \begin{align*}
    \mgf_X(s) \defeq \Ex{\exp(sX)} &\le \exp(s^2\lambda^2/2),
    &\text{for all } \abs{s}\le 1/\lambda.
  \end{align*}
\end{definition}

\begin{definition}[Subgaussian/subexponential random vector]
  A random vector $X\in\Real^b$ is subgaussian (subexponential) if the
  random variable $\innerp{v}{X}$ is subgaussian (subexponential) for
  any $v\in\Real^n$ with $\norm{}{v}=1$.
\end{definition}

\begin{definition}[Subgaussian/subexponential random matrix]
  A random matrix $W\in\Real^{n\times m}$ is subgaussian (subexponential)
  if the random vector $Wu\in\Real^n$ is subgaussian
  (subexponential) for any $u\in\Real^m$ with $\norm{}{u}=1$.
\end{definition}

\section{Proofs}

\begin{lemma}[{\citealp[Lemma~9]{AbbasiYadkoriImprovedAlgorithmsLinear2011}}]\label{lemma:martingale-mgf}
  Let $X_1, X_2, \dotsc \in \Real$ and
  $\sigma_1, \sigma_2, \dotsc \in \Real$ be sequences of random
  variables and let
  $\mathcal{F}_1 \subset \mathcal{F}_2 \subset \dotsb$ be a
  filtration.  Suppose that each $X_t \in \mathcal{F}_t$ and almost
  surely
  $\Ex{\exp(\lambda X_t)\given \mathcal{F}_{t-1}} \le
  \exp(\sigma_t^2\lambda^2/2)$ (i.e. each
  $\sigma_t \in \mathcal{F}_{t-1}$ and $X_t$ is conditionally
  $\sigma_t$-subgaussian given $\mathcal{F}_{t-1}$).  For any
  $\lambda\in\Real$, define
  \begin{align*}
    M_{t,\psi} &\defeq \exp\paren[\bigg]{\sum_{s=1}^t \psi X_s - \frac{1}{2}\sigma_s^2\psi^2}.
  \end{align*}
  Let $\tau$ be a stopping time with respect to
  ${(\mathcal{F}_t)}_{t=1}^\infty$.  Then $M_{\tau,\psi}$ is almost surely
  well-defined and $\Ex{M_{\tau,\psi}} \le 1$.

  \begin{proof}
    We will show that ${(M_{t,\psi})}_{t=1}^\infty$ is a
    supermartingale.  Let
    \begin{align*}
      D_{t,\psi} &\defeq \exp\paren[\bigg]{\psi X_t - \frac{1}{2}\sigma_t^2\psi^2}.
    \end{align*}
    By the conditional subgaussianity of $X_t$, we have
    \begin{align*}
      \Ex{D_{t,\psi}\given\mathcal{F}_{t-1}}
      &= \Ex{\exp(\psi X_t)\given\mathcal{F}_{t-1}}
        \cdot \exp\paren[\bigg]{-\frac{1}{2}\sigma_t^2\psi^2}
      \le \exp\paren[\bigg]{\frac{1}{2}\sigma_t^2\psi^2}
        \cdot \exp\paren[\bigg]{-\frac{1}{2}\sigma_t^2\psi^2}
      = 1.
    \end{align*}
    It is also clear that $D_{t,\psi}$ is $\mathcal{F}_t$-measurable
    (since $X_t,\sigma_t \in \mathcal{F}_t$), and therefore so is
    $M_{t,\psi}$.  Furthermore,
    \begin{align*}
      \Ex{M_{t,\psi}\given\mathcal{F}_{t-1}}
      &= \Ex{M_{t-1,\psi}\cdot D_{t,\psi}\given\mathcal{F}_{t-1}}
        = M_{t-1,\psi} \Ex{D_{t,\psi}\given\mathcal{F}_{t-1}}
        \le M_{t-1,\psi},
    \end{align*}
    showing that $M_{t,\psi}$ is indeed a supermartingale and therefore
    by Doob's optional stopping theorem
    $\Ex{M_{\tau,\psi}} \le \Ex{M_{0,\psi}} = 1$.

    Now, we argue that $M_{\tau,\psi}$ is well-defined.  By the
    convergence theorem for non-negative supermartingales,
    $M_{\infty,\psi} = \lim_{t\to\infty}M_{t,\psi}$ is almost surely
    well-defined.  Hence, $M_{\tau,\psi}$ is almost surely
    well-defined independently of whether $\tau<\infty$ holds or not.
    Next, we show that $\Ex{M_{\tau,\psi}} \le 1$.  For this, let
    $Q_{t,\psi} \defeq M_{\min\{\tau,t\},\psi}$ be a stopped
    version of ${(M_{t,\psi})}_t$.  By Fatou's lemma,
    $\Ex{M_{\tau,\psi}} = \Ex{\liminf_{t\to\infty}Q_{t,\psi}} \le
    \liminf_{t\to\infty}\Ex{Q_{t,\psi}} \le 1$, showing the
    $\Ex{M_{\tau,\psi}} \le 1$ indeed holds.
  \end{proof}
\end{lemma}

\begin{lemma}
  Let $n\in\Nat$ and $\varepsilon>0$ and $\sigma^2>0$.  Let
  $X_1,X_2,\dotsc,X_n\in\Real$ and
  $\sigma_1,\sigma_2,\dotsc,\sigma_n\in\Real$ be sequences of random
  variables and let
  $\mathcal{F}_1 \subset \mathcal{F}_2 \subset \dotsb \subset
  \mathcal{F}_n$ be a filtration.  Suppose that each
  $X_t\in\mathcal{F}_t$ and almost surely each $\sigma_t \le \sigma$
  and each $X_t$ is conditionally $\sigma_t$-subgaussian given
  $\mathcal{F}_{t-1}$.  Then
  \begin{align*}
    \Prob[\Big]{\exists t \le n. \sum_{s=1}^t X_s \ge
    \sqrt{2\gamma_nV_t\log(N/\delta)}}
    &\le \delta,
  \end{align*}
  where
  \begin{align*}
    V_t &\defeq \max\set[\Big]{\varepsilon,\sum_{s=1}^t \sigma_s^2}, &
    \gamma_n &\defeq 1 + \frac{1}{\log(n)}, &
    N &\defeq 1 + \ceil*{\frac{\log(n\sigma^2/\varepsilon)}{\log(\gamma_n)}}.
  \end{align*}

  \begin{proof}
    For $\psi\in\Real$ define
    \begin{align*}
      M_{t,\psi} &\defeq \exp\paren[\bigg]{\sum_{s=1}^t \psi X_s - \frac{1}{2}\psi^2\sigma_s^2}.
    \end{align*}
    If $\tau \le n$ is a stopping time with respect to $\mathcal{F}$,
    then $\Ex{M_{\tau,\psi}} \le 1$ by \cref{lemma:martingale-mgf}.
    Therefore by Markov's inequality,
    \begin{align*}
      \delta/N
      &\ge \Prob{M_{\tau,\psi} \ge N/\delta} \\
      &= \Prob[\bigg]{\exp\paren[\Big]{\sum_{s=1}^\tau \psi X_s - \frac{\psi^2\sigma_s^2}{2}} \ge N/\delta} \\
      &= \Prob[\bigg]{\psi \sum_{s=1}^\tau X_s - \frac{\psi\sigma_s^2}{2} \ge \log(N/\delta)} \\
      &\ge \Prob[\bigg]{\sum_{s=1}^\tau X_s \ge \frac{\log(N/\delta)}{\psi} + \frac{\psi V_\tau}{2}},
      &\text{since } V_\tau \ge \sum_{s=1}^\tau \sigma_s^2.
    \end{align*}
    To get the tightest tail bound, we want to minimize the quantity
    on the right side of the inequality; it attains a minimum of
    $\sqrt{2V_\tau \log(1/\delta)}$ at
    $\psi_{\min} \defeq \sqrt{2\log(N/\delta)/V_\tau}$.  Furthermore,
    for any $\psi\ge\psi_{\min}$ that quantity is at most
    $\psi V_\tau$.

    However, since $V_\tau$ is a random quantity, we cannot simply set
    $\psi \defeq \psi_{\min}$.  Instead, we recognize that
    $V_\tau \in [\varepsilon, n\sigma^2]$ almost surely, and we can
    logarithmically cover this range with the $N$ values
    $\varepsilon\gamma_n^{k-1}$ for $k=1,2,\dotsc,N$; at least one of
    these must lie in the interval $[V_\tau/\gamma_n, V_\tau]$.  Then
    we can define the corresponding values
    \begin{align*}
      \psi_k &\defeq \sqrt{\frac{2\log(N/\delta)}{\varepsilon\gamma_n^{k-1}}},
              &\text{for } k=1,2,\dotsc,N;
    \end{align*}
    there must be some value of $k$ for which
    $\psi_k \in [\psi_{\min}, \gamma_n\psi_{\min}]$.  Using a union bound
    over these $N$ values, we get
    \begin{align*}
      \delta
      &\ge \Prob[\bigg]{\exists k\in[N]. \sum_{s=1}^\tau X_s \ge \frac{\log(N/\delta)}{\psi_k} + \frac{\psi_k V_\tau}{2}} \\
      &\ge \Prob[\bigg]{\sum_{s=1}^\tau X_s \ge \gamma_n\psi_{\min}V_\tau} \\
      &= \Prob[\bigg]{\sum_{s=1}^\tau X_s \ge \sqrt{2\gamma_nV_\tau\log(N/\delta)}}.
    \end{align*}
    To complete the proof, define the stopping time $\tau =
    \min\set{n,\tau_n}$ where
    \begin{align*}
      \tau_n &\defeq \min\set[\Big]{t \le n \given \sum_{s=1}^tX_s \ge \sqrt{2\gamma_nV_t\log(N/\delta)}}.
              \qedhere
    \end{align*}
  \end{proof}
\end{lemma}


\begin{claim}[{\citealp[Theorem~7.8]{ZhangMatrixTheory2011}}]
  If $A \succeq B \succeq 0$, then
  \begin{enumerate}
  \item $\rank(A) \ge \rank(B)$
  \item $\det A \ge \det B$
  \item $\inv{B} \succeq \inv{A}$ if $A$ and $B$ are nonsingular.
  \end{enumerate}
\end{claim}

\bibliographystyle{plainnat}
\bibliography{references}

\end{document}

%%% Local Variables:
%%% mode: latex
%%% TeX-master: t
%%% End:
