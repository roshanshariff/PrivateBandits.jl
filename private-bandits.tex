\documentclass{article}
\usepackage[utf8]{inputenc}

% if you need to pass options to natbib, use, e.g.:
% \PassOptionsToPackage{numbers, compress}{natbib}
% before loading nips_2018
\PassOptionsToPackage{round,colon}{natbib}

% ready for submission
\usepackage{nips_2018}

% to compile a preprint version, e.g., for submission to arXiv, add
% add the [preprint] option:
% \usepackage[preprint]{nips_2018}

% to compile a camera-ready version, add the [final] option, e.g.:
% \usepackage[final]{nips_2018}

% to avoid loading the natbib package, add option nonatbib:
% \usepackage[nonatbib]{nips_2018}

\title{Differentially Private Contextual Linear Bandits}
\author{
  Roshan Shariff
  \And
  Or Sheffet
}

\usepackage{newtxtext}
%\usepackage[semibold]{libertine} % a bit lighter than Times--no osf in math
\usepackage[T1]{fontenc} % best for Western European languages
%\usepackage{textcomp} % required to get special symbols
%\usepackage[varqu,varl]{inconsolata} % a typewriter font must be defined
\usepackage{amsmath,mathtools}
\usepackage{amsthm,thmtools,thm-restate}
\usepackage{amssymb} % loads amsfonts
\usepackage{nicefrac}
%\usepackage[libertine,cmintegrals,bigdelims,vvarbb]{newtxmath} %
%replaces some amssymb symbols
\usepackage{newtxmath}
\usepackage[scr=rsfso]{mathalfa} % Use rsfso to provide mathscr
\usepackage{dsfont}
%\usepackage{upgreek}
\usepackage{bm} % load after all math to give access to bold math

\usepackage{microtype}
\usepackage[inline,shortlabels]{enumitem}
\usepackage{setspace}
\usepackage{tikz-cd}
\usepackage{booktabs}
\usepackage[boxed]{algorithm}
\usepackage{algpseudocode}
\usepackage{multicol}
%\usepackage{fullpage}
\usepackage{color}
\newcommand{\os}[1]{\textcolor{red}{Or's comment:~\textbf{#1}}}
%\usepackage{dblfloatfix}

%% Bibliography/References
%\usepackage[round,colon]{natbib}

%% Cross-references
\usepackage{varioref}
\usepackage[hidelinks]{hyperref}
\usepackage[capitalise]{cleveref}

%% To-do notes
\usepackage[obeyFinal]{todonotes}

% \newcommand{\tinytodo}[2][]{\todo[size=\tiny]{#2}}
\newcommand{\tinytodo}[2][]{\todo[size=\tiny, #1]{\begin{spacing}{1.0}#2\end{spacing}}}
\newcommand{\RStodo}[2][]{\tinytodo[color=red!20, #1]{R:\@#2}} % Roshan
\newcommand{\OStodo}[2][]{\tinytodo[color=blue!20, #1]{Cs: #2}} % Or
\newcommand{\fix}{\marginpar{FIX}}
\newcommand{\new}{\marginpar{NEW}}

%% Macros

\renewcommand{\vec}[1]{\bm{#1}}
\newcommand{\wildcard}{\mathinner{\,{\cdot}\,}}
\newcommand{\defeq}{\coloneq}
\newcommand{\eqdef}{\eqcolon}
\newcommand{\inv}[1]{#1^{-1}}
\newcommand{\Real}{\mathds{R}}
\newcommand{\Nat}{\mathds{N}}
\newcommand{\Int}{\mathds{Z}}
\newcommand{\mgf}{\mathrm{mgf}}
\newcommand{\UCB}{\operatorname{UCB}}
\renewcommand\mid{\mathinner{\vert}}
\DeclareMathOperator*{\argmin}{arg\,min}
\DeclareMathOperator*{\argmax}{arg\,max}
\DeclareMathOperator{\tr}{tr}
\DeclareMathOperator{\rank}{rank}

\newcommand\given[1][\delimsize]{%
  \providecommand{\delimsize}{}
  \nonscript\:#1\vert\allowbreak\nonscript\:\mathopen{}
}

\DeclarePairedDelimiter{\abs}||
\DeclarePairedDelimiter{\paren}()
\DeclarePairedDelimiter{\brck}{[}{]}
\DeclarePairedDelimiterX{\set}[1]\lbrace\rbrace{#1}
\DeclarePairedDelimiter{\floor}\lfloor\rfloor
\DeclarePairedDelimiter{\ceil}\lceil\rceil
\DeclarePairedDelimiterX{\innerp}[2]\langle\rangle{#1,#2}
\DeclarePairedDelimiterXPP{\Prob}[1]{\mathds{P}}(){}{#1}
\DeclarePairedDelimiterXPP{\PrSet}[1]{\mathds{P}}\{\}{}{#1}
\DeclarePairedDelimiterXPP{\Ex}[1]{\mathds{E}}{[}{]}{}{#1}
\DeclarePairedDelimiterXPP{\Exx}[2]{\mathds{E}_{#1}}{[}{]}{}{#2}
\DeclarePairedDelimiterXPP{\Var}[1]{\mathrm{Var}}{[}{]}{}{#1}
\DeclarePairedDelimiterXPP{\One}[1]{\mathds{1}}\{\}{}{#1}
\DeclarePairedDelimiterXPP{\norm}[2]{}\Vert\Vert{_{#1}}{#2}

%% Other symbols
\newcommand{\A}{\mathcal{A}}
\newcommand{\C}{\mathcal{C}}
\newcommand{\E}{\mathcal{E}}
\providecommand\transp{\top}
\let\transpsymbol\transp
\renewcommand{\transp}[1]{#1^\transpsymbol}
\newcommand{\Aset}[1]{\mathcal{A}_{#1}}
\newcommand{\Dset}[1]{\mathcal{D}_{#1}}
\newcommand{\Cset}[1]{\mathcal{C}_{#1}}
\newcommand{\Eset}[1]{\mathcal{E}_{#1}}
\newcommand{\scrF}{\mathscr{F}}
\newcommand{\Wishart}{\mathcal{W}}
\newcommand{\Normal}{\mathcal{N}}
\newcommand{\Eye}[1]{\bm{I}_{#1 \times #1}}
\newcommand{\XtX}[1]{\transp{#1}{#1}}

% Theorem environments

\declaretheorem[style=plain]{theorem}
\declaretheorem[style=plain,sibling=theorem]{lemma}
\declaretheorem[style=plain,sibling=theorem]{corollary}
\declaretheorem[style=remark,sibling=theorem]{claim}
\declaretheorem[style=definition]{definition}
\declaretheorem[style=definition]{assumption}
\declaretheorem[style=remark,numbered=no]{remark}

\Crefname{assumption}{Assumption}{Assumptions}

%%%%%%%%%%%%%%%%%%%%%%%%%%%%%%%%%%%%%%%%%%%%%%%%%%%%%%%%%%%%%%%%%%%%%%%%%%%
%\onehalfspacing

\begin{document}

\maketitle

\begin{abstract}
  We study the contextual linear bandit problem, where an algorithm
  sequentially selects actions that maximize an associated reward,
  which may depend on a provided per-round \emph{context}.  In
  particular, the reward is assumed to be a linear function of a
  \emph{feature vector} that encodes the context and selected action,
  with added stochastic noise.

  In many real-world applications, it is important to maintain the
  privacy of the provided contexts and rewards.  Specifically, the
  algorithm's choice of actions should not reveal too much information
  about the previous contexts and rewards it has learned from.  We
  address this concern by providing a \emph{differentially private}
  algorithm for the contextual linear bandit problem, along with
  bounds on its performance.
\end{abstract}

\section{Introduction}
\label{sec:introduction}

The \emph{stochastic multi-armed bandit} (MAB) is a well-known
sequential decision-making task: in every round $t$, an algorithm
chooses an action (or arm) $a_t\in\A$ and receives a corresponding
reward $r_t$ drawn iid from a distribution associated with the action
$a_t$.  The objective is to maximize the cumulative reward
$\sum_t r_t$ by learning from the past rewards associated with the
different actions.  The \emph{contextual} bandit problem is an
extension of MAB, where each day we are also given an arbitrary
context $c_t\in\C$, and the reward is now drawn from a distribution
that depends on \emph{both} the context and the selected action.

% Since the agent only observes the reward for actions it takes, this
% is an online learning task with partial information.  Learning the
% rewards associated with all the arms requires an agent to
% \emph{explore} them.  Too much exploration, however, would be
% counter-productive; eventually the algorithm should \emph{exploit}
% the best arms it has found.  The main challenge of bandit strategies
% is to balance exploration and exploitation.

As a motivating example, consider the problem of online
advertisements.  The user provides a context (composed of query word,
past history of clicks, website visits etc), and in response the site
surfaces a potential ad and receives a reward if the user clicks the
ad.  Ignoring the context and modeling the problem as a standard MAB
(with each possible ad as an action) suffers from the drawback of
ignoring the variety of users' preferences; whereas modeling the
problem as an independent learning task per user prevents
generalization between users.  As a result, we model the problem as a
contextual linear-bandit problem.  Based on the user-given context,
each ad (action) is mapped to some feature vector; yet the
click-probability is modeled the same way across all users --- it is
assumed to be proportional to the inner-product between the action's
feature vector and some target vector $\vec\theta^*$.

The above example motivates the need for privacy in the contextual bandit setting: users' interests, web site visits, and clicks are
sensitive personal information, yet they are all required for making a reasonable prediction.
% An algorithm may pick an ad for a
% user based on their own information, and it may even learn from
% aggregated information to choose better in the future, but an
% adversary analyzing these ad choices must not be able to learn too
% much about any individual user.
In this work, we give the first upper- and lower-bounds for the
problem of (joint) \emph{differentially private} contextual linear
bandits.  We present algorithms for the contextual bandit problem that
are \emph{differentially private}, the de-facto gold-standard of
privacy preserving data-analysis in both academia and industry.
Standard differential privacy (under continuous observation with
event-level privacy) states the output of the algorithm does not
change in a statistically significant way when any single context and
reward is changed.  However, as illustrate below, in the contextual
bandit setting differential private states we must essentially ignore
each day's context and thus incur linear regret.  We therefore adopt
the more relaxed notion of \emph{joint differential privacy}
\citep{KearnsMechanismDesign2014} which allows for the output of round
$t$ to depend arbitrarily on the context of round $t$, yet have only
limited dependency with the context or reward of any other round
$t'$.  Intuitively, we allow the $t$-th user to observe ads
corresponding to her preferences, without revealing (much) about those
preferences to all users at times $t'>t$.\footnote{The guarantee of
  differential privacy under continuous observation assures that the
  entire sequence of actions and rewards preserve privacy.  Hence, even
  if all later user collude in an effort to learn user $t$'s context,
  they still have very limited advantage over a random guess.}

\todo[inline]{Consider non-contextual bandits; only the rewards are
  private.  Does this help with the counter-example against optimism
  in Tor's paper?}

\todo[inline]{Looks like a $d^2$ dependence is necessary: consider
  $\vec\theta^*=\paren{\pm 1/d, \dotsc}$ and the actions are $e_i$.  Then
  you need $d^2\log d$ samples?\\Why? I need further explanation here.}

\subsection{Background and Problem Formulation}
\label{subsec:intro_background}

\paragraph{Stochastic Contextual Linear Bandits.}
The classic MAB problem is a sequential decision making task in which, at
every round $t$, an algorithm selects an action $a_t$ from a fixed set $\A$.
%Usually, $\A = \set{1,\dotsc,K}$ (the $K$-armed bandit) but we do not
%require this; our algorithms can handle other action sets that may be
%infinite or uncountable.
The algorithm then receives a \emph{reward}
corresponding to the action it chose.  In the (stationary)
\emph{stochastic multi-armed bandit}, the reward is of the form $r_t =
\mu_{a_t} + \eta_t$, where $\mu_a$ is the expected reward of action $a$
and $\eta_t$ is zero-mean noise.
%In other words, the reward is sampled independently from a distribution that depends only on the selected action.

In the contextual bandit problem, the algorithm also receives a
\emph{context} $c_t\in\C$ at the beginning of each round, and the
reward is assumed to depend on both the reward and the context.  To
take advantage of the context while choosing actions, we will assume
that the context affects the reward in a linear way, in the following
sense.  Suppose there is a known \emph{mapping}
$\phi:\Cset{}\times\Aset{}\to\Real^d$ that maps every context-action
pair to a $d$-dimensional \emph{feature vector}.  We will assume that
the reward is a linear function of the feature vector, with some added
noise: $r_t = \innerp{\vec\theta^*}{\phi(c_t,a_t)} + \eta_t$ for some
unknown vector $\vec\theta^*\in\Real^d$. In fact, we simplify the
problem statement to include the \emph{decision set}
$\Dset{t} \defeq \set{\phi(c_t,a)\given a\in\Aset{}} \subset \Real^d$
(where the context is implicitly encoded in the decision set), and so,
rather than choosing an action out of $\Aset{}$ the algorithm chooses
some $x_t\in\Dset{t}$.  Thus, the contextual stochastic linear bandit
framework is composed of repeated rounds in which
\begin{enumerate}
\item The learner receives an arbitrarily chosen \emph{decision set} $\mathcal{D}_t \subset
  \Real^d$.
\item The learner chooses an \emph{action} $\vec x_t \in \mathcal{D}_t$.
\item The learner receives a stochastic \emph{reward}
  $y_t = \innerp{\vec\theta^*}{\vec x_t} + \eta_t$.
\end{enumerate}
The vector $\vec\theta^*\in\Real^d$ is the key unknown parameter of the
environment which the agent aims to learn so that she can maximize reward. The goal of the learner is thus to minimize her expected regret...\os{Define the objective: minimizing expected regret --- perhaps justify why we chose this objective (standard, implies similar objectives etc)}

% \begin{assumption}[Subgaussian noise]\label{assumption:subgaussian-noise}
%   We denote by
%   $\mathcal{F}_t =
%   \sigma(\mathcal{D}_1,X_1,Y_1,\dotsc,\mathcal{D}_{t-1},X_{t-1},Y_{t-1},\mathcal{D}_t,X_t)$
%   all the information available just before the noise $\eta_t$ is
%   observed.  We assume that $\eta_t$ is \emph{conditionally
%     $\sigma^2$-subgaussian:}
%   $
%     \Ex{\exp(\lambda\eta_t)\given \mathcal{F}_t} \le \exp(\lambda^2\sigma^2/2)$,
%     for all  $\lambda\in\Real$
% \end{assumption}

\paragraph{Joint Differential Privacy.}
As discussed above, the context and reward may be considered private
information about the users which we wish to keep private from all \emph{other} users. We thus introduce the notion of joint differentially private learners under continuous observation \citep[a combination of the two definitions given in][]{KearnsMechanismDesign2014,DworkContinualObservation2010}. First, we say two sequences $S = \langle (\mathcal{D}_1, Y_1), (\mathcal{D}_2, Y_2), ..., (\mathcal{D}_T, Y_T) \rangle$ and $S' = \langle (\mathcal{D}'_1, Y'_1), ..., (\mathcal{D}'_T, Y'_T) \rangle$ are \emph{$t$-neighbors} if for all $t'\neq t$ it holds that $(\mathcal{D}_t,Y_t) = (\mathcal{D}'_t, Y_t)$.
\os{maybe omit?} We say $S$ and $S'$ are \emph{neighbors} if there exists a $t$ such that the two sequences are $t$-neighbors.
% For example, a
% search engine might have a context consisting of a user's search
% query, identity, interests, and physical location, while the reward
% indicates which search result the user clicked on.  The search engine
% should, of course, use the context to answer each query; furthermore,
% it should learn from the reward to better respond to future queries
% from other users.  However, it should also maintain privacy: its
% responses to queries should not reveal \emph{too much} information
% about the context and rewards it has learned from.  More precisely, we
% want algorithms that are \emph{jointly differentially private} in the
% following sense:

\begin{definition}
\label{def:JDP_continual_observation}
  A randomized algorithm $A$ for the contextual bandit problem is
  \emph{$(\varepsilon,\delta)$-jointly differentially private} (JDP) under continual observation if for any $t$ and any pair of $t$-neighboring sequences $S$ and $S'$, and any subset ${\cal S}_{>t} \subset \mathcal{D}_{t+1} \times \mathcal{D}_{t+2} \times \cdots \times \mathcal{D}_{T}$ of sequence of actions ranging from day $t+1$ to the end of the sequence, it holds that $\Pr[A(S)\in \mathcal{S}_{>t}] \leq e^\varepsilon\Pr[A(S')\in \mathcal{S}'_{>t}] +\delta$.
\end{definition}
Note that the standard definition of differential privacy requires the distribution proximity to hold for any subset of sequences of actions ranging from day $t$ to the end of sequence. However, in our problem formulation, with given decision sets, this notion isn't not even well-defined --- as the decision set of day $t$ is different under $S$ and under $S'$. Therefore, when we discuss the impossibility of regret-minimization under standard differential privacy, we result back to the setting of different contexts with fixed action set. See further details in Section~\ref{sec:lower_bounds}.

\subsection{Our Contribution}
\label{sec:contributions}

In this work, in addition to formulating the definition of JDP under
continual observation, we also present a framework for implementing
JDP algorithms for the contextual bandit problem. Not surprisingly,
our framework combines the tree-based algorithm of.... with the linear
upper-confidence bound (LinUCB) algorithm.  However, for modularity,
we choose to analyze a family of linear UCB algorithms where in each
day one uses a different regularizer, under the premise that all
regularizers' singular values are bounded. \os{above and below?!?}
Moreover, we repeat our analysis twice. Once in a general framework,
obtaining an upper bound on the regret which is proportional to
$\sqrt T$; and once in an instance dependent setting, where in each
day there's a significant gap between the best arm and any other
arms. (The analysis carries through to $k-1$ gaps between the $i$th
arm and the leading one, yet its result is too cumbersome so we omit
this analysis.)  Our leading application of course is privacy, though
one could postulate other reasons why such a change in regularizers
would be useful (e.g., updating the regularizer when an initial upper
bound of a parameter turns out to be wrong). We then plug in two
particular regularizers into our scheme: one based on additive Wishart
noise \citep{SheffetPrivateApproxRegression2015} which always results
in a PSD; and one based on additive Gaussian
noise \citep{DworkAnalyzeGauss2014} which doesn't result in a PSD, so
we shift it by $\rho I$ so that w.h.p.\ (over all days in the sequence)
the shifted eigenvalues of the noise are all positive. We also present
empirical evidence showing our analysis is in fact optimal.

We also give lower bound for the $\varepsilon$-differentially private
MAB problem. Whereas all previous work on the private MAB problem uses
the standard (non-private) bounds, we show that the regret of
\emph{any} private algorithm must incur \emph{an additional} cost of
$\log(T)/\varepsilon$. This resembles the lower bound of the
adversarial setting, however, the proof technique cannot rely on the
standard packing arguments \citep[see, for
example,][]{HardtTalwarGeometryDP2010} seeing as the input for the
problem is stochastic rather than adversarial.  Instead, we apply a
new result of \citet{KarwaVadhanFiniteSampleDP2017} showing, in essence,
that the effective group-privacy between two possible inputs of $T$
samples drawn iid from distributions $P$ and $Q$ respectively is about
$n\cdot d_{\rm TV}(P,Q)$.



\todo[inline]{Complete this section.}
\section{Preliminaries and Background}
\label{sec:background}

Differential privacy was first introduced by
\citet{DworkCalibratingNoiseSensitivity2006} and its modern
formulation in \citet{DworkDifferentialPrivacy2006}.
\Citet{DworkAlgorithmicFoundationsDifferential2014} is a comprehensive
overview of the field.

\todo[inline]{Finish this section.}


\todo[inline]{Finish this section.}


\subsection{Differentially Private Linear Regression}
\label{sec:dp-regression}

\begin{multicols}{2}[\subsection{Notation}\label{sec:notation}]
  \nolinenumbers
  \begin{description}[style=sameline,leftmargin=5em]
  \item[$n$] horizon, i.e. number of rounds
  \item[$s,t$] indices of rounds
  \item[$d$] dimensionality of action space
  \item[$\Dset{t}\subset\Real^d$] decision set at round $t$
  \item[$\vec x_t \in \Dset{t}$] action at round $t$
  \item[$y_t \in \Real$] reward at round $t$
  \item[$\vec X_t \in \Real^{t\times d}$] matrix with $\vec X_s = \transp{\vec
      x_s}$ for $s\in[t]$
  \item[$\vec y_t \in \Real^t$] vector of rewards $\vec y_s = y_s$
  \item[$\vec\theta^* \in \Real^d$] unknown parameter vector
  \item[$\norm{V}{\vec x}$] $\defeq \sqrt{\transp{\vec x} V \vec x}$
  \end{description}
  \linenumbers
\end{multicols}

\section{Differentially Private Linear UCB}

Our differentially private contextual linear bandit algorithm is based
on the well-studied LinUCB, an adaptation of the Upper Confidence
Bound (UCB) algorithm to stochastic linear bandits
\citep{DaniStochasticLinearOptimization2008,RusmevichientongLinearlyParameterizedBandits2010,AbbasiYadkoriImprovedAlgorithmsLinear2011}.
At every round, LinUCB constructs a \emph{confidence set} $\E_t$
containing the unknown parameter vector $\vec\theta^*$ with high
probability.  It then computes a UCB for the reward of each action in
the decision set $\Dset{t}$, and ``optimistically'' chooses the action
with the highest UCB:
\begin{align}\label{eq:def-ucb}
  \vec x_t &\gets \argmax_{\vec x\in\Dset{t}} \UCB_t(\vec x),
  &\text{where }
  \UCB_t(\vec x) &\defeq \max_{\vec\uptheta\in\E_t} \innerp{\vec\uptheta}{\vec x}.
\end{align}
With the rewards being linear with added subgaussian noise (i.e.,
$y_s = \innerp{\vec\theta^*}{X_s} + \eta_s$ for $s=1,\dotsc,t$) it is
natural to centre the confidence set $\E_{t+1}$ on the (regularized)
linear regression estimate:
\begin{align*}
  \hat{\vec\theta}_t &\defeq \min_{\hat{\vec\theta} \in \Real^d} \norm{}{\vec X_t \hat{\vec\theta} - \vec y_t}^2 + \norm{H_t}{\hat{\vec\theta}}^2
  = \inv{(\transp{\vec X_t} \vec X_t + H_t)}\transp{\vec X_t} \vec y_t.
\end{align*}
The matrix $V_t \defeq G_t + H_t \in \Real^{d\times d}$ is the regularized
version of the Gram matrix $G_t \defeq \transp{\vec X_t} \vec X_t$.  As
with LinUCB, we use ellipsoidal confidence sets of the form
\begin{align}\label{eq:def-ellip}
  \E_{t+1} &\defeq \set{\vec\theta\in\Real^d \given \norm{\inv{V_t}}{\vec\theta-\hat{\vec\theta}_t}^2 \le \beta_t},
\end{align}
where $\beta_t$ scales the size of the ellipsoid and will be calculated
later to yield high-probability confidence sets.

We now discuss our modifications to LinUCB to achieve differential
privacy.  Notice that the history of actions and rewards up to round
$t$ is used only via the confidence set $\E_{t+1}$, which is to say
the Gram matrix $G_t$ and the vector
$\vec u_t \defeq \transp{\vec X_t} \vec y_t$ (these also determine $\beta_t$,
as we will see).  By recording this history with differential privacy,
we obtain a linear UCB algorithm that is also differentially private
because it simply post-processes $G_t$ and $\vec u_t$
\citep[see][Proposition~2.1]{DworkAlgorithmicFoundationsDifferential2014}.

Our first modification is to use a regularizer matrix $H_t$ that
changes at each round, which to our knowledge has not been considered
in previous work.  If $H_t$ is a noise matrix appropriately chosen to
ensure differential privacy, then $V_t$ becomes a ``privacy-preserving
approximation'' of $G_t$.  Similarly, our second modification
maintains the privacy of the observed rewards (not just the action
vectors): we replace $\vec u_t \defeq \transp{\vec X_t} \vec y_t$ with
a private approximation $\tilde{\vec u}_t = \vec u_t + \vec h_t$.
These two modifications result in \cref{alg:linucb}, with confidence
sets $\E_{t+1}$ centered at
$\tilde{\vec\theta}_t \defeq \inv{V_t} \tilde{\vec u}_t$.

\begin{algorithm}
  \caption{Linear UCB with changing regularizers}\label{alg:linucb}
  \begin{algorithmic}
    \For{each round $t \in 1,2,\dotsc,n$}
    \State $\Dset{t} \gets{}$ decision set ${} \subset \Real^d$.
    \State $V_{t-1} \gets G_{t-1} + H_{t-1}$
    \Comment{privacy-preserving ${} \approx G_{t-1}$}
    \State $\tilde{\vec u_{t-1}} \gets \vec u_{t-1} + \vec h_{t-1}$
    \Comment{privacy-preserving ${} \approx \transp{X}y$}
    \State $\tilde{\vec\theta}_{t-1} \gets \inv{V_{t-1}}\tilde{\vec u}_{t-1}$
    \Comment{privacy-preserving ${} \approx \hat{\vec\theta}_{t-1}$}
    \For{each action $\vec x \in \Dset{t}$}
    \State $\UCB_t(\vec x) \gets \innerp{\tilde{\vec\theta}_{t-1}}{\vec x} +
    \sqrt{\beta_{t-1}}\norm{\inv{V_{t-1}}}{\vec x}$
    \Comment{closed form of \eqref{eq:def-ucb} for ellipsoids \eqref{eq:def-ellip}}
    \EndFor
    \State $\vec x_t \gets \argmax_{\vec x\in\Dset{t}} \UCB_t(\vec x)$
    \Comment{Choose action.}
    \State $y_t \gets {}$ reward for action $\vec x_t$
    \Comment{Observe reward.}
    \State $G_t \gets G_{t-1} + \vec x_t \transp{\vec x_t},
    \quad \vec u_t \gets \vec u_{t-1} + \vec x_t y_t$
    \Comment{Record in privacy-preserving history.}
    \EndFor
  \end{algorithmic}
\end{algorithm}

In the rest of this section, we will consider how to choose $H_t$ and
$\vec h_t$ to ensure differential privacy.  For now, we satisfy
ourselves with the following claim, which is a straightforward
application of the post-processing lemma of differential privacy
\citep[Proposition~2.1]{DworkAlgorithmicFoundationsDifferential2014}:
\begin{claim}
  If the sequence $(V_t,\tilde{\vec u}_t)_{t=1}^{n-1}$ is
  $(\varepsilon,\delta)$-differentially private with respect to
  $(\vec x_t, y_t)_{t=1}^{n-1}$, then \cref{alg:linucb} is
  $(\varepsilon,\delta)$-jointly differentially private.
\end{claim}

\begin{remark}
  \Cref{alg:linucb} is only jointly differentially private even though
  the history maintains full differential privacy.  This is because
  the chosen action depends not only on the past contexts $c_s$
  ($s < t$, via the differentially private $\vec X_{t-1}$) but also
  on the current context $c_t$ via the decision set $\Dset{t}$.  Since
  this use of $c_t$ is not differentially private, it is revealed by
  the algorithm's chosen $\vec x_t$ (as discussed in
  \cref{sec:priv-contextual}).
\end{remark}

\subsection{Regret Bounds}
\label{sec:regret-bounds}

The utility of \cref{alg:linucb} comes from it having a small regret,
which depends on constructing accurate confidence sets that
nevertheless tightly concentrate around the unknown $\vec\theta^*$.
In other words, the $\beta_t$ should be as small as possible while
accounting for the noise $\eta_t$ and the perturbations caused by
$H_t$ and $\vec h_t$.  We start by giving a ``generic'' regret bound
that holds if the confidence sets are accurate.

\begin{restatable}[Regret of \Cref{alg:linucb}]{theorem}{ThmLinUCBRegret}
  \label{thm:linucb-regret}%
  Suppose the mean rewards are bounded:
  $\abs{\innerp{\vec\theta^*}{\vec x}} \le B$ for all $\vec x \in \bigcup_t\Dset{t}$;
  and $\bar\beta_n \ge \max\set{\beta_0,\dotsc,\beta_{n-1},1}$.  In
  the event that for all rounds $t\in\set{1,\dotsc,n}$ the confidence
  sets $\Eset{t}$ contain $\vec\theta^*$ (i.e.,
  $\norm{V_{t-1}}{\vec\theta^* - \tilde{\vec\theta}_{t-1}} \le
  \sqrt{\beta_{t-1}}$) and each $V_{t-1} = G_{t-1} + H_{t-1}$ for some
  symmetric $H,H_0,\dotsc,H_{n-1} \in \Real^{d\times d}$ with all
  $H_{t-1} \succeq H \succ 0$, the pseudo-regret of \cref{alg:linucb}
  \begin{align*}
    \widehat R_n
    &\defeq \sum_{t=1}^n \max_{\vec x\in\Dset{t}} \innerp{\vec x}{\vec\theta^*}
      - \innerp{\vec x_t}{\vec\theta*} \\
    \shortintertext{satisfies}
    \widehat R_n
    &\le 2B\sqrt{2n \bar\beta_n \log\frac{\det V_n}{\det H}}. \\
    \shortintertext{Furthermore, if each $\norm{}{\vec x_t} \le L$}
    \widehat R_n
    &\le 2B\sqrt{2dn \bar\beta_n \log\frac{\tr H + nL^2}{d\det^{1/d} H}}.
  \end{align*}
\end{restatable}

The following results will help us calculate how large the $\beta_t$
must be to ensure that the confidence sets contain $\vec\theta^*$ with
high probability.

\begin{lemma}
  \label{lemma:conf-ellip-size}%
  Suppose $\vec y_t = \vec X_t \vec\theta^* + \vec\eta_t$ and let
  $\vec z_t \defeq \transp{\vec X_t}\vec\eta$.  Then
  \begin{align*}
    \norm{V_t}{\vec\theta^* - \tilde{\vec\theta}_t}
    &\le \norm{\inv{V_t}}{\vec z_t} + \norm{H_t}{\vec\theta^*}
      + \norm{\inv{H_t}}{\vec h_t}.
  \end{align*}

  \begin{proof}
    By definition, $\tilde{\vec\theta}_t = \inv{V_t} \tilde{\vec u}_t$,
    $\tilde{\vec u}_t = \vec u_t + \vec h_t$, and $\vec u_t = \transp{\vec
      X_t} \vec y_t$, so that
    \begin{align*}
      \vec\theta^* - \tilde{\vec\theta}_t
      &= \vec\theta^* - \inv{V_t}(\transp{\vec X_t} \vec y_t + \vec h_t) \\
      &= \vec\theta^* - \inv{V_t}(\transp{\vec X_t} \vec X_t \vec\theta^*
        + \transp{\vec X_t} \vec\eta_t + \vec h_t)
      &\text{since } \vec y_t = \vec X_t \vec\theta^* + \vec\eta_t \\
      &= \vec\theta^* - \inv{V_t}(V_t \vec\theta^* - H_t \vec\theta^* + \vec z_t + \vec h_t)
      &\text{defining } \vec z_t \defeq \transp{\vec X_t} \vec\eta_t \\
      &= \inv{V_t}(H_t\vec\theta^* - \vec z_t - \vec h_t) \\
      \shortintertext{Multiplying both sides by $V_t^{1/2}$ gives}
      V_t^{1/2}(\vec\theta^* - \tilde{\vec\theta}_t)
      &= V_t^{-1/2}(H_t\vec\theta^* - \vec z_t - \vec h_t) \\
      \norm{V_t}{\vec\theta^* - \tilde{\vec\theta}_t}
      &= \norm{\inv{V_t}}{H_t\vec\theta^* - \vec z_t - \vec h_t}
      &\text{applying $\norm{}{\wildcard}$ to both sides}\\
      &\le \norm{\inv{V_t}}{H_t\vec\theta^*} + \norm{\inv{V_t}}{z_t}
        + \norm{\inv{V_t}}{\vec h_t} &\text{triangle inequality} \\
      &\le \norm{\inv{V_t}}{\vec z_t} + \norm{\inv{H_t}}{H_t\vec\theta^*}
        + \norm{\inv{H_t}}{\vec h_t}
      &\text{since $\inv{V_t} \preceq \inv{H_t}$ by \cref{claim:psd-matrix-props}} \\
      &= \norm{\inv{V_t}}{\vec z_t} + \norm{H_t}{\vec\theta^*} +
        \norm{\inv{H_t}}{\vec h_t}.
      &\qedhere
    \end{align*}
  \end{proof}
\end{lemma}

\begin{lemma}
  \label{lemma:z-tails}%
  For every round $t$, suppose the noise $\eta_t$ in the reward is
  conditionally $\sigma^2$-subgaussian given all the action vectors
  and previous rewards, i.e.
  \begin{align*}
    \Ex{\exp(\lambda\eta_t) \given \vec x_1, y_1, \dotsc, \vec x_{t-1}, y_{t-1}, \vec x_t}
    &\le \exp(\lambda^2\sigma^2/2), &\text{for all } \lambda\in\Real.
  \end{align*}
  Let $\vec z_t \defeq \transp{\vec X_t} \vec\eta_t$, $H\succeq 0$ be a
  symmetric matrix, and $\delta>0$.  Then with probability at least
  $1-\delta$ for all rounds $t$,
  \begin{align*}
    \norm{\inv{(G_t + H)}}{\vec z_t}
    &\le \sigma\sqrt{2\log\frac{1}{\delta} + \log\frac{\det(G_t + H)}{\det H}}. \\
    \shortintertext{In the further event that $H_t \succeq H$ for all rounds $t$,}
    \norm{\inv{V_t}}{\vec z_t}
    &\le \sigma\sqrt{2\log\frac{1}{\delta} + \log\frac{\det V_t}{\det H}}. \\
    \shortintertext{Lastly, if all $\norm{}{\vec x_t} \le L$,}
    \norm{\inv{V_t}}{\vec z_t}
    &\le \sigma\sqrt{2\log\frac{1}{\delta} + d\log\frac{\tr H + nL^2}{d \det^{1/d}H}}.
  \end{align*}

  \begin{proof}
    The first result is essentially a restatement of the
    ``self-normalized bound for vector-valued martingales'' of
    \citet[Theorem~1]{AbbasiYadkoriImprovedAlgorithmsLinear2011}.  The
    second result follows from \cref{claim:psd-matrix-props} when
    $V_t = G_t + H_t \succeq G_t + H$, since
    $\inv{V_t} \preceq \inv{(G_t + H)}$ and
    $\det(G_t + H) \le \det V_t$.  The third result is another
    application of \cref{lemma:elliptical-potential}, as used in
    \cref{thm:linucb-regret}.
  \end{proof}
\end{lemma}

\Cref{lemma:conf-ellip-size} decomposes
$\norm{V_t}{\vec\theta^* - \tilde{\vec\theta}_t}$ into three terms:
\begin{enumerate*}[(i), itemjoin={{; }}, itemjoin*={{; and }}]
\item $\norm{\inv{V_t}}{\vec z_t}$, which is bounded by
  \cref{lemma:z-tails}
\item $\norm{H_t}{\vec\theta^*}$, which we will bound as
  $\sqrt{\norm{}{H_t}}\norm{}{\vec\theta^*}$
\item $\norm{\inv{H_t}}{\vec h_t}$.
\end{enumerate*}
In the following sections, we will derive high-probability bounds for
the latter two terms for specific choices of $H_t$ and $\vec h_t$.

\subsection{Pan-Privacy with Tree-Based Aggregation}
\label{sec:tree-mechanism}

\todo[inline]{Expand this section, keeping the results below.}  For a horizon
of $n$, the pan-private tree mechanism composes $m = \ceil{1+\log_2n}$
mechanisms.  Suppose we want to achieve a target of
$(\varepsilon,\delta)$-differential privacy after composing these $m$
mechanisms.  Then it is sufficient for each mechanism to be
$(\varepsilon',\delta')$-differentially private where:
\begin{align*}
  \varepsilon' &= \frac{\varepsilon}{2\sqrt{2m\ln\frac{1}{\delta - m\delta'}}} \\
  \delta' &< \delta/m.
\end{align*}

\subsection{Differential Privacy via Additive Wishart Noise}
\label{sec:dp-wishart}

To simplify calculations, we construct the matrices
$A_t \defeq \begin{bmatrix} \vec X_t & \vec y_t \end{bmatrix} \in
\Real^{t\times(d+1)}$.  The top-left $d\times d$ submatrix of
$\XtX{A_t}$ is the Gram matrix $G_t$ and the last column contains the
vector $\vec u_t = \transp{\vec X_t}\vec y_t$.  It therefore suffices
to maintain a differentially private approximation of $M_t$.  We have
assumed before that each action vector is bounded:
$\norm{}{\vec x_t} \le L$; further assume that the rewards are also
bounded: $\abs{y_t} \le \tilde B$ for all $t$.  Then each row of $A_t$
is bounded by $\tilde L \defeq \sqrt{L^2 + \tilde B^2}$.

\begin{theorem}%
  \label{thm:wishart-cont-dp}
  Fix $\varepsilon$ and a sufficiently small $\delta$, let $A\in\Real^{n\times p}$ be a
  matrix with each row having $\norm{}{A_{t,:}} \le \tilde L$, and let
  $m \defeq 1 + \ceil{\log_2n}$.  Generate the sequence of matrices
  $(W_t)_{t=1}^n$ using the tree mechanism (\cref{sec:tree-mechanism})
  with Wishart noise distributed according to
  \begin{align*}
    &\Wishart_p(\tilde L^2\Eye{p}, k),
    &\text{ where } k
    &\defeq p + \floor*{\frac{224}{\varepsilon^2} \cdot m\ln^2\paren*{\frac{4m+1}{\delta}}}.
  \end{align*}
  Then releasing the sequence $(\XtX{A_{1:t,:}} + W_t)_{t=1}^n$ is
  $(\varepsilon,\delta)$-differentially private with respect to
  changing a single row of $A$.

  Furthermore, each $W_t$ is distributed identically, but not
  independently, according to $\Wishart_p(\tilde L^2\Eye{p}, mk)$.
\end{theorem}

\todo[inline]{Be explicit about how to get $H_t$ and $\vec h_t$ from $W_t$.}

\begin{corollary}%
  \label{cor:wishart-alg}
  \Cref{alg:linucb} is $(\varepsilon,\delta)$-jointly differentially
  private when run using the differentially private histories
  described in \cref{thm:wishart-cont-dp}.  Now fix
  $\alpha\in(0,1)$ and define
  \begin{align*}
    H
    &\defeq \tilde L^2\paren*{\sqrt{mk} - \sqrt{d} - \sqrt{2\ln\frac{4n}{\alpha}}}^2\Eye{d} \\
    \sqrt{\beta_t}
    &\defeq \sigma\sqrt{2\ln\frac{n}{\alpha} + \log\frac{\det V_t}{\det H}}
      + S \tilde L \paren*{\sqrt{mk} + \sqrt{d} + \sqrt{2\ln\frac{4n}{\alpha}}}
      + \tilde L \sqrt{d + 2\sqrt{d\ln\frac{2n}{\alpha}} + 2\ln\frac{2n}{\alpha}}.
  \end{align*}
  With probability at least $1-\alpha$, the following hold for rounds
  $t = 1,\dotsc,n-1$: $H_t \succeq H$ and
  $\norm{V_t}{\vec\theta^* - \tilde{\vec\theta}_t} \le
  \sqrt{\beta_t}$.
\end{corollary}

\subsection{Differential Privacy via Additive Gaussian Noise}
\label{sec:dp-gauss}

\begin{theorem}%
  \label{thm:analyze-gauss}
  Fix $\varepsilon\in(0,1)$ and $\delta\in(0,1/e)$.  Let
  $A\in\Real^{n\times p}$ be a matrix whose rows have $l_2$-norm
  bounded by $\tilde L$.  Let $N\in\Real^{p\times p}$ be a symmetric
  random Gaussian matrix with each entry having variance
  $\tilde L^4\ln(1/\delta)/\varepsilon^2$;
  i.e.\,$N_{i,j}\sim\Normal(0,\tilde L^4\ln(1/\delta)/\varepsilon^2)$
  independently for $i \le j$ and $N_{i,j} = N_{j,i}$ for $i > j$.
  Then outputting $\XtX{A} + N$ is
  $(\varepsilon,\delta)$-differentially private with respect to
  changing a single row of $A$.
\end{theorem}

\begin{theorem}%
  \label{thm:analyze-gauss-cont}
  Fix $\varepsilon$ and a sufficiently small $\delta$, let
  $A\in\Real^{n\times p}$ be a matrix with each row having
  $\norm{}{A_{t,:}} \le \tilde L$, and let
  $m \defeq 1 + \ceil{\log_2n}$.  Generate the sequence of matrices
  $(N_t)_{t=1}^n$ using the tree mechanism (\cref{sec:tree-mechanism})
  with symmetric Gaussian noise; i.e., having equal $(i,j)$ and
  $(j,i)$ entries independently drawn (for each $i \le j$) from the
  distribution
  \begin{align*}
    &\Normal(0, \sigma^2),
    &\text{ where } \sigma^2
    &\defeq \frac{\tilde L^4 \ln(1/\delta)}{\varepsilon^2}.
  \end{align*}
  Then releasing the sequence $(\XtX{A_{1:t,:}} + N_t)_{t=1}^n$ is
  $(\varepsilon,\delta)$-differentially private with respect to
  changing a single row of $A$.

  Furthermore, each entry of $N_t$ is distributed identically, but not
  independently, according to $\Normal(0, m\sigma^2)$.
\end{theorem}


\bibliographystyle{plainnat}
\bibliography{references,zotero-references}

\newpage

\appendix

\section{Supplementary Material}

\subsection{Proof of \Cref{thm:linucb-regret}}

\ThmLinUCBRegret*

\begin{lemma}\label{lemma:linucb-regret}
  Suppose the mean rewards are bounded:
  $\abs{\innerp{\vec\theta^*}{\vec x}} \le B$ for all $\vec x\in\bigcup_t\Dset{t}$;
  all the $V_t \succ 0$ are symmetric; and
  $\bar\beta_n \ge \max\set{\beta_0,\dotsc,\beta_{n-1},1}$.  In the
  event that for all rounds $t\in\set{1,\dotsc,n}$ the confidence sets
  $\Eset{t}$ contain $\vec\theta^*$ (i.e.,
  $\norm{V_{t-1}}{\vec\theta^* - \tilde{\vec\theta}_{t-1}} \le \sqrt{\beta_{t-1}}$),
  the pseudo-regret of \cref{alg:linucb} satisfies
  \begin{align*}
    \widehat{R}_n &\le \sqrt{4n\bar\beta_n \sum_{t=1}^n \min\set{B^2,
                   \norm{\inv{V_{t-1}}}{\vec x_t}^2}}.
  \end{align*}
\end{lemma}

\begin{proof}
  At every round $t$, \cref{alg:linucb} selects the ``optimistic''
  action $\vec x_t$ satisfying
  \begin{align}\label{eq:optimistic-action}
    (\vec x_t,\bar{\vec\theta}_t)
    &= \argmax_{\vec x\in\Dset{t},\vec\theta\in\Eset{t}} \innerp{\vec\theta}{\vec x}.
  \end{align}
  Let $\vec x_t^* = \argmax_{\vec x\in\Dset{t}}\innerp{\vec\theta^*}{\vec x}$ be an optimal
  action and $r_t = \innerp{\vec\theta^*}{\vec x_t^* - \vec x_t}$ be the immediate
  pseudo-regret suffered for round $t$:
  \begin{align*}
    r_t &= \innerp{\vec\theta^*}{\vec x_t^*} - \innerp{\vec\theta^*}{\vec x_t} \\
        &\le \innerp{\bar{\vec\theta}_t}{\vec x_t} - \innerp{\vec\theta^*}{\vec x_t}
        &\text{from \eqref{eq:optimistic-action} since }
          \vec x_t^*\in\Dset{t}, \vec\theta^*\in\Eset{t} \\
        &= \innerp{\bar{\vec\theta}_t - \vec\theta^*}{\vec x_t} \\
        &= \innerp{V_{t-1}^{1/2}(\bar{\vec\theta}_t - \vec\theta^*)}{V_{t-1}^{-1/2} \vec x_t} \\
        &\le \norm{V_{t-1}}{\bar{\vec\theta}_t - \vec\theta^*}\norm{\inv{V_{t-1}}}{\vec x_t}
        &\text{by Cauchy-Schwarz}\\
        &\le \paren[\big]{\norm{V_{t-1}}{\bar{\vec\theta}_t - \tilde{\vec\theta}_{t-1}}
          + \norm{V_{t-1}}{\vec\theta^* - \tilde{\vec\theta}_{t-1}}}
          \norm{\inv{V_{t-1}}}{\vec x_t}
        &\text{by the triangle inequality}\\
        &\le 2\sqrt{\beta_{t-1}}\norm{\inv{V_{t-1}}}{\vec x_t}
        &\text{since } \bar{\vec\theta}_t, \vec\theta^* \in \Eset{t} \\
        &\le 2\sqrt{\bar\beta_n}\norm{\inv{V_{t-1}}}{\vec x_t}
        &\text{since } \bar\beta_n \ge \beta_{t-1}.
  \end{align*}
  From our assumptions that the mean absolute reward is bounded by $B$
  and $\bar\beta_n \ge 1$, we also get that
  $r_t \le 2B \le 2B\sqrt{\bar\beta_n}$.  Putting these together,
  \begin{align*}
    r_t &\le 2\sqrt{\bar\beta_n} \min\set{B, \norm{\inv{V_{t-1}}}{\vec x_t}}.
  \end{align*}
  Now we apply Jensen's inequality as follows:
  \begin{align*}
    \widehat R_n^2 &= n^2 \paren[\Big]{\sum_{t=1}^n \frac{r_t}{n}}^2
                   \le n^2 \sum_{t=1}^n \frac{r_t^2}{n} = n\sum_{t=1}^nr_t^2 \\
    \widehat R_n &\le \sqrt{n\sum_{t=1}^n r_t^2}
                  = \sqrt{4n \bar\beta_n \sum_{t=1}^n\min\set{B^2,
                  \norm{\inv{V_{t-1}}}{\vec x_t}^2}}.
                  \qedhere
  \end{align*}
\end{proof}

\begin{lemma}[Elliptical Potential]\label{lemma:elliptical-potential}
  Let $\vec x_1,\dotsc,\vec x_n \in \Real^d$ be bounded vectors with each
  $\norm{}{\vec x_t} \le L$, $U_0\in\Real^{d\times d}$ a positive
  definite matrix, and $B \ge 0$.  Define
  $U_t \defeq U_0 + \sum_{s=1}^t \vec x_s \transp{\vec x_s}$ for all
  $t\in\set{1,\dotsc,n}$.  Then
  \begin{align*}
    \sum_{t=1}^n\min\set{B^2, \norm{\inv{U_{t-1}}}{\vec x_t}^2}
    &\le 2B^2 \log\frac{\det U_n}{\det U_0}
      \le 2B^2d \log\frac{\tr U_0+nL^2}{d\det^{1/d} U_0}.
  \end{align*}
\end{lemma}

\begin{proof}
  We begin by deriving the first inequality with $B=1$, using the fact
  that $\min\set{1, u} \le 2\log(1+u)$ for any $u \ge 0$:
  \begin{align*}
    \sum_{t=1}^n\min\set{1, \norm{\inv{U_{t-1}}}{\vec x_t}^2}
    &\le 2 \sum_{t=1}^n \log(1 + \norm{\inv{U_{t-1}}}{\vec x_t}^2).
  \end{align*}
  We will show that this last summation is $2\log(\det U_n/\det
  U_0)$.  For $t \ge 1$ we have
  \begin{align*}
    U_t &= U_{t-1} + \vec x_t \transp{\vec x_t}
         = U_{t-1}^{1/2}
         \paren[\big]{I + U_{t-1}^{-1/2} \vec x_t \transp{\vec x_t} U_{t-1}^{-1/2}}
         U_{t-1}^{1/2} \\
    \det U_t &=\det U_{t-1}
              \det\paren[\big]{I +
              U_{t-1}^{-1/2} \vec x_t \transp{\vec x_t}
              U_{t-1}^{-1/2}}.
  \end{align*}
  Consider the eigenvectors of the matrix $I + \vec y \transp{\vec y}$
  for an arbitrary vector $\vec y \in \Real^d$.  We know that $\vec y$
  itself is an eigenvector with eigenvalue $1+\norm{}{\vec y}^2$:
  \begin{align*}
    (I + \vec y \transp{\vec y}) \vec y
    &= \vec y + \vec y \innerp{\vec y}{\vec y} = (1+\norm{}{\vec y}^2)\vec y.
  \end{align*}
  Moreover, since $I + \vec y \transp{\vec y}$ is symmetric, every
  other eigenvector $\vec u$ is orthogonal to $\vec y$, so that
  \begin{align*}
    (I + \vec y \transp{\vec y}) \vec u
    &= \vec u + \vec u \innerp{\vec y}{\vec u} = \vec u.
  \end{align*}
  Therefore the only eigenvalues of $I + \vec y \transp{\vec y}$ are
  $1+\norm{}{\vec y}^2$ (with eigenvector $\vec y$) and 1.  In our
  case $\vec y = U_{t-1}^{-1/2} \vec x_t$ and
  $\norm{}{\vec y}^2 = \transp{\vec x_t} \inv{U_{t-1}} \vec x_t =
  \norm{\inv{U_{t-1}}}{\vec x_t}^2$, so we get our first inequality:
  \begin{align*}
    \det U_n &= \det U_0 \prod_{t=1}^n(1 + \norm{\inv{U_{t-1}}}{\vec x_t}^2) \\
    2\log\frac{\det U_n}{\det U_0}
            &= 2\sum_{t=1}^n\log(1+\norm{\inv{U_{t-1}}}{\vec x_t}^2).
  \end{align*}
  For $B \neq 1$, we apply the above result with each $\vec x_t$ scaled by
  $B^{-1}$ and $U_0,\dotsc,U_n$ scaled by $B^{-2}$:
  \begin{align*}
    \sum_{t=1}^n\min\set{B^2,\norm{\inv{U_{t-1}}}{\vec x_t}^2}
    &= B^2 \sum_{t=1}^n\min\set{1,\norm{\inv{U_{t-1}}}{B^{-1} \vec x_t}^2}
      \le 2B^2 \log\frac{\det(B^{-2}U_n)}{\det(B^{-2}U_0)}.
  \end{align*}
  To get the second inequality, we apply the arithmetic-geometric
  mean inequality to the eigenvalues $\lambda_i$ of $U_n$:
  \begin{align*}
    \det U_n &= \prod_{i=1}^d \lambda_i
              \le \paren[\Big]{\frac{1}{d} \sum_{i=1}^d \lambda_i}^d
              = \paren{(1/d)\tr U_n}^d
              \le \paren{(\tr U_0 + nL^2)/d}^d \\
    2B^2 \log\frac{\det U_n}{\det U_0}
            &\le 2B^2d \log\frac{\tr U_0 + nL^2}{d\det^{1/d}U_0}
              \qedhere
  \end{align*}
\end{proof}

\begin{proof}[Proof of \Cref{thm:linucb-regret}]
  From our assumptions and \cref{lemma:linucb-regret}, we know that
  the Linear UCB regret is bounded by
  \begin{align*}
    \widehat{R}_n &\le \sqrt{4n \bar\beta_n \sum_{t=1}^n\min\set{B^2,
                   \norm{\inv{V_{t-1}}}{\vec x_t}^2}}.
  \end{align*}
  Define $U_t \defeq G_t + H$, so that $U_t \preceq V_t$ and by
  \cref{claim:psd-matrix-props} it follows that
  $\inv{V_t} \preceq \inv{U_t}$, which means
  $\norm{\inv{V_{t-1}}}{\vec x_t}^2 \le \norm{\inv{U_{t-1}}}{\vec x_t}^2$.  We
  now apply \cref{lemma:elliptical-potential} to the sequence $U_t$ to
  get
  \begin{align*}
    \sum_{t=1}^n\min\set{B^2,\norm{\inv{V_{t-1}}}{\vec x_t}^2}
    &\le \sum_{t=1}^n\min\set{B^2,\norm{\inv{U_{t-1}}}{\vec x_t}^2} \\
    &\le 2B^2 \log\det\frac{U_n}{U_0} \\
    &\le 2B^2d \log\frac{\tr U_0 + nL^2}{d \det^{1/d} U_0}.
  \end{align*}
  To get the required bounds, it only remains to see that $U_0 = H$ and
  that $\det U_n \le \det V_n$ (again by
  \cref{claim:psd-matrix-props}).
\end{proof}

\section{Privacy Proofs}

\begin{theorem}[{\citealp[Theorem~4.1]{SheffetPrivateApproxRegression2015}}]%
  \label{thm:wishart-dp}%
  Fix $\varepsilon\in(0,1)$ and $\delta\in(0,1/e)$.  Let
  $A\in\Real^{n\times p}$ be a matrix whose rows have $l_2$-norm
  bounded by $\tilde L$.  Let $W$ be a matrix sampled from the
  $d$-dimensional Wishart distribution with $k$ degrees of freedom
  using the scale matrix $\tilde L^2\Eye{p}$ (i.e.\
  $W \sim \Wishart_p(\tilde L^2\Eye{p}, k)$) for
  $k \ge p + \floor[\big]{\frac{14}{\varepsilon^2}\cdot 2\log(4/\delta)}$.
  Then outputting $\XtX{A} + N$ is
  $(\varepsilon,\delta)$-differentially private with respect to
  changing a single row of $A$.
\end{theorem}

\section{Useful Results}

\begin{claim}[{\citealp[Theorem~7.8]{ZhangMatrixTheory2011}}]%
  \label{claim:psd-matrix-props}%
  If $A \succeq B \succeq 0$, then
  \begin{enumerate}[nolistsep]
  \item $\rank(A) \ge \rank(B)$
  \item $\det A \ge \det B$
  \item $\inv{B} \succeq \inv{A}$ if $A$ and $B$ are nonsingular.
  \end{enumerate}
\end{claim}

\begin{claim}[{\citealp[Corollary to
    Lemma~1,][p.~1325]{LaurentAdaptiveEstimation2000}}]%
  \label{claim:chi2-tails}
  If $U\sim\chi^2(d)$ and $\alpha\in(0,1)$,
  \begin{align*}
    \Prob[\bigg]{U \ge d + 2\sqrt{d\ln\frac{1}{\alpha}} + 2\ln\frac{1}{\alpha}} &\le \alpha,
    & \Prob[\bigg]{U \le d - 2\sqrt{d\ln\frac{1}{\alpha}}} &\le \alpha.
  \end{align*}
\end{claim}

\begin{claim}[{\citealp[Adaptation
    of][Corollary~5.35]{VershyninRandomMatrices2010}}]%
  \label{claim:gaussian-matrix-tails}
  Let $A$ be an $n\times d$ matrix whose entries are independent
  standard normal variables.  Then for every $\alpha\in(0,1)$, with
  probability at least $1-\alpha$ it holds that
  \begin{align*}
    \sigma_{\min}(A), \sigma_{\max}(A) &\in \sqrt{n} \pm (\sqrt{d} + \sqrt{2\ln(2/\alpha)})
  \end{align*}
\end{claim}

\begin{claim}[{\citealp[Lemma~A.3]{SheffetPrivateApproxRegression2015}}]%
  \label{claim:wishart-tails}
  Fix $\alpha\in(0,1/e)$ and let $W\sim\Wishart_d(V, k)$ with $\sqrt{m}
  > \sqrt{d} + \sqrt{2\ln(2/\alpha)}$.  Then with probability at least
  $1-\alpha$ it holds that for every $j = 1,2,\dotsc,d$:
  \begin{align*}
    \sigma_j(W) &\in \paren*{\sqrt{m} \pm \paren[\Big]{\sqrt{d} + \sqrt{2\ln(2/\alpha)}}}^2 \sigma_j(V).
  \end{align*}
\end{claim}

%%%%%%%%%%%%%%%%%%%%%%%%%%%%%%%%%%%%%%%%%%%%%%%%%%%%%%%%%%%%%%%%%%%%%%%%%%%

\todo[inline]{Remove following material from final version.}

\section{Ellipsoidal Confidence Sets}
\label{sec:ellips-conf-bounds}

Throughout this section we will fix some particular round $t \le n$
and suppress the subscripted $t$.

We will consider confidence sets for the value of $\theta^*$ that are
ellipsoidal with centre $\hat{\theta}$, of the form
$\set{\theta\in\Real^d \given \norm{V}{\theta-\hat{\theta}}^2 \le
  \beta}$.  Then the upper confidence bound of an action $x$ is
\begin{align*}
  \UCB(x) &= \max_{\theta: \norm{V}{\theta-\hat\theta}^2 \le \beta} \innerp{x}{\theta}
\end{align*}
We solve this constrained optimization problem by the method of
Lagrange multipliers:
\begin{align*}
  \mathcal{L}(\theta, \lambda) &= \innerp{x}{\theta} - \lambda(\norm{V}{\theta-\hat\theta}^2 - \beta) \\
  \nabla_\theta\mathcal{L}(\theta, \lambda) &= \transp{x} - \lambda\transp{(\theta-\hat\theta)}V = 0 \\
  \theta &= \hat\theta + \frac{{\inv{V}x}}{\lambda}
  \shortintertext{To find $\lambda$:}
  \beta &= \norm{V}{\theta-\hat\theta}^2
          = \norm[\bigg]{V}{\frac{\inv{V}x}{\lambda}}^2
          = \frac{1}{\lambda^2}\transp{x}\inv{V}x \\
  \lambda &= \norm{\inv{V}}{x} / \sqrt\beta
  \intertext{Which we substitute into the expression above:}
  \UCB(x) &= \innerp[\bigg]{x}{\hat\theta + \frac{\sqrt\beta}{\norm{\inv{V}}{x}}\inv{V}x} \\
          &= \innerp{x}{\hat\theta} + \sqrt\beta \norm{\inv{V}}{x}.
\end{align*}
We have therefore shown that if we construct an ellipsoidal confidence
set with an appropriate $V$ and $\beta$, then the requirements of
\cref{thm:linucb-regret} will be satisfied with the above $\UCB$
index.

Let $X_s \in \Real^d$ be the action vector selected by the algorithm at
time $s$, so that $X$ is a $t \times d$ matrix.  The \emph{Gram matrix} is given by
$G \defeq \transp{X}X \in \Real^{d \times d}$.  Let $y_s$ be the
reward received at time $s$, so that $y\in\Real^t$.

We will take $\hat{\theta}$ to be a regularized solution of the
least-squares system
\begin{align*}
  X\theta &\approx y,
\end{align*}
regularized by the symmetric positive definite matrix $H \succeq 0$:
\begin{align*}
  \hat{\theta} &= \argmin_\theta \frac{1}{2}\norm{}{X\theta - y}^2 + \frac{1}{2}\norm{H}{\theta}^2 \\
               &= \inv{\paren{\transp{X}X + H}} \transp{X}y \\
               &= \inv{\paren{\transp{X}X + H}} \transp{X} \paren{X\theta^* + \eta} \\
               &= \theta^* + \inv{\paren{\transp{X}X + H}}\paren{\transp{X}\eta - H\theta^*} \\
               &= \theta^* + \inv{V}Z - \inv{V}H\theta^*
\end{align*}
where we used the fact that $y = X\theta^* + \eta$ (and $\eta_s$ is the
noise in the reward at round $s$) and we defined $V \defeq \transp{X}X
+ H$ and $Z\defeq X^T\eta$.

\begin{lemma}\label{lemma:subgaussian-z}
  Under \cref{assumption:subgaussian-noise} and for any
$\lambda\in\Real^d$, the random variable $Z = \transp{X}\eta$ satisfies
  \begin{align*}
    \Ex{\exp(\transp{\lambda}Z)} \le \exp\paren{\norm{G}{\lambda}^2/2}.
  \end{align*}

  \begin{proof}
    We have
    \begin{align*}
      \Ex{\exp(\transp{\lambda}Z)}
      &= \Ex{\exp(\transp{\lambda}\transp{X}\eta)} \\
      &= \Ex[\Big]{\Ex[\Big]{\prod_{s=1}^t \exp\paren{\transp{\lambda}X_s\eta_s} \given \mathcal{F}_t}} \\
      &= \Ex[\Big]{\Ex[\Big]{\exp\paren{\transp{\lambda}X_t\eta_t} \given \mathcal{F}_t} \prod_{s=1}^{t-1} \exp\paren{\transp{\lambda}X_s\eta_s}} \\
      &\le \Ex[\Big]{\exp\paren{{(\transp{\lambda}X_t)}^2/2}
        \prod_{s=1}^{t-1} \exp\paren{\transp{\lambda}X_s\eta_s}}
      &\text{since $\eta_t$ is conditionally 1-subgaussian given $\mathcal{F}_t$}\\
      &\le \Ex[\Big]{\prod_{s=1}^t\exp({(\transp{\lambda}X_s)}^2/2)}
      &\text{similarly conditioning on $\mathcal{F}_{t-1},\mathcal{F}_{t-2},\dotsc$}\\
      &= \Ex[\Big]{\exp\paren[\Big]{\sum_{s=1}^t\frac{\transp{\lambda}X_s\transp{X_s}\lambda}{2}}} \\
      &= \exp\paren[\Big]{\frac{1}{2}\transp{\lambda}\transp{X}X\lambda}
        = \exp(\norm{G}{\lambda}^2/2)
      &\qedhere
    \end{align*}
  \end{proof}
\end{lemma}


\begin{lemma}\label{lemma:z-norm-bounded-whp}
  Suppose for any $\lambda\in\Real^d$ the random variable
  $Z=\transp{X}\eta$ satisfies
  $\Ex{\exp(\transp{\lambda}Z)} \le \exp(\norm{G}{\lambda}^2/2)$ (see,
  for example, \cref{lemma:subgaussian-z}).  Let
  $0 \prec H \in \Real^{d\times d}$ be any symmetric positive definite
  matrix.  Then for any $0 < \delta \le 1$, we have
  \begin{align*}
    \Prob[\Bigg]{\norm{\inv{(G+H)}}{Z} \ge \sqrt{2\log\frac{1}{\delta} + \log\frac{\det(G+H)}{\det H}}} &\le \delta.
  \end{align*}

  \begin{proof}
    We define
    \begin{align*}
      M_\lambda &\defeq \exp\paren[\Big]{\transp{\lambda}Z - \frac{1}{2}\transp{\lambda}G\lambda}.
    \end{align*}
    Now consider $\lambda\sim\mathcal{N}(0, \inv{H})$ to be a random
    variable with the density function $h(\lambda)$.  Define
    \begin{align*}
      \bar{M}
      &\defeq \int M_\lambda \,h(\lambda)\,d\lambda \\
      &= \frac{1}{\sqrt{{(2\pi)}^d \det\inv{H}}} \int\exp\paren[\Big]{\transp{\lambda}Z
        - \frac{1}{2}\transp{\lambda}G\lambda
        - \frac{1}{2}\transp{\lambda}H\lambda
        } \,d\lambda.
    \end{align*}
    We complete the square in the integrand:
    \begin{align*}
      \transp{\lambda}Z - \frac{1}{2}\transp{\lambda}G\lambda - \frac{1}{2}\transp{\lambda}H\lambda
      &= \frac{1}{2}\transp{Z}\inv{(G+H)}Z - \frac{1}{2}\transp{(\lambda - \inv{(G+H)}Z)}(G+H)(\lambda-\inv{(G+H)}Z) \\
      &= \frac{1}{2} \norm{\inv{(G+H)}}{Z}^2 - \frac{1}{2}\norm{G+H}{\lambda-\inv{(G+H)}Z}^2
    \end{align*}
    which we substitute into the previous equation to get
    \begin{align*}
      \bar{M}
      &= \frac{\exp\paren[\big]{\frac{1}{2} \norm{\inv{(G+H)}}{Z}^2}}{\sqrt{{(2\pi)}^d \det\inv{H}}}
        \int \exp\paren[\Big]{-\frac{1}{2} \norm{G+H}{\lambda-\inv{(G+H)}Z}^2} \,d\lambda \\
      &= \sqrt{\frac{\det H}{\det(G+H)}} \exp\paren[\Big]{\frac{1}{2}\norm{\inv{(G+H)}}{Z}^2}, \\
      \log\bar{M} &= \frac{1}{2}\norm{\inv{(G+H)}}{Z}^2 + \frac{1}{2}\log\frac{\det H}{\det(G+H)}.
    \end{align*}
    Our assumption gives us
    \begin{align*}
      \Ex{\exp(\transp{\lambda}Z)} &\le \exp\paren[\Big]{\frac{1}{2}\transp{\lambda}G\lambda} \\
      \Ex[\bigg]{\frac{\exp(\transp{\lambda}Z)}{\exp(\frac{1}{2}\transp{\lambda}G\lambda)}}
      &= \Ex{M_\lambda} \le 1, &\text{for all }\lambda\in\Real^d.
    \end{align*}
    Now, by Fubini's theorem we can exchange the expectation with the
    integral over $\lambda$:
    \begin{align*}
      \Ex{\bar{M}} &= \Ex[\Big]{\int M_\lambda\,h(\lambda)\,d\lambda} = \int \Ex{M_\lambda} \,h(\lambda)\,d\lambda \le 1
    \end{align*}
    and use a Chernoff bound:
    \begin{align*}
      \Prob{\log(\bar{M}) \ge u} &\le \exp(-u) \\
      \Prob[\Big]{\frac{1}{2}\norm{\inv{(G+H)}}{Z}^2 \ge u + \frac{1}{2}\log\frac{\det(G+H)}{\det H}} &\le \exp(-u) \\
      \Prob[\Big]{\norm{\inv{(G+H)}}{Z} \ge \sqrt{2u + \log\frac{\det(G+H)}{\det H}}} &\le \exp(-u).
    \end{align*}
    Substituting $u = \log(1/\delta)$ completes the proof.
  \end{proof}

\end{lemma}

%===============================================================================
% \hrule

% We see that
% \begin{align*}
%   \max_{\lambda\in\Real^d} \exp\paren[\Big]{\transp{\lambda}Z - \frac{1}{2}\transp{\lambda}G\lambda}
%   &= \exp\paren[\Big]{\max_{\lambda\in\Real^d} \transp{\lambda}Z - \frac{1}{2}\transp{\lambda}G\lambda}.
% \end{align*}
% The maximizer of this expression is given by taking the gradient and
% setting it to zero:
% \begin{align*}
%   \nabla_\lambda\brck[\Big]{\transp{\lambda}Z - \frac{1}{2}\transp{\lambda}G\lambda}_{\lambda=\lambda^*} = Z - G\lambda^* &= 0 \\
%   \lambda^* &= \inv{G}Z
% \end{align*}
% and thus
% \begin{align*}
%   \max_{\lambda\in\Real^d} \exp\paren[\Big]{\transp{\lambda}Z - \frac{1}{2}\transp{\lambda}G\lambda}
%   &= \exp\paren[\Big]{\transp{Z}\inv{G}Z - \frac{1}{2}\transp{Z}\inv{G}Z} \\
%   &= \exp\paren[\Big]{\frac{1}{2}\norm{\inv{G}}{Z}^2}.
% \end{align*}
% We define
% \begin{align*}
%   M_\lambda \defeq \exp\paren[\Big]{\transp{\lambda}Z - \frac{1}{2}\transp{\lambda}G\lambda}
% \end{align*}
% and use Chernoff's bound:
% \begin{align*}
%   \Pr\paren[\Big]{\frac{1}{2}\norm{\inv{G}}{Z}^2 > u}
%   &= \Pr\paren[\Big]{\max_{\lambda\in\Real^d}M_\lambda > u} \\
%   &\le \exp(-u) \Ex[\Big]{\max_{\lambda\in\Real^d} M_\lambda}.
% \end{align*}
% \hrule
%===============================================================================

\subsection{Ridge Regression}
\label{sec:ridge-regression}

We will now instantiate a concrete algorithm based on \emph{ridge
  regression}, i.e.\ using the regularizer $H=\rho I$ for some
$\rho>0$.  Thus we will have
\begin{align*}
  V_t &\defeq \rho I + \sum_{s=1}^t X_s\transp{X_s}
  \shortintertext{and}
  \hat{\theta}_t - \theta^* &= \inv{V_t}Z_t - \inv{V_t}H\theta^* \\
                          &= \inv{V_t}Z_t - \rho\inv{V_t}\theta^* \\
  V_t^{1/2}(\hat{\theta}_t - \theta^*) &= V_t^{-1/2}Z - \rho V_t^{-1/2}\theta^* \\
  \norm{V_t}{\hat{\theta}_t - \theta^*} &= \norm{}{V_t^{-1/2}Z - \rho V_t^{-1/2}\theta^*} \\
  &\le \norm{\inv{V_t}}{Z} + \rho\norm{\inv{V_t}}{\theta^*}.\
                          &\text{Triangle inequality} \\
  &\le \norm{\inv{V_t}}{Z} + \sqrt\rho \norm{}{\theta^*} &\text{Since } V_t \succeq \rho I.
\end{align*}
We now use \cref{lemma:subgaussian-z,lemma:z-norm-bounded-whp} in our
confidence bound:
\begin{align*}
  \Prob[\Bigg]{\norm{V_t}{\hat{\theta}_t - \theta^*} > \sqrt\rho\norm{}{\theta^*} + \sqrt{2\log\frac{1}{\delta} + \log\frac{\det V_t}{\det \rho I}}}
  &\le \delta.
\end{align*}
We assume that $\norm{}{\theta^*} \le S$.  We use the union bound,
replacing $\delta$ with $\delta/n$ to get a bound that holds uniformly
at every round:
\begin{align*}
  \sqrt{\beta_t} &= \sqrt\rho S + \sqrt{2\log\frac{n}{\delta} + \log\frac{\det V_t}{\det \rho I}}.
\end{align*}
Applying \cref{thm:linucb-regret} gives us the regret bound
\begin{align*}
  \widehat{R}_n
  &\le \sqrt{8dn\beta_{n-1}\log\frac{\tr V_0 + nL^2}{d\det^{1/d} V_0}}
    = \sqrt{8dn\beta_{n-1}\log\frac{d\rho + nL^2}{d\rho}}
  \shortintertext{where}
  \sqrt{\beta_{n-1}}
  &= \sqrt\rho S + \sqrt{2\log\frac{n}{\delta} + \log\frac{\det V_{n-1}}{\det \rho I}} \\
  &\le \sqrt\rho S + \sqrt{2\log\frac{n}{\delta} + d\log\paren*{1 + \frac{nL^2}{d\rho}}}.
\end{align*}
Choosing $\delta=1/n$ and constant $\rho$ gives $\sqrt{\beta_{n-1}} =
O(d^{1/2}\log^{1/2}(n/d))$ and thus the expected regret of Linear UCB
with ellipsoidal confidence sets satisfies
\begin{align*}
  \widehat{R}_n &= O(\beta_n^{1/2}\sqrt{dn\log(n/d)}) = O(d\log(n/d)\sqrt{n}).
\end{align*}




% \section{Facts About Random Variables}

% \begin{definition}[Subgaussian random variable]\label{def:subG}
%   A real-valued random variable $X$ is \emph{$\sigma^2$-subgaussian}
%   if its moment-generating function satisfies:
%   \begin{align*}
%     \mgf_X(s) \defeq \Ex{\exp(sX)} &\le \exp(s^2\sigma^2/2).
%   \end{align*}
% \end{definition}

% \begin{definition}[Subexponential random variable]\label{def:subExp}
%   A real-valued random variable $X$ is \emph{$\lambda$-subexponential}
%   if its moment-generating function satisfies:
%   \begin{align*}
%     \mgf_X(s) \defeq \Ex{\exp(sX)} &\le \exp(s^2\lambda^2/2),
%     &\text{for all } \abs{s}\le 1/\lambda.
%   \end{align*}
% \end{definition}

% \begin{definition}[Subgaussian/subexponential random vector]
%   A random vector $X\in\Real^b$ is subgaussian (subexponential) if the
%   random variable $\innerp{v}{X}$ is subgaussian (subexponential) for
%   any $v\in\Real^n$ with $\norm{}{v}=1$.
% \end{definition}

% \begin{definition}[Subgaussian/subexponential random matrix]
%   A random matrix $W\in\Real^{n\times m}$ is subgaussian (subexponential)
%   if the random vector $Wu\in\Real^n$ is subgaussian
%   (subexponential) for any $u\in\Real^m$ with $\norm{}{u}=1$.
% \end{definition}

\section{Other Proofs}

\begin{lemma}[{\citealp[Lemma~9]{AbbasiYadkoriImprovedAlgorithmsLinear2011}}]\label{lemma:martingale-mgf}
  Let $X_1, X_2, \dotsc \in \Real$ and
  $\sigma_1, \sigma_2, \dotsc \in \Real$ be sequences of random
  variables and let
  $\mathcal{F}_1 \subset \mathcal{F}_2 \subset \dotsb$ be a
  filtration.  Suppose that each $X_t \in \mathcal{F}_t$ and almost
  surely
  $\Ex{\exp(\lambda X_t)\given \mathcal{F}_{t-1}} \le
  \exp(\sigma_t^2\lambda^2/2)$ (i.e. each
  $\sigma_t \in \mathcal{F}_{t-1}$ and $X_t$ is conditionally
  $\sigma_t$-subgaussian given $\mathcal{F}_{t-1}$).  For any
  $\lambda\in\Real$, define
  \begin{align*}
    M_{t,\psi} &\defeq \exp\paren[\bigg]{\sum_{s=1}^t \psi X_s - \frac{1}{2}\sigma_s^2\psi^2}.
  \end{align*}
  Let $\tau$ be a stopping time with respect to
  ${(\mathcal{F}_t)}_{t=1}^\infty$.  Then $M_{\tau,\psi}$ is almost surely
  well-defined and $\Ex{M_{\tau,\psi}} \le 1$.

  \begin{proof}
    We will show that ${(M_{t,\psi})}_{t=1}^\infty$ is a
    supermartingale.  Let
    \begin{align*}
      D_{t,\psi} &\defeq \exp\paren[\bigg]{\psi X_t - \frac{1}{2}\sigma_t^2\psi^2}.
    \end{align*}
    By the conditional subgaussianity of $X_t$, we have
    \begin{align*}
      \Ex{D_{t,\psi}\given\mathcal{F}_{t-1}}
      &= \Ex{\exp(\psi X_t)\given\mathcal{F}_{t-1}}
        \cdot \exp\paren[\bigg]{-\frac{1}{2}\sigma_t^2\psi^2}
      \le \exp\paren[\bigg]{\frac{1}{2}\sigma_t^2\psi^2}
        \cdot \exp\paren[\bigg]{-\frac{1}{2}\sigma_t^2\psi^2}
      = 1.
    \end{align*}
    It is also clear that $D_{t,\psi}$ is $\mathcal{F}_t$-measurable
    (since $X_t,\sigma_t \in \mathcal{F}_t$), and therefore so is
    $M_{t,\psi}$.  Furthermore,
    \begin{align*}
      \Ex{M_{t,\psi}\given\mathcal{F}_{t-1}}
      &= \Ex{M_{t-1,\psi}\cdot D_{t,\psi}\given\mathcal{F}_{t-1}}
        = M_{t-1,\psi} \Ex{D_{t,\psi}\given\mathcal{F}_{t-1}}
        \le M_{t-1,\psi},
    \end{align*}
    showing that $M_{t,\psi}$ is indeed a supermartingale and therefore
    by Doob's optional stopping theorem
    $\Ex{M_{\tau,\psi}} \le \Ex{M_{0,\psi}} = 1$.

    Now, we argue that $M_{\tau,\psi}$ is well-defined.  By the
    convergence theorem for non-negative supermartingales,
    $M_{\infty,\psi} = \lim_{t\to\infty}M_{t,\psi}$ is almost surely
    well-defined.  Hence, $M_{\tau,\psi}$ is almost surely
    well-defined independently of whether $\tau<\infty$ holds or not.
    Next, we show that $\Ex{M_{\tau,\psi}} \le 1$.  For this, let
    $Q_{t,\psi} \defeq M_{\min\{\tau,t\},\psi}$ be a stopped
    version of ${(M_{t,\psi})}_t$.  By Fatou's lemma,
    $\Ex{M_{\tau,\psi}} = \Ex{\liminf_{t\to\infty}Q_{t,\psi}} \le
    \liminf_{t\to\infty}\Ex{Q_{t,\psi}} \le 1$, showing the
    $\Ex{M_{\tau,\psi}} \le 1$ indeed holds.
  \end{proof}
\end{lemma}

\begin{lemma}
  Let $n\in\Nat$ and $\varepsilon>0$ and $\sigma^2>0$.  Let
  $X_1,X_2,\dotsc,X_n\in\Real$ and
  $\sigma_1,\sigma_2,\dotsc,\sigma_n\in\Real$ be sequences of random
  variables and let
  $\mathcal{F}_1 \subset \mathcal{F}_2 \subset \dotsb \subset
  \mathcal{F}_n$ be a filtration.  Suppose that each
  $X_t\in\mathcal{F}_t$ and almost surely each $\sigma_t \le \sigma$
  and each $X_t$ is conditionally $\sigma_t$-subgaussian given
  $\mathcal{F}_{t-1}$.  Then
  \begin{align*}
    \Prob[\Big]{\exists t \le n. \sum_{s=1}^t X_s \ge
    \sqrt{2\gamma_nV_t\log(N/\delta)}}
    &\le \delta,
  \end{align*}
  where
  \begin{align*}
    V_t &\defeq \max\set[\Big]{\varepsilon,\sum_{s=1}^t \sigma_s^2}, &
    \gamma_n &\defeq 1 + \frac{1}{\log(n)}, &
    N &\defeq 1 + \ceil*{\frac{\log(n\sigma^2/\varepsilon)}{\log(\gamma_n)}}.
  \end{align*}

  \begin{proof}
    For $\psi\in\Real$ define
    \begin{align*}
      M_{t,\psi} &\defeq \exp\paren[\bigg]{\sum_{s=1}^t \psi X_s - \frac{1}{2}\psi^2\sigma_s^2}.
    \end{align*}
    If $\tau \le n$ is a stopping time with respect to $\mathcal{F}$,
    then $\Ex{M_{\tau,\psi}} \le 1$ by \cref{lemma:martingale-mgf}.
    Therefore by Markov's inequality,
    \begin{align*}
      \delta/N
      &\ge \Prob{M_{\tau,\psi} \ge N/\delta} \\
      &= \Prob[\bigg]{\exp\paren[\Big]{\sum_{s=1}^\tau \psi X_s - \frac{\psi^2\sigma_s^2}{2}} \ge N/\delta} \\
      &= \Prob[\bigg]{\psi \sum_{s=1}^\tau X_s - \frac{\psi\sigma_s^2}{2} \ge \log(N/\delta)} \\
      &\ge \Prob[\bigg]{\sum_{s=1}^\tau X_s \ge \frac{\log(N/\delta)}{\psi} + \frac{\psi V_\tau}{2}},
      &\text{since } V_\tau \ge \sum_{s=1}^\tau \sigma_s^2.
    \end{align*}
    To get the tightest tail bound, we want to minimize the quantity
    on the right side of the inequality; it attains a minimum of
    $\sqrt{2V_\tau \log(1/\delta)}$ at
    $\psi_{\min} \defeq \sqrt{2\log(N/\delta)/V_\tau}$.  Furthermore,
    for any $\psi\ge\psi_{\min}$ that quantity is at most
    $\psi V_\tau$.

    However, since $V_\tau$ is a random quantity, we cannot simply set
    $\psi \defeq \psi_{\min}$.  Instead, we recognize that
    $V_\tau \in [\varepsilon, n\sigma^2]$ almost surely, and we can
    logarithmically cover this range with the $N$ values
    $\varepsilon\gamma_n^{k-1}$ for $k=1,2,\dotsc,N$; at least one of
    these must lie in the interval $[V_\tau/\gamma_n, V_\tau]$.  Then
    we can define the corresponding values
    \begin{align*}
      \psi_k &\defeq \sqrt{\frac{2\log(N/\delta)}{\varepsilon\gamma_n^{k-1}}},
              &\text{for } k=1,2,\dotsc,N;
    \end{align*}
    there must be some value of $k$ for which
    $\psi_k \in [\psi_{\min}, \gamma_n\psi_{\min}]$.  Using a union bound
    over these $N$ values, we get
    \begin{align*}
      \delta
      &\ge \Prob[\bigg]{\exists k\in[N]. \sum_{s=1}^\tau X_s \ge \frac{\log(N/\delta)}{\psi_k} + \frac{\psi_k V_\tau}{2}} \\
      &\ge \Prob[\bigg]{\sum_{s=1}^\tau X_s \ge \gamma_n\psi_{\min}V_\tau} \\
      &= \Prob[\bigg]{\sum_{s=1}^\tau X_s \ge \sqrt{2\gamma_nV_\tau\log(N/\delta)}}.
    \end{align*}
    To complete the proof, define the stopping time $\tau =
    \min\set{n,\tau_n}$ where
    \begin{align*}
      \tau_n &\defeq \min\set[\Big]{t \le n \given \sum_{s=1}^tX_s \ge \sqrt{2\gamma_nV_t\log(N/\delta)}}.
              \qedhere
    \end{align*}
  \end{proof}
\end{lemma}


\end{document}

%%% Local Variables:
%%% mode: latex
%%% TeX-master: t
%%% End:
