\documentclass{article}
\usepackage[utf8]{inputenc}

% if you need to pass options to natbib, use, e.g.:
% \PassOptionsToPackage{numbers, compress}{natbib}
% before loading nips_2018
\PassOptionsToPackage{square,comma,numbers}{natbib}

% ready for submission
\usepackage{nips_2018}

% to compile a preprint version, e.g., for submission to arXiv, add
% add the [preprint] option:
% \usepackage[preprint]{nips_2018}

% to compile a camera-ready version, add the [final] option, e.g.:
% \usepackage[final]{nips_2018}

% to avoid loading the natbib package, add option nonatbib:
% \usepackage[nonatbib]{nips_2018}

\title{Differentially Private Contextual Linear Bandits}
% \author{
%   Roshan Shariff
%   \And
%   Or Sheffet
% }

\usepackage{newtxtext}
%\usepackage[semibold]{libertine} % a bit lighter than Times--no osf in math
\usepackage[T1]{fontenc} % best for Western European languages
%\usepackage{textcomp} % required to get special symbols
%\usepackage[varqu,varl]{inconsolata} % a typewriter font must be defined
\usepackage{amsmath,mathtools}
\usepackage{amsthm,thmtools,thm-restate}
\usepackage{amssymb} % loads amsfonts
\usepackage{nicefrac}
%\usepackage[libertine,cmintegrals,bigdelims,vvarbb]{newtxmath} %
%replaces some amssymb symbols
\usepackage{newtxmath}
\usepackage[scr=rsfso]{mathalfa} % Use rsfso to provide mathscr
\usepackage{dsfont}
%\usepackage{upgreek}
\usepackage{bm} % load after all math to give access to bold math
\usepackage{etoolbox}

\usepackage{lineno}
\usepackage{microtype}
\usepackage[inline,shortlabels]{enumitem}
\usepackage{setspace}
\usepackage{tikz-cd}
\usepackage{booktabs}
\usepackage{algpseudocode,algorithm}
\usepackage{multicol}
%\usepackage{fullpage}
\usepackage{color}
\newcommand{\os}[1]{\textcolor{red}{Or's comment:~\textbf{#1}}}
%\usepackage{dblfloatfix}

%% Bibliography/References
% \usepackage[round,colon]{natbib}

\bibpunct{\nolinebreak{}[}{]}{,}{n}{}{,}

%% Cross-references
\usepackage{varioref}
\usepackage[pdfusetitle]{hyperref}
% \usepackage{bookmark}
\usepackage[capitalise]{cleveref}

\hypersetup{
  hidelinks,
  bookmarks,
  bookmarksnumbered,
  bookmarksopen,
  bookmarksopenlevel=2
}

%% To-do notes
\usepackage[obeyFinal]{todonotes}

% \newcommand{\tinytodo}[2][]{\todo[size=\tiny]{#2}}
\newcommand{\tinytodo}[2][]{\todo[size=\tiny, #1]{\begin{spacing}{1.0}#2\end{spacing}}}
\newcommand{\RStodo}[2][]{\tinytodo[color=red!20, #1]{R:\@#2}} % Roshan
\newcommand{\OStodo}[2][]{\tinytodo[color=blue!20, #1]{Cs: #2}} % Or
\newcommand{\fix}{\marginpar{FIX}}
\newcommand{\new}{\marginpar{NEW}}

%% Macros

\renewcommand{\vec}[1]{\bm{#1}}
\newcommand{\wildcard}{\mathinner{\,{\cdot}\,}}
\newcommand{\defeq}{\coloneq}
\newcommand{\eqdef}{\eqcolon}
\newcommand{\inv}[1]{#1^{-1}}
\newcommand{\Real}{\mathds{R}}
\newcommand{\Nat}{\mathds{N}}
\newcommand{\Int}{\mathds{Z}}
\newcommand{\mgf}{\mathrm{mgf}}
\newcommand{\UCB}{\mathrm{UCB}}
\newcommand{\ie}{\text{i.e.\@}}
\newcommand{\iid}{\text{i.i.d.\@}}
\renewcommand{\Pr}{\mathds{P}}
\renewcommand\mid{\mathinner{\vert}}
\newcommand{\argmin}{\operatorname*{arg\,min}}
\newcommand{\argmax}{\operatorname*{arg\,max}}
\newcommand{\tr}{\operatorname{tr}}
\newcommand{\rank}{\operatorname{rank}}
\renewcommand{\det}{\operatorname{det}}

\newcommand\given[1][\delimsize]{%
  \providecommand{\delimsize}{}
  \nonscript\:#1\vert\allowbreak\nonscript\:\mathopen{}
}

\DeclarePairedDelimiter{\abs}||
\DeclarePairedDelimiter{\paren}()
\DeclarePairedDelimiter{\brck}{[}{]}
\DeclarePairedDelimiterX{\set}[1]\lbrace\rbrace{#1}
\DeclarePairedDelimiter{\floor}\lfloor\rfloor
\DeclarePairedDelimiter{\ceil}\lceil\rceil
\DeclarePairedDelimiterX{\innerp}[2]\langle\rangle{#1,#2}
\DeclarePairedDelimiterXPP{\Prob}[1]{\Pr}(){}{#1}
\DeclarePairedDelimiterXPP{\PrSet}[1]{\Pr}\{\}{}{#1}
\DeclarePairedDelimiterXPP{\Ex}[1]{\mathds{E}}{[}{]}{}{#1}
\DeclarePairedDelimiterXPP{\Exx}[2]{\mathds{E}_{#1}}{[}{]}{}{#2}
\DeclarePairedDelimiterXPP{\Var}[1]{\mathrm{Var}}{[}{]}{}{#1}
\DeclarePairedDelimiterXPP{\One}[1]{\mathds{1}}\{\}{}{#1}
\DeclarePairedDelimiterXPP{\norm}[2]{}\Vert\Vert{_{#1}}{#2}

%% Other symbols
\newcommand{\A}{\mathcal{A}}
\newcommand{\C}{\mathcal{C}}
\newcommand{\D}{\mathcal{D}}
\newcommand{\E}{\mathcal{E}}
\providecommand\transp{\top}
\let\transpsymbol\transp
\renewcommand{\transp}[1]{#1^\transpsymbol}
\newcommand{\scrF}{\mathscr{F}}
\newcommand{\Wishart}{\mathcal{W}}
\newcommand{\Normal}{\mathcal{N}}
\newcommand{\Eye}[1][]{\bm{I}\notblank{#1}{_{{#1}\times{#1}}}{}}
\newcommand{\XtX}[1]{\transp{#1}{#1}}

% Theorem environments

\declaretheorem[style=plain]{theorem}
\declaretheorem[style=plain,sibling=theorem]{lemma}
\declaretheorem[style=plain,sibling=theorem]{corollary}
\declaretheorem[style=plain,sibling=theorem]{proposition}
\declaretheorem[style=remark,sibling=theorem]{claim}
\declaretheorem[style=definition]{definition}
\declaretheorem[style=definition]{assumption}
\declaretheorem[style=remark,numbered=no]{remark}

\declaretheorem[style=definition,numbered=no]{assumptions}

\newenvironment{assumptions*}[2][]{%
  \begin{assumptions}[#1]
    #2
    \begin{enumerate}[nolistsep]
      \setcounter{enumi}{\theassumption}
      \newcommand{\assume}[1][]{\item\label[assumption]{##1}}
    }{
      \setcounter{assumption}{\theenumi}
    \end{enumerate}
  \end{assumptions}%
}

\Crefname{assumption}{Assumption}{Assumptions}

%%%%%%%%%%%%%%%%%%%%%%%%%%%%%%%%%%%%%%%%%%%%%%%%%%%%%%%%%%%%%%%%%%%%%%%%%%%
%\onehalfspacing

\begin{document}
\setlength{\intextsep}{4pt}
\setlength{\textfloatsep}{4pt}
\setlength{\abovedisplayskip}{3pt}
\setlength{\belowdisplayskip}{3pt}

\maketitle

\begin{abstract}
  We study the contextual linear bandit problem, a version of the
  standard stochastic mutli-armed bandit problem where an algorithm
  sequentially selects actions that maximize a reward, which may
  depend on a provided per-round \emph{context}. Though the context
  can be chosen arbitrarily or adversarially, the reward is assumed to
  be a linear function of a feature vector that encodes the
  context and selected action, with added random noise. Our goal in
  this work is devise differentially private learners for the
  contextual linear bandit problem.

  We first show that using the standard definition of differential
  privacy results in linear regret. Instead, we adopt \emph{joint}
  differential privacy, where we assume that the action chosen on day
  $t$ is only revealed to user $t$ and thus needn't be kept private
  that day, only on following days. We give a general scheme
  converting the linear UCB algorithm into a joint differentially
  private algorithm using the tree-based algorithm of
  \citet{ChanPrivateContinualRelease2010,DworkContinualObservation2010}.
  We bound the resulting algorithms' regret and test their performance
  empirically.  We also give the first lower bound of the
  \emph{additional} regret private algorithms must incur.
 \end{abstract}

\section{Introduction}
\label{sec:introduction}

The well-known \emph{stochastic multi-armed bandit} (MAB) is a
sequential decision-making task in which a learner repeatedly chooses
an action (or arm) and receives a noisy reward.  The objective is to
maximize cumulative reward by quickly discovering the optimal arm: the
one with the best reward in expectation.  The \emph{contextual} bandit
problem is an extension of MAB, where the learner also receives a
\emph{context} in each round, and the expected reward depends on
\emph{both} the context and the selected action.

% Since the agent only observes the reward for actions it takes, this
% is an online learning task with partial information.  Learning the
% rewards associated with all the arms requires an agent to
% \emph{explore} them.  Too much exploration, however, would be
% counter-productive; eventually the algorithm should \emph{exploit}
% the best arms it has found.  The main challenge of bandit strategies
% is to balance exploration and exploitation.

As a motivating example, consider online shopping: the user provides a
context (composed of query words, past purchases, etc.), and the site
responds with a suggested product and receives a reward if the user
buys it.  Ignoring the context and modeling the problem as a standard
MAB (with an action for each possible product) suffers from the
drawback of ignoring the variety of users' preferences; whereas
learning each user's preference separately prevents generalization
between users.  Therefore it is common to model the task as a
contextual \emph{linear bandit} problem: Based on the user-given
context, each action is mapped to a feature vector; the
reward probability is assumed to be the \emph{same} unknown linear
function of the feature vector for all users. 

The above example motivates the need for privacy in the contextual
bandit setting: users' past purchases are sensitive personal
information, yet they strongly predict future purchases.
% An algorithm may pick an ad for a
% user based on their own information, and it may even learn from
% aggregated information to choose better in the future, but an
% adversary analyzing these ad choices must not be able to learn too
% much about any individual user.
In this work, we give the first upper- and lower-bounds for the
problem of (joint) \emph{differentially private} contextual linear
bandits.  Differential privacy is the de facto gold-standard of
privacy-preserving data analysis in both academia and industry,
requiring that an algorithm's output have very limited dependency on
any single user interaction (one context and reward).  However, as we
later illustrate, adhering to the standard notion of differential
privacy (under continual observation) in the contextual bandit
requires us to essentially ignore the context and thus incur linear
regret.  We therefore adopt the more relaxed notion of \emph{joint
  differential privacy}~\citep{KearnsMechanismDesign2014} which,
intuitively, allows us to present the $t$-th user with products
corresponding to her preferences, without revealing (much) about those
preferences to users at times $t'>t$.\footnote{The guarantee of
  differential privacy under continuous observation assures that the
  entire sequence of actions and rewards preserve privacy.  Hence,
  even if all later user collude in an effort to learn user $t$'s
  context, they still have very limited advantage over a random
  guess.}

\subsection{Stochastic Contextual Linear Bandits}%
\label{sec:contextual-bandits}

In the classic multi-armed bandit, in every round $t$ a learner
selects an \emph{action} $a_t$ from a fixed set $\A$ and receives a
\emph{reward} $y_t$.  In the (stationary) \emph{stochastic} MAB, the
reward is noisy with a fixed but unknown expectation
$\Ex{y_t\given a_t}$ that depends only which action was selected.  In
the stochastic contextual bandit problem, the learner also receives a
\emph{context} $c_t\in\C$ at the beginning of each round, and the
expected reward $\Ex{y_t\given c_t,a_t}$ depends on both $c_t$ and
$a_t$.  It is common to assume that the context
affects the reward in a linear way: map every context-action pair to a
\emph{feature vector} $\phi(c_t,a_t)\in\Real^d$ (where $\phi$ is an
arbitrary but known function) and assume that
$\Ex{y_t\given c_t,a_t} = \innerp{\vec\theta^*}{\phi(c_t,a_t)}$.  The
vector $\vec\theta^*\in\Real^d$ is the key unknown parameter of the
environment which the learner must discover to maximize reward.  Now
everything the learner needs to know about the context is implicitly
encoded by the \emph{decision set}
$\D_t \defeq \set{\phi(c_t,a)\given a\in\A \subset \Real^d}$, and
choosing some $x_t\in\A_t$ effectively determines an action
$a_t\in\A_t$.  Thus, the contextual stochastic linear bandit framework
consists of repeated rounds in which the learner
\begin{enumerate*}[(i),before=\unskip{: },itemjoin={{; }},itemjoin*={{; and }}]
\item receives a \emph{decision set} $\mathcal{D}_t \subset \Real^d$
\item chooses an \emph{action} $\vec x_t \in \mathcal{D}_t$
\item receives a stochastic \emph{reward}
  $y_t = \innerp{\vec\theta^*}{\vec x_t} + \eta_t$.
\end{enumerate*}
Note that when the context is identical over all days (with orthogonal vectors) the problem reduces to standard MAB.

The objective is to maximize cumulative reward, which is equivalent to
minimizing a quantity called \emph{regret} --- the extra reward a
learner would have received by always choosing the best available
action.  An effective learner may nevertheless earn only a small
reward if all the actions are poor, but would then still achieve low
regret.  In other words, the regret characterizes the cost of having
to \emph{learn} the optimal action over just \emph{knowing} it beforehand.
For stochastic problems, we are usually interested in a related
quantity called \emph{pseudo-regret}, which is the extra
\emph{expected} reward that could have been earned.  In our setting,
the cumulative pseudo-regret after $n$ rounds is
$\widehat{R}_n \defeq
\sum_{t=1}^n\max_{x\in\D_t}\innerp{\vec\theta^*}{\vec x - \vec x_t}$.%
\footnote{The pseudo-reward ignores the reward noise but not the
  randomness in the learner's choice of actions.  It equals the reward
  in expectation, but is more amenable to high-probability bounds such
  as ours, because the learner's effectiveness is not masked by the
  unavoidable reward noise.}


\subsection{Joint Differential Privacy.}%
\label{sec:joint-dp}

As discussed above, the context and reward may be considered private
information about the users which we wish to keep private from all
\emph{other} users. We thus introduce the notion of joint
differentially private learners under continuous observation, a
combination of two definitions \citep[given
in][]{KearnsMechanismDesign2014,DworkContinualObservation2010}. First,
we say two sequences
$S = \langle (\mathcal{D}_1, y_1), (\mathcal{D}_2, y_2), ...,
(\mathcal{D}_n, y_n) \rangle$ and
$S' = \langle (\mathcal{D}'_1, y'_1), ..., (\mathcal{D}'_n, y'_n)
\rangle$ are \emph{$t$-neighbors} if for all $t'\neq t$ it holds that
$(\mathcal{D}_{t'},y_{t'}) = (\mathcal{D}'_{t'}, y'_{t'})$.
%\os{maybe omit?} We say $S$ and $S'$ are \emph{neighbors} if there exists a $t$ such that the two sequences are $t$-neighbors.
% For example, a
% search engine might have a context consisting of a user's search
% query, identity, interests, and physical location, while the reward
% indicates which search result the user clicked on.  The search engine
% should, of course, use the context to answer each query; furthermore,
% it should learn from the reward to better respond to future queries
% from other users.  However, it should also maintain privacy: its
% responses to queries should not reveal \emph{too much} information
% about the context and rewards it has learned from.  More precisely, we
% want algorithms that are \emph{jointly differentially private} in the
% following sense:

\begin{definition}
\label{def:joint-dp}
  A randomized algorithm $A$ for the contextual bandit problem is
  \emph{$(\varepsilon,\delta)$-jointly differentially private} (JDP) under continual observation if for any $t$ and any pair of $t$-neighboring sequences $S$ and $S'$, and any subset ${\cal S}_{>t} \subset \mathcal{D}_{t+1} \times \mathcal{D}_{t+2} \times \cdots \times \mathcal{D}_{n}$ of sequence of actions ranging from day $t+1$ to the end of the sequence, it holds that $\Pr[A(S)\in \mathcal{S}_{>t}] \leq e^\varepsilon\Pr[A(S')\in \mathcal{S}'_{>t}] +\delta$.
\end{definition}
Note that the standard definition of differential privacy requires the
distribution proximity to hold for any subset of sequences of actions
ranging from day $t$ to the end of sequence. However, in our problem
formulation, with given decision sets, this notion isn't even
well-defined --- the decision set of day $t$ is different under $S$
and under $S'$. Therefore, when we discuss the impossibility of
regret-minimization under standard differential privacy, we revert
back to the setting of different contexts with fixed action set. See
\cref{sec:lower_bounds} for further details.

\subsection{Our Contribution and Paper Organization}
\label{subsec:contributions}

In this work, in addition to formulating the definition of JDP under
continual observation, we also present a framework for implementing
JDP algorithms for the contextual bandit problem. Not surprisingly,
our framework combines the tree-based algorithm~\cite{ChanPrivateContinualRelease2010,DworkContinualObservation2010}  with the linear
upper-confidence bound (LinUCB) algorithm\todo{Cite LinUCB}.  However, for modularity,
we choose to analyze a family of linear UCB algorithms where in each
day one uses a different regularizer, under the premise that all
regularizers' singular values are bounded. Moreover, we repeat our analysis twice. Once in a general framework,
obtaining an upper bound on the regret which is proportional to
$\tilde O(\sqrt n)$; and once in an instance dependent setting, where in each
day there's a significant gap of at least $\Delta$ between the best arm of each round and any other
arms, obtaining a regret upper bound of $\mathrm{polylog}(n)/\Delta$.\footnote{The analysis carries through to $k-1$ gaps between the $i$th
arm and the leading one, yet its result is too cumbersome so we omit
this analysis.} Our leading application of course is privacy, though
one could postulate other reasons why such a change in regularizers
would be useful (e.g., updating the regularizer when an initial upper
bound of a parameter turns out to be wrong). This analysis is given in Section~\ref{sec:LinUCB}. We then plug in two
particular regularizers into our scheme: one based on additive Wishart
noise~\citep{SheffetPrivateApproxRegression2015} which always results
in a PSD, acheiving regret of $\tilde O(\sqrt{n}(d+\nicefrac{\sqrt d}{\varepsilon}))$; and one based on additive Gaussian
noise \citep{DworkAnalyzeGauss2014} which doesn't have to result in a PSD, so
we shift it by $\rho I$ so that w.h.p.\ (over all days in the sequence)
the shifted eigenvalues of the noise are all positive, and obtain a regret bound of $\tilde O(\sqrt n d/\varepsilon)$. (The actual bounds depends on \emph{many} variables, so there are settings where the latter is better than the former.) The details of the two techniques are given in Section~\ref{sec:alg-dp}.  We also present
empirical evidence showing our analysis is in fact optimal, presented in Section~\ref{}.

We also give lower bound for the $\varepsilon$-differentially private
MAB problem. Whereas all previous work on the private MAB problem uses
the standard (non-private) bounds, we show that the regret of
\emph{any} private algorithm must incur \emph{an additional} cost of
$\Omega(k\log(T)/\varepsilon)$. This resembles the lower bound of the
adversarial setting, however, the proof technique cannot rely on the
standard packing arguments~\citep[e.g.][]{HardtTalwarGeometryDP2010} seeing as the input for the
problem is stochastic rather than adversarial.  Instead, we apply a
new result of \citet{KarwaVadhanFiniteSampleDP2017} showing, in essence,
that the effective group-privacy between two possible inputs of $n$
samples drawn iid from distributions $P$ and $Q$ respectively is about
$n\cdot d_{\rm TV}(P,Q)$. Details appear in Section~\ref{sec:lower_bounds}.



\todo[inline]{Complete this section.}
\section{Preliminaries, Background and Related Work}
\label{sec:background}

\subsection{Differential Privacy.} Differential privacy was first introduced by
\citet{DworkCalibratingNoiseSensitivity2006,DworkOurData2006}, and it is a rigorous mathematical notion that determines that the probability of any event changes very little with the change of a single datum. (We omit the definition as the Joint-DP was already defined.) A classic result~\cite{DworkOurData2006} states that if $f$ is a function of the input whose $L_2$-norm cannot change by more than $B$ with a change of a single datum, then adding Gaussian noise to $f$ of $0$-mean and variance $B^2\log(\nicefrac 1 \delta)/\varepsilon^2$ preserves $(\varepsilon,\delta)$-differential privacy. Among its many elegant traits is the notion of group privacy: should $k$ datums change the data, the change to any every is still limited, by (roughly) $k$ times the magnitude of a single entry's change. A different property of differential privacy is composition~\cite{DworkBoosting2010}: The combination of $k$ $(\varepsilon,\delta)$-differentially private algorithms is $\paren[\big]{O(k\varepsilon^2 + 2\sqrt{k\log(\nicefrac 1 {\delta'})}), k\delta+\delta'}$-differentially private for any $\delta'>0$.

The notion of differential privacy under continual observation was
first defined by \citet{DworkContinualObservation2010} have also been
defined, using the \textbf{tree-based algorithm}~\citep[originally
appearing in][]{ChanPrivateContinualRelease2010}). This algorithm
maintains a binary tree whose $n$ leafs correspond to the $n$ entries
in the input sequence. Each node in the tree maintains a noisy
(privacy-preserving) count of the sum of the input entries in its
subtree, thus enabling us to maintain a running count of the sum of
inputs thus far by combining no more than $\log(n)$ noisy counts. This
algorithm is the key ingredient in a variety of works the deal with
privacy in an online setting, including
counts~\cite{DworkContinualObservation2010}, online convex
optimization~\cite{JainDPOnlineLearning2012}, and regret minimization
in both the
adversarial~\citep{SmithThakurtaPrivateOnlineLearning2013,TossouAchievingPrivacyAdversarial2017}
and
stochastic~\cite{MishraNearlyOptimalDPBandits2015,TossouAlgDPBandits2016}
settings. We comment that \citet{MishraNearlyOptimalDPBandits2015}
proposed an algorithm similar to our own for the contextual bandit
setting, however (i) without maintaining PSD, (ii) without any
analysis, only empirical evidence, and (iii) without presenting lower
bounds.

In the offline setting, there have been numerous works studying differentially private linear regression~\citep{ChaudhuriDPERM2011,BassilyPrivateEmpiricalRisk2014}. In this work we rely on two techniques that maintain a private version of the second-moment matrix of the data (the Gram matrix) --- the work of~\cite{DworkAnalyzeGauss2014} which uses additive Gaussian noise (which doesn't necessarily result in a PSD output) and the work of~\cite{SheffetPrivateApproxRegression2015} which uses additive Wishart noise (and is always PSD).

\subsection{The MAB and the Contextual MAB Problems.}

\begin{multicols}{2}[\paragraph{Notation.}\label{sec:notation}]
  \nolinenumbers
  \begin{description}[style=sameline,leftmargin=5em,nosep]
  \item[$n$] horizon, i.e. number of rounds
  \item[$s,t$] indices of rounds
  \item[$d$] dimensionality of action space
  \item[$\D_t\subset\Real^d$] decision set at round $t$
  \item[$\vec x_t \in \D_t$] action at round $t$
  \item[$y_t \in \Real$] reward at round $t$
  \item[$\vec X_t \in \Real^{t\times d}$] matrix with $\vec X_s = \transp{\vec
      x_s}$ for $s\in[t]$
  \item[$\vec y_t \in \Real^t$] vector of rewards $\vec y_s = y_s$
  \item[$\vec\theta^* \in \Real^d$] unknown parameter vector
  \item[$\norm{V}{\vec x}$] $\defeq \sqrt{\transp{\vec x} V \vec x}$
  \end{description}
  \linenumbers
\end{multicols}

\section{Linear UCB with Changing Regularizers}
\label{sec:LinUCB}

Our differentially private contextual linear bandit algorithm is based
on the well-studied LinUCB algorithm, an application of the Upper
Confidence Bound (UCB) idea to stochastic linear bandits
\citep{DaniStochasticLinearOptimization2008,RusmevichientongLinearlyParameterizedBandits2010,AbbasiYadkoriImprovedAlgorithmsLinear2011}.
At every round $t$, LinUCB constructs a \emph{confidence set} $\E_t$
containing the unknown parameter vector $\vec\theta^*$ with high
probability.  It then computes an upper confidence bound on the reward
of each action in the decision set $\D_t$, and ``optimistically''
chooses the action with the highest UCB:
$\vec x_t \gets \argmax_{\vec x\in\D_t} \UCB_t(\vec x)$, where
$\UCB_t(\vec x) \defeq \max_{\vec\theta\in\E_t}
\innerp{\vec\theta}{\vec x}$.  We assume the rewards are linear with
added subgaussian noise (i.e.,
$y_s = \innerp{\vec\theta^*}{\vec x_s} + \eta_s$ for $s<t$), so it is
natural to center the confidence set $\E_t$ on the (regularized)
linear regression estimate:
\begin{align*}
  \hat{\vec\theta}_t
  &\defeq \min_{\hat{\vec\theta} \in \Real^d} \norm{}{\vec X_{t-1} \hat{\vec\theta}
    - \vec y_{t-1}}^2 + \norm{H_t}{\hat{\vec\theta}}^2
    = \inv{(G_t + H_t)} \transp{\vec X_{t-1}} \vec y_{t-1}.
    &\text{where } G_t \defeq \XtX{\vec X_{t-1}}
\end{align*}
The matrix $V_t \defeq G_t + H_t \in \Real^{d\times d}$ is a regularized
version of the Gram matrix $G_t$.  Whenever the learner chooses an
action vector, the corresponding reward gives it some information
about the projection of $\vec\theta^*$ onto that vector.  In other
words, the estimate $\hat{\vec\theta}$ is probably closer to
$\vec\theta^*$ along directions of $\Real^d$ where many actions have
been taken.  This motivates the use of ellipsoidal confidence sets
that are smaller in these directions, inversely related to the
corresponding eigenvalues of $G_t$ (and ideally $V_t$). The ellipsoid
is uniformly scaled by $\beta_t$ to achieve the desired confidence
level, as prescribed by \cref{prop:calc-beta}.
\begin{align}\label{eq:def-ellip}
  \E_t &\defeq \set{\vec\theta\in\Real^d \given
        \norm{V_t}{\vec\theta-\hat{\vec\theta}_t} \le \beta_t},
  &\text{ for which }
    \UCB_t(\vec x) &= \innerp{\hat{\vec\theta}}{\vec x} + \beta_t\norm{\inv{V_t}}{\vec x}.
\end{align}
Just as the changing regularizer $H_t$ perturbs the Gram matrix $G_t$,
our algorithm allows for the vector
$\vec u_t \defeq \transp{\vec X_{t-1}} \vec y_{t-1}$ to be perturbed
by a changing $\vec h_t$ to get
$\tilde{\vec u}_t \defeq \vec u_t + \vec h_t$.  The estimate
$\hat{\vec\theta}_t$ is replaced by
$\tilde{\vec\theta}_t \defeq \inv{V_t}\tilde{\vec u}_t$.

\begin{algorithm}
  \caption{Linear UCB with Changing Perturbations}\label{alg:linucb}
  \begin{algorithmic}
    \State \textbf{Initialize:} $G_1 \gets 0_{d\times d}$,
    $u_1\gets \vec 0_{d}$.
    \For{each round $t = 1,2,\dotsc,n$}
    \State Receive $\D_t \gets{}$ decision set ${} \subset \Real^d$.
    \State Receive regularized $V_t \gets G_t + H_t$ and perturbed $\tilde{\vec u}_t \gets \vec u_t + \vec h_t$
    \State Compute regressor $\tilde{\vec\theta}_{t} \gets \inv{V_{t}}\tilde{\vec u}_{t}$
    \State Compute confidence-set bound $\beta_t$ based on Porposition~\ref{prop:calc-beta}.
    \State Pick action $\vec x_t \gets \argmax_{\vec x\in\D_t}
    \innerp{\tilde{\vec \theta}_t}{\vec x} +
    \beta_t\norm{\inv{V_t}}{\vec x}$.
    %\Comment{Choose action.}
    \State Observe $y_t \gets {}$ reward for action $\vec x_t$
    %\Comment{Observe reward.}
    \State Update: $G_{t+1} \gets G_{t} + \vec x_t \transp{\vec x_t},
    \quad \vec u_{t+1} \gets \vec u_{t} + \vec x_t y_t$
    %\Comment{Record in privacy-preserving history.}
    \EndFor
  \end{algorithmic}
\end{algorithm}

We will refer back to the following assumptions about the environment
and algorithm:
\begin{assumptions*}[\Cref{alg:linucb}]{%
    For all rounds $t=1,\dotsc,n$ and actions $\vec x\in\D_t$:}
  \assume[ass:x-bound] $\norm{}{\vec x} \le L$.
  \assume[ass:meanreward-bound]
  $\abs{\innerp{\vec\theta^*}{\vec x}} \le 1$.
  \assume[ass:thetastar-bound] $\norm{}{\vec\theta^*} \le S$.
  \assume[ass:linear-reward]
  $y_t = \innerp{\vec\theta^*}{\vec x_t} + \eta_t$
  \assume[ass:psd-reg] $H_t \succ 0$ is symmetric.
  \assume[ass:subgaussian] $\eta_t$ is $\sigma^2$-conditionally
  subgaussian given actions and previous rewards:
  \begin{align*}
    \Ex{\exp(\lambda\eta_t) \given \vec x_1, y_1, \dotsc, \vec x_{t-1}, y_{t-1}, \vec x_t}
    &\le \exp(\lambda^2\sigma^2/2), &\text{for all } \lambda\in\Real.
  \end{align*}
\end{assumptions*}
\cref{ass:x-bound,ass:meanreward-bound} are not needed for the
correctness of the algorithm (i.e.\ its confidence sets), only for the
regret bounds in \cref{sec:regret-bounds}.  Conversely, it is
important to note that of these, only \cref{ass:x-bound} is needed for
differential privacy (see \cref{ass:bounded-data} in \cref{sec:alg-dp}).

The algorithm relies upon the confidence ellipsoids to be accurate but
nevertheless tightly concentrated around the unknown $\vec\theta^*$.
In other words, the $\beta_t$ should be as small as possible while
accounting for the noise $\eta_t$ and the perturbations caused by
$H_t$ and $\vec h_t$.

\begin{definition}[Accurate $(\beta_t)_t$]\label{def:accurate-beta}
  A sequence $(\beta_t)_{t=1}^n$ is $(\alpha, n)$-\emph{accurate} for
  $(H_t)_{t=1}^n$ and $(\vec h_t)_{t=1}^n$ if, with probability at
  least $1-\alpha$, it satisfies
  $\norm{V_t}{\vec\theta^* - \tilde{\vec\theta}_t} \le \beta_t$
  for all rounds $t=1,\dotsc,n$ simultaneously.
\end{definition}

\begin{definition}[Accurate $\rho_{\min}$, $\rho_{\max}$, and
  $\gamma$]\label{def:accurate-params}
  The tail bounds $0 < \rho_{\min} \le \rho_{\max}$ and $\gamma$ are
  $\alpha$-\emph{accurate} for some $H_t$ and $\vec h_t$ if the
  following hold with probability at least $1-\alpha$ for each round
  separately:
  \begin{align*}
    \norm{}{H_t} &\le \rho_{\max},
    &\norm{}{\inv{H_t}} &\le \nicefrac{1}{\rho_{\min}},
    &\norm{\inv{H_t}}{\vec h_t} &\le \gamma.
  \end{align*}
\end{definition}

\begin{restatable}[Calculating $\beta_t$]{proposition}{CalcBeta}%
  \label{prop:calc-beta}
  Suppose
  \cref{ass:thetastar-bound,ass:linear-reward,ass:psd-reg,ass:subgaussian}
  hold and let $\rho_{\min}$, $\rho_{\max}$, and $\gamma$ be
  $(\alpha/2n)$-accurate for some $\alpha\in(0,1)$ and horizon $n$.
  Then $(\beta_t)_{t=1}^n$ is $(\alpha,n)$-accurate where
  \begin{align*}
    \beta_t
    &\defeq \sigma\sqrt{2\log(2/\alpha) + \log\det V_t - d\log\rho_{\min}}
      + S\sqrt{\rho_{\max}} + \gamma \\
    &\le \sigma\sqrt{2\log\tfrac{2}{\alpha} + d\log\paren[\Big]{\tfrac{\rho_{\max}}{\rho_{\min}}
      + \tfrac{tL^2}{d\rho_{\min}}}}
      + S\sqrt{\rho_{\max}} + \gamma.
      &\text{(Under \cref{ass:x-bound})}
  \end{align*}
\end{restatable}

\vspace{-5pt}
\subsection{Regret Bounds}
\label{sec:regret-bounds}
Having established that the bounds on the confidence sets $\beta_t$ require only bounds on the regularizers, we now give the resulting regret bounds of~\Cref{alg:linucb}. The proofs of these theorems, which are rather long yet are by no means novel, are deferred to the supplementary material.
\begin{restatable}[Regret of \Cref{alg:linucb}]{theorem}{ThmLinUCBRegret}%
  \label{thm:linucb-regret}%\
  Suppose \cref{ass:x-bound,ass:meanreward-bound,ass:thetastar-bound,%
    ass:linear-reward,ass:psd-reg,ass:subgaussian} hold and the
  $\beta_t$ are as given by \cref{prop:calc-beta}.  Then with
  probability at least $1-\alpha$ the pseudo-regret of
  \cref{alg:linucb} is bounded by
    \begin{align}
    \label{eq:regret_general_formula}
    \widehat R_n
    &\le \sqrt{8n}\brck[\bigg]{\sigma\paren*{2\log\tfrac{2}{\alpha}
      + d\log\paren*{\tfrac{\rho_{\max}}{\rho_{\min}} + \tfrac{nL^2}{d\rho_{\min}}}}
      + (S\sqrt{\rho_{\max}} + \gamma)\sqrt{d\log\paren*{1 + \tfrac{nL^2}{d\rho_{\min}}}}}
  \end{align}
\end{restatable}

\begin{restatable}[Gap-Dependent Regret of
  \Cref{alg:linucb}]{theorem}{ThmLinUCBGapRegret}%
  \label{thm:linucb-gap-regret}
  Suppose \cref{ass:x-bound,ass:meanreward-bound,ass:thetastar-bound,%
    ass:linear-reward,ass:psd-reg,ass:subgaussian} hold and the
  $\beta_t$ are as given by \cref{prop:calc-beta}.  If the optimal
  actions in every decision set $\D_t$ are separated from the
  sub-optimal arms by a reward gap of at least $\Delta$, then with
  probability at least $1-\alpha$ the pseudo-regret of
  \cref{alg:linucb} satisfies
  \begin{align}
  \label{eq:regret_gap_general_formula}
    \widehat R_n
    &\le \frac{8}{\Delta} \brck[\bigg]{\sigma\paren*{2\log\frac{2}{\alpha}
      + d\log\paren*{\tfrac{\rho_{\max}}{\rho_{\min}} + \tfrac{nL^2}{d\rho_{\min}}}}
      + (S\sqrt{\rho_{\max}}+\gamma)\sqrt{d\log\paren*{1+\tfrac{nL^2}{d\rho_{\min}}}}}^2
  \end{align}
\end{restatable}


\section{Linear UCB with Joint Differential Privacy}
\label{sec:alg-dp}

Notice that \cref{alg:linucb} uses its history of actions and rewards
up to round $t$ only via the confidence set $\E_t$, which is to say
via $V_t$ and $\tilde{\vec u}_t$, which are perturbations of the Gram
matrix $G_t\defeq\XtX{\vec X_{t-1}}$ and the vector
$\vec u_t \defeq \transp{\vec X_{t-1}} \vec y_{t-1}$, respectively;
these also determine $\beta_t$.  By recording this history with
differential privacy, we obtain a Linear UCB algorithm that is jointly
differentially private (\cref{def:joint-dp}) because it simply
post-processes $G_t$ and $\vec u_t$.

\begin{claim}[see {\citet[Proposition~2.1]{DworkAlgorithmicFoundationsDifferential2014}}]
  If the sequence $(V_t,\tilde{\vec u}_t)_{t=1}^{n-1}$ is
  $(\varepsilon,\delta)$-differentially private with respect to
  $(\vec x_t, y_t)_{t=1}^{n-1}$, then \cref{alg:linucb} is
  $(\varepsilon,\delta)$-jointly differentially private.
\end{claim}

\begin{remark}
  \Cref{alg:linucb} is only jointly differentially private even though
  the history maintains full differential privacy --- its action
  choice depends not only on the past contexts $c_s$ ($s < t$, via the
  differentially private $\vec X_{t-1}$) but also on the current
  context $c_t$ via the decision set $\D_t$.  Since this use of $c_t$
  is not differentially private, it is revealed by the algorithm's
  chosen $\vec x_t$ (as discussed in \cref{sec:joint-dp}).
\end{remark}

To make the following exposition simpler, define the matrix
$A \defeq \begin{bmatrix} \vec X_n & \vec y_n \end{bmatrix} \in
\Real^{n\times(d+1)}$, and the matrices $A_t$ holding the top $t-1$ rows of $A$ (and
$A_1 = \vec 0_{1\times(d+1)}$).  Denote $M_t \defeq \XtX{A_t}$, and note that the top-left $d\times d$
sub-matrix of $M_t$ is the Gram matrix $G_t$ and the
first $d$ entries of its last column are
$\vec u_t = \transp{\vec X_{t-1}}\vec y_{t-1}$.  It therefore suffices
to maintain a differentially private approximation of $M_t$. Since
\cref{alg:linucb} allows for additive perturbations $H_t$ and $h_t$,
we can use any privacy mechanism that uses additive noise, releasing
$M_t + N_t$.  The top-left $d\times d$ sub-matrix of $N_t$ becomes
$H_t$ and the first $d$ entries of its last column become $\vec h_t$. Lastly, in order to maintain a private estimation of $M_t$ we require the following assumption.
\begin{assumption}\label{ass:bounded-data}
  Each row of $A$ is bounded:
  $\norm{}{\vec x_t}^2 + y_t^2 \le \tilde L^2$ for all rounds $t$.  In
  particular, if $\vec x_t \le L$ (\cref{ass:x-bound}) and
  the rewards are bounded: $\abs{y_t} \le \tilde B$ (not just their
  means as in \cref{ass:meanreward-bound}), we can set $\tilde L^2 = L^2 + \tilde B^2$.
\end{assumption}
Below we present two techniques for maintaining (and updating) the
private estimations of $M_t$. As mentioned in
Section~\ref{sec:background}, the key component of our technique is
the tree-based algorithm, allowing us to estimate $M_t$ using at most
$m \defeq 1 + \ceil{\log_2n}$ noisy counters. In order for the entire
tree-based algorithm to be $(\varepsilon,\delta)$-differentially
private, we add noise to each node in the tree so that each noisy
count on its own preserves
$(\nicefrac{\varepsilon}{\sqrt{8m\ln(\nicefrac{2}{\delta})}},
\nicefrac{\delta}{2m})$-differential privacy. Thus in each day, the
noise $N_t$, that we add to $M_t$, comes from the sum of at most $m$
such counters.

% \subsection{Pan-Privacy with Tree-Based Aggregation}
% \label{sec:tree-mechanism}

% \todo[inline]{Expand this section, keeping the results below.}  For a horizon
% of $n$, the pan-private tree mechanism composes $m = \ceil{1+\log_2n}$
% mechanisms.  Suppose we want to achieve a target of
% $(\varepsilon,\delta)$-differential privacy after composing these $m$
% mechanisms.  Then it is sufficient for each mechanism to be
% $(\varepsilon',\delta')$-differentially private where:
%  \begin{align*}
%    \varepsilon'' &= \frac{\varepsilon}{2\sqrt{2m\ln\frac{1}{\delta - m\delta'}}} \\
%    \delta' &< \delta/m.
%  \end{align*}

\subsection{Differential Privacy Under Continual Observation via Wishart Noise}
\label{sec:dp-wishart}

First, we instantiate the tree-based algorithm with noise from a
suitably chosen Wishart distribution $\Wishart_{d+1}(V, k)$, which is the
result of sampling $k$ independent $(d+1)$-dimensional Gaussians from
$\Normal(\vec 0_{d+1}, V)$ and computing their $2^{\rm nd}$-moment
matrix. Under suitably chosen $V$ and $k$, this noise preserves
$(\varepsilon,\delta)$-differential privacy.

\begin{theorem}[\cite{SheffetPrivateApproxRegression2015}, Theorem~4.1]%
  \label{thm:wishart-cont-dp}
  Fix positive $\varepsilon_0$ and $\delta_0$. If the $L_2$-norm of each
  row in the input is bouded by $\tilde L$ then adding to the inputs'
  $2^{\rm nd}$-moment matrix noise sampled from $\Wishart_{d+1}(\tilde
  L^2 \Eye, k_0)$ is $(\varepsilon_0,\delta_0)$-differentially private,
  provided $k_0\geq d+1 + 28\varepsilon_0^{-2}\ln(4/\delta_0)$.
\end{theorem}
Applying this guarantee to our setting, where each count needs to
preserve
$(\varepsilon/\sqrt{8m\ln(2/\delta)}, \delta/2m)$-differential
privacy, it suffices to sample a matrix from
$\Wishart_{d+1}(\tilde{L} \Eye, k)$ with
$k \defeq \ceil{d+1 + 224
  m\varepsilon^{-2}\ln(8m/\delta)\ln(2/\delta)}$. Moreover, combining
the noise of $m$ samples from the Wishart distribution is equivalent
to sampling a noise matrix
$N_t\sim \Wishart_{d+1}(\tilde L^2\Eye, mk)$. (Intuitively, we merely
concatenate the $m$ batches of multivariate Gaussians sampled in the
generation of each of the $m$ Wishart noises.) Furthermore, consider
the $H_t$ and $\vec h_t$ we generate from $N_t$ ($N_t$'s top-left
submatrix and the right-most subcolumn resp.)~--- $H_t$ is sampled
from $\Wishart_d(\tilde L^2\Eye,k)$ , and each entry of $\vec h_t$ is
the dot-product of two multivariate Gaussians. Knowing their
distribution, we can infer the accurate bounds required for our regret
bounds.
\begin{restatable}{proposition}{PropWishartTails}
\label{pro:accurate_bounds_for_Wishart}
Fix any $\alpha>0$. If for each $t$ the $H_t$ and $\vec h_t$ are
generated by the tree-based algorithm with Wishart noise
$\Wishart_{d+1}(\tilde L^2 \Eye, k)$, then the following
are $(\alpha/2n)$-accurate bounds:
{\small\begin{align*}
    \rho_{\min} \defeq \tilde L^2 \paren[\big]{\sqrt{mk}
                 - \sqrt{d} - \sqrt{2\ln(8n/\alpha})}^2,\;\,
    \rho_{\max} \defeq \tilde L^2 \paren[\big]{\sqrt{mk}
                 + \sqrt{d} + \sqrt{2\ln(8n/\alpha)}}^2,\;\,
    \gamma \defeq \sqrt{d} + \sqrt{2\ln(2n/\alpha)}.
 \end{align*}}
\end{restatable}

The proof applies known tail bounds and is deferred to the appendix.
It only remains to plug in the above bounds and derive the
regret of \cref{alg:linucb} regularized with Wishart noise.
\begin{corollary}
\label{cor:regret_with_Wishart}
\cref{alg:linucb} with $H_t$ and $\vec h_t$ generated by the tree-based
mechanism with each node adding noise independently from
$\Wishart_{d+1}( (L^2+\tilde B^2)\Eye, k )$, assuming
$n/\log(n)\gg k$, we get a pseudo-regret bound of
\[ O(\sqrt n\left( d\log(n)\left( \sigma+S\tilde{L} \right) +S\tilde{L}\sqrt d \left( \tfrac{\log(n)\log(\nicefrac{\log(n)}{\delta})}{\varepsilon} +\sqrt{\log(\nicefrac 2 \alpha)\log(\tfrac{n}{d^2\log(n)}}) \right) +\sigma\log(\nicefrac 1 \alpha) \right) )  \] in general, and a gap-dependent pseudo-regret bound of
\[ O(\tfrac1 \Delta\left( d\log(n)\left( \sigma+S\tilde{L} \right) +S\tilde{L}\sqrt d \left( \tfrac{\log(n)\log(\nicefrac{\log(n)}{\delta})}{\varepsilon} +\sqrt{\log(\nicefrac 2 \alpha)\log(\tfrac{n}{d^2\log(n)}}) \right) +\sigma\log(\nicefrac 1 \alpha) \right)^2 ) \]
\end{corollary}

\subsection{Differential Privacy via Additive Gaussian Noise}
\label{sec:dp-gauss}

Our second alternative is to instantiate the tree-based algorithm with
symmetric Gaussian noise: $\left((d+1)\times (d+1)\right)$-matrices
which are symmetric and their upper-diagonal entries taken from a
Gaussian distribution. Recall that each datum has a bounded $L_2$-norm
of $\tilde L$, hence a change to a single datum may alter the
Frobenius norm of $M_t$ by $\tilde L^2$. It follows that in order to
make sure each node in the tree-based algorithm preserves
$(\nicefrac{\varepsilon}{\sqrt{8m\ln(\nicefrac{2}{\delta})}},
\nicefrac{\delta}{2})$-differential privacy,%
\footnote{We use here the slightly better bounds for the composition
  of Gaussian noise based on zero-Concentrated
  DP~\citep{BunConcentratedDifferentialPrivacy2016}.} %
the variance of each coordinate must be
$\sigma_{\rm noise}^2 = 16m\tilde L ^4 \ln(\nicefrac 4 \delta)^2 /
\varepsilon^2$.

Instead of ensuring the noise is symmetric by copying the upper-diagonal entries to the entries below the diagonal, we analyze here a slight variation --- where for each node we sample a matrix $Z$ which all of entries are sampled iid from a Gaussian, then symmetrize $Z$ using $\tfrac 1 {\sqrt 2}(Z+\transp{Z})$.\footnote{This increases the variance along the diagonal entries beyond the noise magnitude required to preserve privacy, but by only a constant factor of $2$.} When all entries on $Z$ are sampled from $\Normal(0,\sigma_{\rm noise}^2)$, known concentration results on the top singular value of $Z$ give that $\Pr[ \|Z\| > \sigma_{\rm noise}(4\sqrt{d+1} + 2\ln(\nicefrac 1 \alpha)) ] < \alpha$. Note however that in each day $t$ we sum $\leq m$ such $Z$ matrices (then symmetrize the noise), thus the variances of each coordinate is $m\sigma^2$. As a result, applying the union-bound, we have that w.p. $\geq 1-\alpha$ it holds that in all $n$ days our noise $N_t$ has operator norm of at most $\sqrt{2m\sigma_{\rm noise}^2}(4\sqrt{d+1} + 2\ln(\nicefrac n \alpha))$. In particular, the top-left $(d\times d)$-submatrix of $N_t$ has operator norm of at most $\Upsilon \defeq \sqrt{2m\sigma_{\rm noise}^2}(4\sqrt{d} + 2\ln(\nicefrac n \alpha)) = \sqrt{32} m\tilde L ^2 \ln(\nicefrac 4 \delta)(4\sqrt{d} + 2\ln(\nicefrac n \alpha)) / \varepsilon$.

However, it is important to note that the result of adding Gaussian noise may not preserve the PSD property of the noisy Gram-matrix. To that end, we add $(\Upsilon +1)\Eye$ to the Gaussian noise, in order to make sure that we maintain strictly positive eigenvalues throughout the execution of the algorithm. The end result is that for each day $t$ we have that $H_t$ is given by summing $\leq m$ matrices whose entries are sampled iid from $\Normal(0,\sigma_{\rm noise}^2)$, symmetrizing the result as shown above and adding $(\upsilon+1)\Eye$; and $\vec h_t$ is the result of summing $\leq m$ $d$-dimensional vectors whose entries are sampled i.i.d from $\Normal(0,\sigma_{\rm noise}^2)$ (the symmterization doesn't change the distribution of each coordinate). As the least eigenvalue of $H_t$ (we can guarantee) is $1$, then $\|\vec h_t\|_{H_t^{-1}} \leq \|\vec h_t\|$ which we can bound using standard concentration bounds on the $\chi^2$-distribution (see Claim~\ref{claim:chi2-tails}). This culminates in the following bounds.
\begin{proposition}
\label{pro:accurate_bounds_for_Gaussian}
Fix any $\alpha>0$. Given that for each $t$ the regularizers $H_t, \vec h_t$ are taken by applying the tree-based algorithm with symmterized Guassian noise whose entries are sampled iid from $\Normal(0, \sigma_{\rm noise}^2)$, then the following $\rho_{\min},\rho_{\max}$ and $\gamma$ are $(\alpha/2n)$-accurate bounds:
\begin{align*}
    &\rho_{\max}\defeq  2\Upsilon+1 &&\rho_{\min}\defeq 1 && \gamma \defeq \sqrt m\sigma_{\rm noise}(\sqrt d + \sqrt{2\ln(\nicefrac{2n}\alpha)})=4m\tilde L^2 \ln(\nicefrac 4 \delta) (\sqrt d + \sqrt{2\ln(\nicefrac{2n}\alpha)})/\varepsilon
  \end{align*}
\end{proposition}
Note how $\gamma=O(\Upsilon)$ (the only difference is $\sqrt{\log(\tfrac n \alpha)}$).
As a result we get a bound on the regret of Algorithm~\ref{alg:linucb} using the tree-based algorithm using Gaussian noise.
\begin{corollary}
\label{cor:regret_with_Wishart}
Applying Algorithm~\ref{alg:linucb} where the regularizers $H_t$ and $\vec h_t$ are derived by applying the tree-based algorithm where each node holds a symmterized matrix whose entries are sampled iid from $\Normal(0,\sigma_{\rm noise}^2)$ and adding $(\Upsilon+1)\Eye$, assuming $n/d>\Upsilon$, we get a regret bound of
\[O(\sqrt n \left(  \sigma\left(d\log(\nicefrac{nL^2}{d}) + \log(\nicefrac 1 \alpha) \right) + (S\tilde{L}+\tilde{L}^2)\sqrt{d\log(\nicefrac{nL^2}{d})}\tfrac{\log(n)\log(\nicefrac 1 \delta)(\sqrt d + \sqrt{\log(\nicefrac n \alpha)})}{\varepsilon}\right) )\]
in general, and a gap-dependent pseudo-regret bound of
\[O(\tfrac 1 \Delta \left(  \sigma\left(d\log(\nicefrac{nL^2}{d}) + \log(\nicefrac 1 \alpha) \right) + (S\tilde{L}+\tilde{L}^2)\sqrt{d\log(\nicefrac{nL^2}{d})}\tfrac{\log(n)\log(\nicefrac 1 \delta)(\sqrt d + \sqrt{\log(\nicefrac n \alpha)})}{\varepsilon}\right)^2 )\]
\end{corollary}



\section{Lower Bounds}
\label{sec:lower_bounds}

In this section, we present lower bounds for two versions of the problem we deal with in this work. The first, and probably the more obvious of the two, deals with the impossibility of obtaining sub-linear regret for the contextual bandit problem under the standard notion of differential privacy (under continual observation). Here, we assume user $t$ provides a context $c_t$ which actually determines the mapping of the arms into feature vectors: $\phi(c_t, a) \in \Real^d$. The sequence of choice thus made by the learner is $a_1,..., a_n \in \A^n$ which we aim to maintain private. Formally, two sequences $S = \langle (c_1, y_1),..., (c_n,y_n)\rangle$ and $S' = \langle (c'_1, y'_1),..., (c'_n,y'_n)\rangle$ are called neighbors if there exists a single $t$ such that for any $t'\neq t$ we have $(c_{t'},y_{t'}) = (c'_{t'},y'_{t'})$); and an algorithm $A$ is $(\varepsilon,\delta)$-differentially private if for any two neigboring sequences $S$ and $S'$ and any subsets of sequences of actions ${\cal S}\subset \A^n$ it holds that $\Pr[A(S)\in\mathcal{S}] \leq e^\varepsilon \Pr[A(S')\in \mathcal{S}] +\delta$.

\begin{claim}
\label{clm:standard_contextual_DP_implies_linear_regret}
For any $\varepsilon < \ln(2)$ and $\delta < 0.25$, any $(\varepsilon,\delta)$-differentially private algorithm $A$ for the contextual bandit problem must incur pseudo-regret of $\Omega(n)$.
\end{claim}
\begin{proof}
We consider a setting with two arms $\A=\{a^1,a^2\}$ and two possible contexts: $c^1$ which maps $a^1\mapsto \vec\theta^*$ and $a^2 \mapsto -\vec\theta^*$; and $c^2$ which flips the mapping. Assuming $\|\vec\theta^*\|=1$ it is evident we incur a pseudo-regret of $2$ when pulling arm $a^1$ is under context $c^2$ or pulling arm $a^2$ under $c^1$. Fix a day $t$ and a history of previous inputs and arm pulls $H_{t-1}$. Consider a pair of neighboring sequences that agree on the history $H_{t-1}$ and differ just on day $t$ --- in $S$ the context $c_t =c^1$ whereas in $S'$ it is set as $c_t =c^2$. Denote $\mathcal{S}$ as the subset of action sequences that are fixed on the first $t-1$ days according to $H_{t-1}$, have the $t$-th action be $a^1$ and on days $>t$ may have any action. Thus, applying the guarantee of differential privacy w.r.t to $\mathcal{S}$ we get that $\Pr[a_t = a^1 |~S] = \Pr[A(S)\in \mathcal{S}] \leq e^\varepsilon \Pr[a_t=a^1 |~S'] + \delta$. Consider an adversary that sets the context of day $t$ to be either $c^1$ or $c^2$ uniformly at random and independently of other days. The pseudo-regret incurred on day $t$ is thus: $2\cdot \tfrac 1 2 \left( \Pr[a_t=a^2|~S] + \Pr[a_t=a^1|~S'] \right)  \geq (1- \Pr[a_t=a^1|~S]) + e^{-\varepsilon}(\Pr[a_t=a^1|~S]-\delta) = 1 + (e^{-\varepsilon}-1)\Pr[a_t=a^2|~S] - \delta > 1 - 1\cdot \tfrac 1 2 - \tfrac 1 4 = \tfrac 1 4$. As the above applies to any day $t$,  the algorithm's pseudo-regret is $\geq \tfrac n 4$ against such random adversary.
\end{proof}

The second lower bound we show is more challenging. We show that any $\varepsilon$-differentially private algorithm for the classic MAB problem must incur \emph{an additional} pseudo-regret of $\Omega( k\log(n)/\epsilon)$ on top of the standard (non-private) regret bounds.  We consider an instance of the MAB where the leading arm is $a^1$, the rewards are drawn from a distribution over $\{-1,1\}$, and the gap between the means of arm $a^1$ and arm $a\neq a^1$ is $\Delta_a$. Simple calculation shows that for such distributions, the total-variation distance between two distributions whose means are $\mu$ and $\mu-\Delta$ is $\Delta/2$. Fix $\Delta_2, \Delta_3,...,\Delta_k$ as some small constants, and we now argue the following.
\begin{claim}
\label{clm:LB_pulling_any_arms_many_times}
Let $A$ be any $\varepsilon$-differentially private algorithm for the MAB problems with $k$ arms whose expected regret is at most $n^{3/4}$. Fix any arm $a\neq a^1$, whose difference between it and the optimal arm $a^1$ is $\Delta_a$. Then, for sufficiently large $n$s, $A$ pulls arm $a$ at least $\tfrac {\log(n)}{100\varepsilon\Delta_a}$ many times w.p. $\geq \nicefrac 1 2$.
\end{claim}
We comment that the bound $n^{3/4}$ was chosen arbitrarily, and we only require a regret upper bound of $n^{1-c}$ for some $c>0$. Of course, we could have used standard assumptions, where the regret is asymptotically smaller than \emph{any} polynomial; or discuss algorithms of regret $\tilde O(\sqrt n)$ (best minimax regret). Aiming to separate the standard lower-bounds on regret from the private bounds, we decided to use $n^{3/4}$.
As an immediate corollary we obtain the following \emph{private} regret bound:
\begin{corollary}
\label{cor:LB_private_MAB}
The expected pseudo-regret of any $\varepsilon$-differentially private algorithm from the MAB is $\Omega(k\log(n)/\varepsilon)$. Combined with the non-private bound of $\Omega\left( \sum_{a\neq a^1} \tfrac{\log(n)}{\Delta_a}\right)$ we get that the private regret bound of the $\max$ of the two terms, i.e.: $\Omega\left( \tfrac {k\log(n)}{\varepsilon}+\sum\limits_{a\neq a^1} \tfrac{\log(n)}{\Delta_a}\right)$.
\end{corollary}
\begin{proof}
Based on Claim~\ref{clm:LB_pulling_any_arms_many_times}, the expected pseudo-regret is $\geq \sum\limits_{a\neq a^1} \tfrac 1 2 \cdot \Delta_a \tfrac{\log(n)}{100 \varepsilon\Delta_a} = \tfrac{(k-1)\log(n)}{200\varepsilon}$.
\end{proof}
\begin{proof}[Proof of Claim~\ref{clm:LB_pulling_any_arms_many_times}]
Fix arm $a$. Let $\bar P$ be the vector of the $k$-probability distributions associated with the $k$ arms. Denote $E$ as the event that arm $a$ is pulled $<\tfrac{\log(n)}{100\varepsilon\Delta_a} := t_a$ many times. Our goal is to show that $\Pr_{A;~{\rm rewards}\sim \bar P}[E]<\nicefrac 1 2$.

To that end, we postulate a different distribution for the rewards of arm $a$ -- a new distribution whose mean is \emph{greater} by $\Delta_a$ than the mean reward of arm $a^1$. The total-variation distance between the given distribution and the postulated distribution is $\Delta_a$. Denote $\bar Q$ as the vector of distributions of arm-rewards (where only $P_a \neq Q_a$). We now argue that should the rewards be drawn from $\bar Q$, then the event $E$ is  very unlikely: $\Pr_{A;~{\rm rewards}\sim \bar Q}[E] \leq \tfrac 2{\Delta_a}n^{-1/4}$. Indeed, the argument is based on a standard Markov-like argument: the expected pseudo-regret of $A$ is $<n^{3/4}$, yet it is $\geq \Pr_{A;~{\rm rewards}\sim \bar Q}[E]\cdot (n-t_a) \Delta_a \geq n\tfrac{\Delta_a}2\Pr_{A;~{\rm rewards}\sim \bar Q}[E]$, for sufficiently large $n$.


We now apply a beautiful result of~\cite{KarwaVadhanFiniteSampleDP2017} (Lemma 6.1), stating that the ``effective'' group privacy between the case where the $n$ datums of the inputs are drawn iid from either distribution $P$ or from distribution $Q$ is proportional to $\varepsilon n\cdot d_{\rm TV}(P,Q)$. In our case, the key point is that we only consider this change \emph{under the event $E$}, thus the number of samples we need to redraw from the distribution $P_a$ rather than $Q_a$ is strictly smaller than $t_a$, and the elegant coupling arguming of~\cite{KarwaVadhanFiniteSampleDP2017} reduces it to $6\Delta_a \cdot t_a$. To better illustrate the argument, consider a version of the algorithm described in~\cite{KarwaVadhanFiniteSampleDP2017}: it generates a collection of $t_a$ \emph{pairs} of points, the left ones are iid samples from $P_a$ and the right ones are iid samples from $Q_a$, and, in expectation, in $(1-\Delta_a)$ fraction of the pairs the right- and the left-samples are identical. This algorithm declares {\tt success} when it never runs out of samples and {\tt failure} otherwise. Due to group privacy, oracle invocations coming from $A$ to this algorithm that either always ask for the left sample or always ask for the right sample should therefore trigger {\tt success} with similar probability, different only up to a factor of $\exp(\epsilon \Delta_a t_a)$. And seeing as {\tt success} is quite unlikely when asked only for right-samples, it is only fairly unlikely when asked only for left-samples.

Formaly, we conclude the proof by applying the result of~\cite{KarwaVadhanFiniteSampleDP2017} to infer that $\Pr_{A;~{\rm rewards}\sim \bar P}[E] \leq \exp(6\varepsilon\Delta_a t_a)\Pr_{A;~{\rm rewards}\sim \bar Q}[E] \leq \exp(0.06 \log(n)) \cdot \tfrac 2{\Delta_a}n^{-1/4} = n^{-0.19}\tfrac 2{\Delta_a} \leq \nicefrac 1 2$ for sufficiently large $n$s, proving the required.
\end{proof}

\section{Conclusions and Future Work}
\label{sec:conclusions}

\subsection{Future Work}
\label{sec:future-work}

\todo[inline]{Consider non-contextual bandits; only the rewards are
  private.  Does this help with the counter-example against optimism
  in Tor's paper?}


\bibliographystyle{plainnat}
{\small\bibliography{references,zotero-references}}


\cleardoublepage
\appendix
\phantomsection
\addcontentsline{toc}{chapter}{Supplementary Material}
\begin{center}
  \LARGE\bf Supplementary Material
\end{center}


\section{Additional Background Information}
\label{apx_sec:more_background}

\subsection{Differential Privacy}
\label{apx_subsec:DP}

In the offline setting, a dataset $D$ is a $n$-tuple of elements from some universe $\mathcal{U}$. Two datasets are called neighbors if they differ just on a single element. An algorithm $A$ is said to be $(\varepsilon,\delta)$-differentially private if for any pair of neighboring datasets $D$ and $D'$ and any subset of possible outputs $\mathcal{S}$ we have that $\Pr[A(D)\in\mathcal{S}]\leq e^\varepsilon\Pr[A(D')\in\mathcal{S}] + \delta$. A common technique~\cite{DworkOurData2006} for approximating the value of a query $f$ on dataset $D$ is to first find its $L_2$-sensitivity, $GS_2 :=\max_{D,D'\textrm{ neighboring}} \|f(D)-f(D')\|_2$, and then add Gaussian noise of $0$-mean and variance $\tfrac{2GS_2^2\ln(\nicefrac 2 \delta)}{\varepsilon^2}$.

\subsection{The Tree-Based Mechanism}

Assume for simplicity that $n=2^i$ for some positive integer $i$. Let $T$ be a complete binary tree with its leaf nodes being $l_1,...,l_n$. Each internal
node $x\in T$ stores the sum of all the leaf nodes in the tree
rooted at $x$. First notice that one can compute any partial sum $\sum_{j=1}^i l_i$ using at most $m:=\lceil\log(n)+1\rceil$ nodes of $T$. Second, notice that for any two neighbor-
ing data sequences $D$ and $D'$ the partial sums stored at no more than $m$ nodes in $T$ are different. Thus, if the count in each node preserves $(\varepsilon_0,\delta_0)$-differential privacy, using the advanced composition of~\cite{DworkBoosting2010} we get that the entire algorithm is $\left(O(m\varepsilon_0^2+\varepsilon_0\sqrt{2m\ln(\nicefrac 1{\delta'})}), m\delta_0 + \delta'\right)$-differentially private. Aternatively, to make sure the entire tree is $(\varepsilon,\delta)$-differentially private, it suffices to set $\varepsilon_0 = \varepsilon / \sqrt{8m\ln(\nicefrac 2 \delta)}$ and $\delta_0= \tfrac \delta{2m}$ (with $\delta'=\nicefrac \delta 2$).

\section{Proofs from Section~\ref{sec:LinUCB}}

The proofs in this section are based on those of
\citet{AbbasiYadkoriImprovedAlgorithmsLinear2011}, modified to allow a
different regularizer at each round.

\CalcBeta*

\begin{proof}
  By definition, $\tilde{\vec\theta}_t = \inv{V_t} \tilde{\vec u}_t$,
  $\tilde{\vec u}_t = \vec u_t + \vec h_t$, and $\vec u_t = \transp{\vec
    X_{t-1}} \vec y_{t-1}$, so that
  \begin{align*}
    \vec\theta^* - \tilde{\vec\theta}_t
    &= \vec\theta^* - \inv{V_t}(\transp{\vec X_{t-1}} \vec y_{t-1} + \vec h_t) \\
    &= \vec\theta^* - \inv{V_t}(\XtX{\vec X_{t-1}} \vec\theta^*
      + \transp{\vec X_{t-1}} \vec\eta_{t-1} + \vec h_t)
    &\text{since } \vec y_{t-1} = \vec X_{t-1} \vec\theta^* + \vec\eta_{t-1} \\
    &= \vec\theta^* - \inv{V_t}(V_t \vec\theta^* - H_t \vec\theta^* + \vec z_t + \vec h_t)
    &\text{defining } \vec z_t \defeq \transp{\vec X_{t-1}} \vec\eta_{t-1} \\
    &= \inv{V_t}(H_t\vec\theta^* - \vec z_t - \vec h_t) \\
    \shortintertext{Multiplying both sides by $V_t^{1/2}$ gives}
    V_t^{1/2}(\vec\theta^* - \tilde{\vec\theta}_t)
    &= V_t^{-1/2}(H_t\vec\theta^* - \vec z_t - \vec h_t) \\
    \norm{V_t}{\vec\theta^* - \tilde{\vec\theta}_t}
    &= \norm{\inv{V_t}}{H_t\vec\theta^* - \vec z_t - \vec h_t}
    &\text{applying $\norm{}{\wildcard}$ to both sides}\\
    &\le \norm{\inv{V_t}}{H_t\vec\theta^*} + \norm{\inv{V_t}}{z_t}
      + \norm{\inv{V_t}}{\vec h_t} &\text{triangle inequality} \\
    &\le \norm{\inv{V_t}}{\vec z_t} + \norm{\inv{H_t}}{H_t\vec\theta^*}
      + \norm{\inv{H_t}}{\vec h_t}
    &\text{by \cref{claim:psd-matrix-props} since } V_t \succeq H_t \\
    &= \norm{\inv{(G_t+\rho_{\min}\Eye)}}{\vec z_t} + \norm{H_t}{\vec\theta^*} +
      \norm{\inv{H_t}}{\vec h_t},
    &\text{since } V_t \succeq G_t + \rho_{\min}\Eye.
  \end{align*}
  We use a union bound over all $n$ rounds to bound
  $\norm{H_t}{\vec\theta^*} \le
  \sqrt{\norm{}{H_t}}\norm{}{\vec\theta^*} \le S\sqrt{\rho_{\max}}$
  and $\norm{\inv{H_t}}{\vec h_t} \le \gamma$ with probability at
  least $1-\alpha/2$.  Finally, by the ``self-normalized bound for
  vector-valued martingales'' of
  \citet[Theorem~1]{AbbasiYadkoriImprovedAlgorithmsLinear2011}, with
  probability $1-\alpha/2$ for all rounds simultaneously
  \begin{align*}
    \norm{\inv{(G_t + \rho_{\min}\Eye)}}{\vec z_t}
    &\le \sigma\sqrt{2\log\frac{2}{\alpha} + \log\frac{\det(G_t+\rho_{\min}\Eye)}{\det\rho_{\min}\Eye}}
      \le \sigma\sqrt{2\log\frac{2}{\alpha} + \log\det V_t - d\log\rho_{\min}}.
  \end{align*}
  It only remains to show the upper-bound on each $\beta_t$.  By
  \cref{claim:psd-matrix-props}, we have
  $\det V_t = \det(G_t+H_t) \le \det(G_t+\rho_{\max}\Eye)$ and
  \begin{align*}
    \log\det V_t
    &\le \log\det(G_t + \rho_{\max}\Eye)
    \le d\log(\rho_{\max} + tL^2/d).
  \end{align*}
  using the trace-determinant inequality as in the proof of
  \cref{lemma:elliptical-potential}.  All the $\beta_t$ are therefore
  bounded by the constants
  \begin{align*}
    \bar\beta_t &\defeq \sigma\sqrt{2\log\frac{2}{\alpha} + d\log\paren*{\frac{\rho_{\max}}{\rho_{\min}}
                 + \frac{tL^2}{d\rho_{\min}}}} + S\sqrt{\rho_{\max}} + \gamma. \qedhere
  \end{align*}
\end{proof}

\begin{lemma}[LinUCB Regret]\label{lemma:linucb-regret}
  Suppose \cref{ass:meanreward-bound,ass:psd-reg} hold (i.e.\
  $\abs{\innerp{\vec\theta^*}{\vec x}} \le 1$ and all $H_t \succeq 0$)
  and $\bar\beta_n \ge \max\set{\beta_1,\dotsc,\beta_n,1}$.  If all
  the confidence sets $\E_t$ contain $\vec\theta^*$ (i.e.,
  $\norm{V_t}{\vec\theta^* - \tilde{\vec\theta}_t} \le
  \beta_t$), then the pseudo-regret of \cref{alg:linucb} is bounded by
  \begin{align*}
    \widehat{R}_n &\le \bar\beta_n\sqrt{4n \sum_{t=1}^n \min\set{1,
                   \norm{\inv{V_t}}{\vec x_t}^2}}.
  \end{align*}
\end{lemma}

\begin{proof}
  At every round $t$, \cref{alg:linucb} selects an ``optimistic''
  action $\vec x_t$ satisfying
  \begin{align}\label{eq:optimistic-action}
    (\vec x_t,\bar{\vec\theta}_t)
    &\in \argmax_{(\vec x, \vec\theta) \in\D_t\times\E_t} \innerp{\vec\theta}{\vec x}.
  \end{align}
  Let $\vec x_t^* \in \argmax_{\vec x\in\D_t}\innerp{\vec\theta^*}{\vec x}$ be an optimal
  action and $r_t = \innerp{\vec\theta^*}{\vec x_t^* - \vec x_t}$ be the immediate
  pseudo-regret suffered for round $t$:
  \begin{align*}
    r_t &= \innerp{\vec\theta^*}{\vec x_t^*} - \innerp{\vec\theta^*}{\vec x_t} \\
        &\le \innerp{\bar{\vec\theta}_t}{\vec x_t} - \innerp{\vec\theta^*}{\vec x_t}
        &\text{from \eqref{eq:optimistic-action} since }
          (\vec x_t^*,\vec\theta^*) \in \D_t\times\E_t \\
        &= \innerp{\bar{\vec\theta}_t - \vec\theta^*}{\vec x_t} \\
       &= \innerp{V_t^{1/2}(\bar{\vec\theta}_t - \vec\theta^*)}{V_t^{-1/2} \vec x_t}
        &\text{since } V_t \succeq H_t \succeq 0 \\
        &\le \norm{V_t}{\bar{\vec\theta}_t - \vec\theta^*}\norm{\inv{V_t}}{\vec x_t}
        &\text{by Cauchy-Schwarz}\\
        &\le \paren[\big]{\norm{V_t}{\bar{\vec\theta}_t - \tilde{\vec\theta}_t}
          + \norm{V_t}{\vec\theta^* - \tilde{\vec\theta}_t}}
          \norm{\inv{V_t}}{\vec x_t}
        &\text{by the triangle inequality}\\
        &\le 2\beta_t\norm{\inv{V_t}}{\vec x_t}
        &\text{since } \bar{\vec\theta}_t, \vec\theta^* \in \E_t \\
        &\le 2\bar\beta_n\norm{\inv{V_t}}{\vec x_t}
        &\text{since } \bar\beta_n \ge \beta_t.
  \end{align*}
  From our assumptions that the mean absolute reward is at most $1$
  and $\bar\beta_n \ge 1$, we also get that $r_t \le 2 \le 2\bar\beta_n$.
  Putting these together,
  \begin{align}\label{eq:regret-bound-oneround}
    r_t &\le 2\bar\beta_n \min\set{1, \norm{\inv{V_t}}{\vec x_t}}
  \end{align}
  Now we apply Jensen's inequality as follows:
  \begin{align*}
    \widehat R_n^2 &= n^2 \paren[\Big]{\sum_{t=1}^n \frac{r_t}{n}}^2
                   \le n^2 \sum_{t=1}^n \frac{r_t^2}{n} = n\sum_{t=1}^nr_t^2
                   \le 4n\bar\beta_n^2\sum_{t=1}^n\min\set{1,\norm{\inv{V_t}}{\vec x_t}^2}.
  \end{align*}
  Taking square roots completes the proof.
\end{proof}

\begin{lemma}[Elliptical Potential]\label{lemma:elliptical-potential}
  Let $\vec x_1,\dotsc,\vec x_n \in \Real^d$ be vectors with each
  $\norm{}{\vec x_t} \le L$.  Given a positive definite matrix
  $U_1\in\Real^{d\times d}$, define
  $U_{t+1} \defeq U_t + \vec x_t \transp{\vec x_t}$ for all $t$.  Then
  \begin{align*}
    \sum_{t=1}^n\min\set{1, \norm{\inv{U_t}}{\vec x_t}^2}
    &\le 2\log\frac{\det U_{n+1}}{\det U_1}
      \le 2d\log\frac{\tr U_1+nL^2}{d\det^{1/d} U_1}.
  \end{align*}
\end{lemma}

\begin{proof}
  We use the fact that $\min\set{1, u} \le 2\log(1+u)$ for any
  $u \ge 0$:
  \begin{align*}
    \sum_{t=1}^n\min\set{1, \norm{\inv{U_t}}{\vec x_t}^2}
    &\le 2 \sum_{t=1}^n \log(1 + \norm{\inv{U_t}}{\vec x_t}^2).
  \end{align*}
  We will show that this last summation is $2\log(\det U_{n+1}/\det
  U_n)$.  For all $t$, we have
  \begin{align*}
    U_{t+1} &= U_t + \vec x_t \transp{\vec x_t}
             = U_t^{1/2}
             \paren[\big]{I + U_t^{-1/2} \vec x_t \transp{\vec x_t} U_t^{-1/2}}
             U_t^{1/2} \\
    \det U_{t+1} &=\det U_t
                  \det\paren[\big]{I +
                  U_t^{-1/2} \vec x_t \transp{\vec x_t}
                  U_t^{-1/2}}.
  \end{align*}
  Consider the eigenvectors of the matrix $I + \vec y \transp{\vec y}$
  for an arbitrary vector $\vec y \in \Real^d$.  We know that $\vec y$
  itself is an eigenvector with eigenvalue $1+\norm{}{\vec y}^2$:
  \begin{align*}
    (I + \vec y \transp{\vec y}) \vec y
    &= \vec y + \vec y \innerp{\vec y}{\vec y} = (1+\norm{}{\vec y}^2)\vec y.
  \end{align*}
  Moreover, since $I + \vec y \transp{\vec y}$ is symmetric, every
  other eigenvector $\vec u$ is orthogonal to $\vec y$, so that
  \begin{align*}
    (I + \vec y \transp{\vec y}) \vec u
    &= \vec u + \vec u \innerp{\vec y}{\vec u} = \vec u.
  \end{align*}
  Therefore the only eigenvalues of $I + \vec y \transp{\vec y}$ are
  $1+\norm{}{\vec y}^2$ (with eigenvector $\vec y$) and 1.  In our
  case $\vec y = U_t^{-1/2} \vec x_t$ and
  $\norm{}{\vec y}^2 = \transp{\vec x_t} \inv{U_t} \vec x_t =
  \norm{\inv{U_t}}{\vec x_t}^2$, so we get our first inequality:
  \begin{align*}
    \det U_{n+1} &= \det U_1 \prod_{t=1}^n(1 + \norm{\inv{U_t}}{\vec x_t}^2) \\
    2\log\frac{\det U_{n+1}}{\det U_1}
            &= 2\sum_{t=1}^n\log(1+\norm{\inv{U_t}}{\vec x_t}^2).
  \end{align*}
  To get the second inequality, we apply the arithmetic-geometric
  mean inequality to the eigenvalues $\lambda_i$ of $U_n$:
  \begin{align*}
    \det U_n &= \prod_{i=1}^d \lambda_i
              \le \paren[\Big]{\frac{1}{d} \sum_{i=1}^d \lambda_i}^d
              = \paren{(1/d)\tr U_n}^d
              \le \paren{(\tr U_1 + nL^2)/d}^d \\
    2\log\frac{\det U_n}{\det U_1}
            &\le 2d \log\frac{\tr U_1 + nL^2}{d\det^{1/d}U_1}
              \qedhere
  \end{align*}
\end{proof}

\ThmLinUCBRegret*

\begin{proof}
  We restrict ourselves to the event that all the confidence
  ellipsoids contain $\vec\theta^*$ and all
  $\rho_{\min}\Eye \preceq H_t \preceq \rho_{\max}\Eye$.
  \Cref{prop:calc-beta} assures us that this happens with probability at least
  $1-\alpha$, and furthermore gives us the bound $\beta_t \le \bar\beta_n$:
  \begin{align*}
    \bar\beta_n &\defeq \sigma\sqrt{2\log\frac{2}{\alpha} + d\log\paren*{\frac{\rho_{\max}}{\rho_{\min}}
                 + \frac{nL^2}{d\rho_{\min}}}}
                 + S\sqrt{\rho_{\max}} + \gamma.
  \end{align*}
  Next, we have
  $\norm{\inv{V_t}}{\vec x_t} \le
  \norm{\inv{(G_t+\rho_{\min}\Eye)}}{\vec x_t}$, which applied to the
  result of \cref{lemma:linucb-regret} gives, using
  \cref{lemma:elliptical-potential}
  \begin{align*}
    \widehat{R}_n
    &\le \bar\beta_n\sqrt{8dn\log\paren*{1+\frac{nL^2}{d\rho_{\min}}}} \\
    &\le \sqrt{8n}\brck*{\sigma\paren*{2\log\frac{2}{\alpha}
      + d\log\paren*{\frac{\rho_{\max}}{\rho_{\min}} + \frac{nL^2}{d\rho_{\min}}}}
      + (S\sqrt{\rho_{\max}}+\gamma)\sqrt{d\log\paren*{1+\frac{nL^2}{d\rho_{\min}}}}} \qedhere
  \end{align*}
\end{proof}

\ThmLinUCBGapRegret*

\begin{proof}
  Because of the gap assumption, for every round $t$ if the per-round
  pseudo-regret $r_t \neq 0$ then $r_t \ge \Delta$.  We use this fact
  to decompose the regret in a different way than we did in
  \cref{lemma:linucb-regret}.  The rest of the proof is similar to
  that of \cref{thm:linucb-regret}
  \begin{align*}
    \widehat R_n
    &= \sum_{t\in B_n} r_t \le \sum_{t\in B_n} \frac{r_t^2}{\Delta} \\
    &\le \frac{4}{\Delta} \bar\beta_n^2 \sum_{t\in B_n} \min\set{1, \norm{\inv{V_t}}{\vec x_t}^2}
    &\text{from~\eqref{eq:regret-bound-oneround}} \\
    &\le \frac{8}{\Delta} \bar\beta_n^2d\log\paren*{1+\frac{nL^2}{d\rho_{\min}}} \\
    &\le \frac{8}{\Delta} \brck*{\sigma\paren*{2\log\frac{2}{\alpha}
      + d\log\paren*{\frac{\rho_{\max}}{\rho_{\min}} + \frac{nL^2}{d\rho_{\min}}}}
      + (S\sqrt{\rho_{\max}}+\gamma)\sqrt{d\log\paren*{1+\frac{nL^2}{d\rho_{\min}}}}}^2
    &&\qedhere
  \end{align*}

\end{proof}

\section{Privacy Proofs}

\PropWishartTails*
\begin{proof}
Seeing as $H_t\sim\Wishart_d(\tilde L^2 \Eye, mk)$, straight-forward bounds on the eigenvalues of the Wishart distribution (e.g.~\cite{SheffetPrivateApproxRegression2015}, Lemma~A.3) give that w.p. $\geq 1- \tfrac \alpha{2n}$ all of the eigen values of $H_t$ lie in the interval $\tilde L^2 \paren*{\sqrt{mk} \pm \paren*{\sqrt{d} + \sqrt{2\ln(\nicefrac{8n}{\alpha})}}}^2$. To estimate $\|\vec h_t\|_{H_t^{-1}}$ we draw back to the definition of the Wishart distribution as the $2^{\rm nd}$-moment matrix of Gaussians with itself. Denote this matrix of Gaussians as $[Z ; \vec z]$ where $Z\in \Real^{mk\times d}$ and $\vec z \in \Real^{mk}$, and we have that $H_t = \transp Z Z$ and $h_t = \transp Z \vec z$, thus $\|\vec h_t\|_{H_t^{-1}} = \sqrt{ \transp{\vec z} Z (\transp Z Z)^{-1}  \transp Z \vec z }$. The matrix $Z (\transp Z Z)^{-1}  \transp Z$ is a projection matrix onto a $d$-dimensional space, and projecting the spherical Gaussian $\vec z$ onto this subspace results in a $d$-dimensional sphrical Gaussian. Using concentration bounds on the $\chi^2$-distribution~\ref{claim:chi2-tails} we have that w.p. $\geq 1- \tfrac \alpha{2n}$ it holds that $\|\vec h_t\|_{H_t^{-1}}\leq \sqrt{d} + \sqrt{2\ln(\tfrac{2n}{\alpha})}$. Taking a union bound over all $n$ days concludes the proof.
\end{proof}

\begin{theorem}[{\citealp[Theorem~4.1]{SheffetPrivateApproxRegression2015}}]%
  \label{thm:wishart-dp}%
  Fix $\varepsilon\in(0,1)$ and $\delta\in(0,1/e)$.  Let
  $A\in\Real^{n\times p}$ be a matrix whose rows have $l_2$-norm
  bounded by $\tilde L$.  Let $W$ be a matrix sampled from the
  $d$-dimensional Wishart distribution with $k$ degrees of freedom
  using the scale matrix $\tilde L^2\Eye{p}$ (i.e.\
  $W \sim \Wishart_p(\tilde L^2\Eye{p}, k)$) for
  $k \ge p + \floor[\big]{\frac{14}{\varepsilon^2}\cdot 2\log(4/\delta)}$.
  Then outputting $\XtX{A} + N$ is
  $(\varepsilon,\delta)$-differentially private with respect to
  changing a single row of $A$.
\end{theorem}

\section{Useful Results}

\begin{claim}[{\citealp[Theorem~7.8]{ZhangMatrixTheory2011}}]%
  \label{claim:psd-matrix-props}%
  If $A \succeq B \succeq 0$, then
  \begin{enumerate}[nolistsep]
  \item $\rank(A) \ge \rank(B)$
  \item $\det A \ge \det B$
  \item $\inv{B} \succeq \inv{A}$ if $A$ and $B$ are nonsingular.
  \end{enumerate}
\end{claim}

\begin{claim}[{\citealp[Corollary to
    Lemma~1,][p.~1325]{LaurentAdaptiveEstimation2000}}]%
  \label{claim:chi2-tails}
  If $U\sim\chi^2(d)$ and $\alpha\in(0,1)$,
  \begin{align*}
    \Prob[\bigg]{U \ge d + 2\sqrt{d\ln\frac{1}{\alpha}} + 2\ln\frac{1}{\alpha}} &\le \alpha,
    & \Prob[\bigg]{U \le d - 2\sqrt{d\ln\frac{1}{\alpha}}} &\le \alpha.
  \end{align*}
\end{claim}

\begin{claim}[{\citealp[Adaptation
    of][Corollary~5.35]{VershyninRandomMatrices2010}}]%
  \label{claim:gaussian-matrix-tails}
  Let $A$ be an $n\times d$ matrix whose entries are independent
  standard normal variables.  Then for every $\alpha\in(0,1)$, with
  probability at least $1-\alpha$ it holds that
  \begin{align*}
    \sigma_{\min}(A), \sigma_{\max}(A) &\in \sqrt{n} \pm (\sqrt{d} + \sqrt{2\ln(2/\alpha)})
  \end{align*}
\end{claim}

\begin{claim}[{\citealp[Lemma~A.3]{SheffetPrivateApproxRegression2015}}]%
  \label{claim:wishart-tails}
  Fix $\alpha\in(0,1/e)$ and let $W\sim\Wishart_d(V, k)$ with $\sqrt{m}
  > \sqrt{d} + \sqrt{2\ln(2/\alpha)}$.  Then with probability at least
  $1-\alpha$ it holds that for every $j = 1,2,\dotsc,d$:
  \begin{align*}
    \sigma_j(W) &\in \paren*{\sqrt{m} \pm \paren[\Big]{\sqrt{d} + \sqrt{2\ln(2/\alpha)}}}^2 \sigma_j(V).
  \end{align*}
\end{claim}

%%%%%%%%%%%%%%%%%%%%%%%%%%%%%%%%%%%%%%%%%%%%%%%%%%%%%%%%%%%%%%%%%%%%%%%%%%%
\cleardoublepage
\todo[inline]{Remove following material from final version.}

\section{Ellipsoidal Confidence Sets}
\label{sec:ellips-conf-bounds}

Throughout this section we will fix some particular round $t \le n$
and suppress the subscripted $t$.

We will consider confidence sets for the value of $\theta^*$ that are
ellipsoidal with centre $\hat{\theta}$, of the form
$\set{\theta\in\Real^d \given \norm{V}{\theta-\hat{\theta}}^2 \le
  \beta}$.  Then the upper confidence bound of an action $x$ is
\begin{align*}
  \UCB(x) &= \max_{\theta: \norm{V}{\theta-\hat\theta}^2 \le \beta} \innerp{x}{\theta}
\end{align*}
We solve this constrained optimization problem by the method of
Lagrange multipliers:
\begin{align*}
  \mathcal{L}(\theta, \lambda) &= \innerp{x}{\theta} - \lambda(\norm{V}{\theta-\hat\theta}^2 - \beta) \\
  \nabla_\theta\mathcal{L}(\theta, \lambda) &= \transp{x} - \lambda\transp{(\theta-\hat\theta)}V = 0 \\
  \theta &= \hat\theta + \frac{{\inv{V}x}}{\lambda}
  \shortintertext{To find $\lambda$:}
  \beta &= \norm{V}{\theta-\hat\theta}^2
          = \norm[\bigg]{V}{\frac{\inv{V}x}{\lambda}}^2
          = \frac{1}{\lambda^2}\transp{x}\inv{V}x \\
  \lambda &= \norm{\inv{V}}{x} / \sqrt\beta
  \intertext{Which we substitute into the expression above:}
  \UCB(x) &= \innerp[\bigg]{x}{\hat\theta + \frac{\sqrt\beta}{\norm{\inv{V}}{x}}\inv{V}x} \\
          &= \innerp{x}{\hat\theta} + \sqrt\beta \norm{\inv{V}}{x}.
\end{align*}
We have therefore shown that if we construct an ellipsoidal confidence
set with an appropriate $V$ and $\beta$, then the requirements of
\cref{thm:linucb-regret} will be satisfied with the above $\UCB$
index.

Let $X_s \in \Real^d$ be the action vector selected by the algorithm at
time $s$, so that $X$ is a $t \times d$ matrix.  The \emph{Gram matrix} is given by
$G \defeq \transp{X}X \in \Real^{d \times d}$.  Let $y_s$ be the
reward received at time $s$, so that $y\in\Real^t$.

We will take $\hat{\theta}$ to be a regularized solution of the
least-squares system
\begin{align*}
  X\theta &\approx y,
\end{align*}
regularized by the symmetric positive definite matrix $H \succeq 0$:
\begin{align*}
  \hat{\theta} &= \argmin_\theta \frac{1}{2}\norm{}{X\theta - y}^2 + \frac{1}{2}\norm{H}{\theta}^2 \\
               &= \inv{\paren{\transp{X}X + H}} \transp{X}y \\
               &= \inv{\paren{\transp{X}X + H}} \transp{X} \paren{X\theta^* + \eta} \\
               &= \theta^* + \inv{\paren{\transp{X}X + H}}\paren{\transp{X}\eta - H\theta^*} \\
               &= \theta^* + \inv{V}Z - \inv{V}H\theta^*
\end{align*}
where we used the fact that $y = X\theta^* + \eta$ (and $\eta_s$ is the
noise in the reward at round $s$) and we defined $V \defeq \transp{X}X
+ H$ and $Z\defeq X^T\eta$.

\begin{lemma}\label{lemma:subgaussian-z}
  Under \cref{assumption:subgaussian-noise} and for any
$\lambda\in\Real^d$, the random variable $Z = \transp{X}\eta$ satisfies
  \begin{align*}
    \Ex{\exp(\transp{\lambda}Z)} \le \exp\paren{\norm{G}{\lambda}^2/2}.
  \end{align*}

  \begin{proof}
    We have
    \begin{align*}
      \Ex{\exp(\transp{\lambda}Z)}
      &= \Ex{\exp(\transp{\lambda}\transp{X}\eta)} \\
      &= \Ex[\Big]{\Ex[\Big]{\prod_{s=1}^t \exp\paren{\transp{\lambda}X_s\eta_s} \given \mathcal{F}_t}} \\
      &= \Ex[\Big]{\Ex[\Big]{\exp\paren{\transp{\lambda}X_t\eta_t} \given \mathcal{F}_t} \prod_{s=1}^{t-1} \exp\paren{\transp{\lambda}X_s\eta_s}} \\
      &\le \Ex[\Big]{\exp\paren{{(\transp{\lambda}X_t)}^2/2}
        \prod_{s=1}^{t-1} \exp\paren{\transp{\lambda}X_s\eta_s}}
      &\text{since $\eta_t$ is conditionally 1-subgaussian given $\mathcal{F}_t$}\\
      &\le \Ex[\Big]{\prod_{s=1}^t\exp({(\transp{\lambda}X_s)}^2/2)}
      &\text{similarly conditioning on $\mathcal{F}_{t-1},\mathcal{F}_{t-2},\dotsc$}\\
      &= \Ex[\Big]{\exp\paren[\Big]{\sum_{s=1}^t\frac{\transp{\lambda}X_s\transp{X_s}\lambda}{2}}} \\
      &= \exp\paren[\Big]{\frac{1}{2}\transp{\lambda}\transp{X}X\lambda}
        = \exp(\norm{G}{\lambda}^2/2)
      &\qedhere
    \end{align*}
  \end{proof}
\end{lemma}


\begin{lemma}\label{lemma:z-norm-bounded-whp}
  Suppose for any $\lambda\in\Real^d$ the random variable
  $Z=\transp{X}\eta$ satisfies
  $\Ex{\exp(\transp{\lambda}Z)} \le \exp(\norm{G}{\lambda}^2/2)$ (see,
  for example, \cref{lemma:subgaussian-z}).  Let
  $0 \prec H \in \Real^{d\times d}$ be any symmetric positive definite
  matrix.  Then for any $0 < \delta \le 1$, we have
  \begin{align*}
    \Prob[\Bigg]{\norm{\inv{(G+H)}}{Z} \ge \sqrt{2\log\frac{1}{\delta} + \log\frac{\det(G+H)}{\det H}}} &\le \delta.
  \end{align*}

  \begin{proof}
    We define
    \begin{align*}
      M_\lambda &\defeq \exp\paren[\Big]{\transp{\lambda}Z - \frac{1}{2}\transp{\lambda}G\lambda}.
    \end{align*}
    Now consider $\lambda\sim\mathcal{N}(0, \inv{H})$ to be a random
    variable with the density function $h(\lambda)$.  Define
    \begin{align*}
      \bar{M}
      &\defeq \int M_\lambda \,h(\lambda)\,d\lambda \\
      &= \frac{1}{\sqrt{{(2\pi)}^d \det\inv{H}}} \int\exp\paren[\Big]{\transp{\lambda}Z
        - \frac{1}{2}\transp{\lambda}G\lambda
        - \frac{1}{2}\transp{\lambda}H\lambda
        } \,d\lambda.
    \end{align*}
    We complete the square in the integrand:
    \begin{align*}
      \transp{\lambda}Z - \frac{1}{2}\transp{\lambda}G\lambda - \frac{1}{2}\transp{\lambda}H\lambda
      &= \frac{1}{2}\transp{Z}\inv{(G+H)}Z - \frac{1}{2}\transp{(\lambda - \inv{(G+H)}Z)}(G+H)(\lambda-\inv{(G+H)}Z) \\
      &= \frac{1}{2} \norm{\inv{(G+H)}}{Z}^2 - \frac{1}{2}\norm{G+H}{\lambda-\inv{(G+H)}Z}^2
    \end{align*}
    which we substitute into the previous equation to get
    \begin{align*}
      \bar{M}
      &= \frac{\exp\paren[\big]{\frac{1}{2} \norm{\inv{(G+H)}}{Z}^2}}{\sqrt{{(2\pi)}^d \det\inv{H}}}
        \int \exp\paren[\Big]{-\frac{1}{2} \norm{G+H}{\lambda-\inv{(G+H)}Z}^2} \,d\lambda \\
      &= \sqrt{\frac{\det H}{\det(G+H)}} \exp\paren[\Big]{\frac{1}{2}\norm{\inv{(G+H)}}{Z}^2}, \\
      \log\bar{M} &= \frac{1}{2}\norm{\inv{(G+H)}}{Z}^2 + \frac{1}{2}\log\frac{\det H}{\det(G+H)}.
    \end{align*}
    Our assumption gives us
    \begin{align*}
      \Ex{\exp(\transp{\lambda}Z)} &\le \exp\paren[\Big]{\frac{1}{2}\transp{\lambda}G\lambda} \\
      \Ex[\bigg]{\frac{\exp(\transp{\lambda}Z)}{\exp(\frac{1}{2}\transp{\lambda}G\lambda)}}
      &= \Ex{M_\lambda} \le 1, &\text{for all }\lambda\in\Real^d.
    \end{align*}
    Now, by Fubini's theorem we can exchange the expectation with the
    integral over $\lambda$:
    \begin{align*}
      \Ex{\bar{M}} &= \Ex[\Big]{\int M_\lambda\,h(\lambda)\,d\lambda} = \int \Ex{M_\lambda} \,h(\lambda)\,d\lambda \le 1
    \end{align*}
    and use a Chernoff bound:
    \begin{align*}
      \Prob{\log(\bar{M}) \ge u} &\le \exp(-u) \\
      \Prob[\Big]{\frac{1}{2}\norm{\inv{(G+H)}}{Z}^2 \ge u + \frac{1}{2}\log\frac{\det(G+H)}{\det H}} &\le \exp(-u) \\
      \Prob[\Big]{\norm{\inv{(G+H)}}{Z} \ge \sqrt{2u + \log\frac{\det(G+H)}{\det H}}} &\le \exp(-u).
    \end{align*}
    Substituting $u = \log(1/\delta)$ completes the proof.
  \end{proof}

\end{lemma}

%===============================================================================
% \hrule

% We see that
% \begin{align*}
%   \max_{\lambda\in\Real^d} \exp\paren[\Big]{\transp{\lambda}Z - \frac{1}{2}\transp{\lambda}G\lambda}
%   &= \exp\paren[\Big]{\max_{\lambda\in\Real^d} \transp{\lambda}Z - \frac{1}{2}\transp{\lambda}G\lambda}.
% \end{align*}
% The maximizer of this expression is given by taking the gradient and
% setting it to zero:
% \begin{align*}
%   \nabla_\lambda\brck[\Big]{\transp{\lambda}Z - \frac{1}{2}\transp{\lambda}G\lambda}_{\lambda=\lambda^*} = Z - G\lambda^* &= 0 \\
%   \lambda^* &= \inv{G}Z
% \end{align*}
% and thus
% \begin{align*}
%   \max_{\lambda\in\Real^d} \exp\paren[\Big]{\transp{\lambda}Z - \frac{1}{2}\transp{\lambda}G\lambda}
%   &= \exp\paren[\Big]{\transp{Z}\inv{G}Z - \frac{1}{2}\transp{Z}\inv{G}Z} \\
%   &= \exp\paren[\Big]{\frac{1}{2}\norm{\inv{G}}{Z}^2}.
% \end{align*}
% We define
% \begin{align*}
%   M_\lambda \defeq \exp\paren[\Big]{\transp{\lambda}Z - \frac{1}{2}\transp{\lambda}G\lambda}
% \end{align*}
% and use Chernoff's bound:
% \begin{align*}
%   \Pr\paren[\Big]{\frac{1}{2}\norm{\inv{G}}{Z}^2 > u}
%   &= \Pr\paren[\Big]{\max_{\lambda\in\Real^d}M_\lambda > u} \\
%   &\le \exp(-u) \Ex[\Big]{\max_{\lambda\in\Real^d} M_\lambda}.
% \end{align*}
% \hrule
%===============================================================================

\subsection{Ridge Regression}
\label{sec:ridge-regression}

We will now instantiate a concrete algorithm based on \emph{ridge
  regression}, i.e.\ using the regularizer $H=\rho I$ for some
$\rho>0$.  Thus we will have
\begin{align*}
  V_t &\defeq \rho I + \sum_{s=1}^t X_s\transp{X_s}
  \shortintertext{and}
  \hat{\theta}_t - \theta^* &= \inv{V_t}Z_t - \inv{V_t}H\theta^* \\
                          &= \inv{V_t}Z_t - \rho\inv{V_t}\theta^* \\
  V_t^{1/2}(\hat{\theta}_t - \theta^*) &= V_t^{-1/2}Z - \rho V_t^{-1/2}\theta^* \\
  \norm{V_t}{\hat{\theta}_t - \theta^*} &= \norm{}{V_t^{-1/2}Z - \rho V_t^{-1/2}\theta^*} \\
  &\le \norm{\inv{V_t}}{Z} + \rho\norm{\inv{V_t}}{\theta^*}.\
                          &\text{Triangle inequality} \\
  &\le \norm{\inv{V_t}}{Z} + \sqrt\rho \norm{}{\theta^*} &\text{Since } V_t \succeq \rho I.
\end{align*}
We now use \cref{lemma:subgaussian-z,lemma:z-norm-bounded-whp} in our
confidence bound:
\begin{align*}
  \Prob[\Bigg]{\norm{V_t}{\hat{\theta}_t - \theta^*} > \sqrt\rho\norm{}{\theta^*} + \sqrt{2\log\frac{1}{\delta} + \log\frac{\det V_t}{\det \rho I}}}
  &\le \delta.
\end{align*}
We assume that $\norm{}{\theta^*} \le S$.  We use the union bound,
replacing $\delta$ with $\delta/n$ to get a bound that holds uniformly
at every round:
\begin{align*}
  \sqrt{\beta_t} &= \sqrt\rho S + \sqrt{2\log\frac{n}{\delta} + \log\frac{\det V_t}{\det \rho I}}.
\end{align*}
Applying \cref{thm:linucb-regret} gives us the regret bound
\begin{align*}
  \widehat{R}_n
  &\le \sqrt{8dn\beta_{n-1}\log\frac{\tr V_0 + nL^2}{d\det^{1/d} V_0}}
    = \sqrt{8dn\beta_{n-1}\log\frac{d\rho + nL^2}{d\rho}}
  \shortintertext{where}
  \sqrt{\beta_{n-1}}
  &= \sqrt\rho S + \sqrt{2\log\frac{n}{\delta} + \log\frac{\det V_{n-1}}{\det \rho I}} \\
  &\le \sqrt\rho S + \sqrt{2\log\frac{n}{\delta} + d\log\paren*{1 + \frac{nL^2}{d\rho}}}.
\end{align*}
Choosing $\delta=1/n$ and constant $\rho$ gives $\sqrt{\beta_{n-1}} =
O(d^{1/2}\log^{1/2}(n/d))$ and thus the expected regret of Linear UCB
with ellipsoidal confidence sets satisfies
\begin{align*}
  \widehat{R}_n &= O(\beta_n^{1/2}\sqrt{dn\log(n/d)}) = O(d\log(n/d)\sqrt{n}).
\end{align*}




% \section{Facts About Random Variables}

% \begin{definition}[Subgaussian random variable]\label{def:subG}
%   A real-valued random variable $X$ is \emph{$\sigma^2$-subgaussian}
%   if its moment-generating function satisfies:
%   \begin{align*}
%     \mgf_X(s) \defeq \Ex{\exp(sX)} &\le \exp(s^2\sigma^2/2).
%   \end{align*}
% \end{definition}

% \begin{definition}[Subexponential random variable]\label{def:subExp}
%   A real-valued random variable $X$ is \emph{$\lambda$-subexponential}
%   if its moment-generating function satisfies:
%   \begin{align*}
%     \mgf_X(s) \defeq \Ex{\exp(sX)} &\le \exp(s^2\lambda^2/2),
%     &\text{for all } \abs{s}\le 1/\lambda.
%   \end{align*}
% \end{definition}

% \begin{definition}[Subgaussian/subexponential random vector]
%   A random vector $X\in\Real^b$ is subgaussian (subexponential) if the
%   random variable $\innerp{v}{X}$ is subgaussian (subexponential) for
%   any $v\in\Real^n$ with $\norm{}{v}=1$.
% \end{definition}

% \begin{definition}[Subgaussian/subexponential random matrix]
%   A random matrix $W\in\Real^{n\times m}$ is subgaussian (subexponential)
%   if the random vector $Wu\in\Real^n$ is subgaussian
%   (subexponential) for any $u\in\Real^m$ with $\norm{}{u}=1$.
% \end{definition}

\section{Other Proofs}

\begin{lemma}[{\citealp[Lemma~9]{AbbasiYadkoriImprovedAlgorithmsLinear2011}}]\label{lemma:martingale-mgf}
  Let $X_1, X_2, \dotsc \in \Real$ and
  $\sigma_1, \sigma_2, \dotsc \in \Real$ be sequences of random
  variables and let
  $\mathcal{F}_1 \subset \mathcal{F}_2 \subset \dotsb$ be a
  filtration.  Suppose that each $X_t \in \mathcal{F}_t$ and almost
  surely
  $\Ex{\exp(\lambda X_t)\given \mathcal{F}_{t-1}} \le
  \exp(\sigma_t^2\lambda^2/2)$ (i.e. each
  $\sigma_t \in \mathcal{F}_{t-1}$ and $X_t$ is conditionally
  $\sigma_t$-subgaussian given $\mathcal{F}_{t-1}$).  For any
  $\lambda\in\Real$, define
  \begin{align*}
    M_{t,\psi} &\defeq \exp\paren[\bigg]{\sum_{s=1}^t \psi X_s - \frac{1}{2}\sigma_s^2\psi^2}.
  \end{align*}
  Let $\tau$ be a stopping time with respect to
  ${(\mathcal{F}_t)}_{t=1}^\infty$.  Then $M_{\tau,\psi}$ is almost surely
  well-defined and $\Ex{M_{\tau,\psi}} \le 1$.

  \begin{proof}
    We will show that ${(M_{t,\psi})}_{t=1}^\infty$ is a
    supermartingale.  Let
    \begin{align*}
      D_{t,\psi} &\defeq \exp\paren[\bigg]{\psi X_t - \frac{1}{2}\sigma_t^2\psi^2}.
    \end{align*}
    By the conditional subgaussianity of $X_t$, we have
    \begin{align*}
      \Ex{D_{t,\psi}\given\mathcal{F}_{t-1}}
      &= \Ex{\exp(\psi X_t)\given\mathcal{F}_{t-1}}
        \cdot \exp\paren[\bigg]{-\frac{1}{2}\sigma_t^2\psi^2}
      \le \exp\paren[\bigg]{\frac{1}{2}\sigma_t^2\psi^2}
        \cdot \exp\paren[\bigg]{-\frac{1}{2}\sigma_t^2\psi^2}
      = 1.
    \end{align*}
    It is also clear that $D_{t,\psi}$ is $\mathcal{F}_t$-measurable
    (since $X_t,\sigma_t \in \mathcal{F}_t$), and therefore so is
    $M_{t,\psi}$.  Furthermore,
    \begin{align*}
      \Ex{M_{t,\psi}\given\mathcal{F}_{t-1}}
      &= \Ex{M_{t-1,\psi}\cdot D_{t,\psi}\given\mathcal{F}_{t-1}}
        = M_{t-1,\psi} \Ex{D_{t,\psi}\given\mathcal{F}_{t-1}}
        \le M_{t-1,\psi},
    \end{align*}
    showing that $M_{t,\psi}$ is indeed a supermartingale and therefore
    by Doob's optional stopping theorem
    $\Ex{M_{\tau,\psi}} \le \Ex{M_{0,\psi}} = 1$.

    Now, we argue that $M_{\tau,\psi}$ is well-defined.  By the
    convergence theorem for non-negative supermartingales,
    $M_{\infty,\psi} = \lim_{t\to\infty}M_{t,\psi}$ is almost surely
    well-defined.  Hence, $M_{\tau,\psi}$ is almost surely
    well-defined independently of whether $\tau<\infty$ holds or not.
    Next, we show that $\Ex{M_{\tau,\psi}} \le 1$.  For this, let
    $Q_{t,\psi} \defeq M_{\min\{\tau,t\},\psi}$ be a stopped
    version of ${(M_{t,\psi})}_t$.  By Fatou's lemma,
    $\Ex{M_{\tau,\psi}} = \Ex{\liminf_{t\to\infty}Q_{t,\psi}} \le
    \liminf_{t\to\infty}\Ex{Q_{t,\psi}} \le 1$, showing the
    $\Ex{M_{\tau,\psi}} \le 1$ indeed holds.
  \end{proof}
\end{lemma}

\begin{lemma}
  Let $n\in\Nat$ and $\varepsilon>0$ and $\sigma^2>0$.  Let
  $X_1,X_2,\dotsc,X_n\in\Real$ and
  $\sigma_1,\sigma_2,\dotsc,\sigma_n\in\Real$ be sequences of random
  variables and let
  $\mathcal{F}_1 \subset \mathcal{F}_2 \subset \dotsb \subset
  \mathcal{F}_n$ be a filtration.  Suppose that each
  $X_t\in\mathcal{F}_t$ and almost surely each $\sigma_t \le \sigma$
  and each $X_t$ is conditionally $\sigma_t$-subgaussian given
  $\mathcal{F}_{t-1}$.  Then
  \begin{align*}
    \Prob[\Big]{\exists t \le n. \sum_{s=1}^t X_s \ge
    \sqrt{2\gamma_nV_t\log(N/\delta)}}
    &\le \delta,
  \end{align*}
  where
  \begin{align*}
    V_t &\defeq \max\set[\Big]{\varepsilon,\sum_{s=1}^t \sigma_s^2}, &
    \gamma_n &\defeq 1 + \frac{1}{\log(n)}, &
    N &\defeq 1 + \ceil*{\frac{\log(n\sigma^2/\varepsilon)}{\log(\gamma_n)}}.
  \end{align*}

  \begin{proof}
    For $\psi\in\Real$ define
    \begin{align*}
      M_{t,\psi} &\defeq \exp\paren[\bigg]{\sum_{s=1}^t \psi X_s - \frac{1}{2}\psi^2\sigma_s^2}.
    \end{align*}
    If $\tau \le n$ is a stopping time with respect to $\mathcal{F}$,
    then $\Ex{M_{\tau,\psi}} \le 1$ by \cref{lemma:martingale-mgf}.
    Therefore by Markov's inequality,
    \begin{align*}
      \delta/N
      &\ge \Prob{M_{\tau,\psi} \ge N/\delta} \\
      &= \Prob[\bigg]{\exp\paren[\Big]{\sum_{s=1}^\tau \psi X_s - \frac{\psi^2\sigma_s^2}{2}} \ge N/\delta} \\
      &= \Prob[\bigg]{\psi \sum_{s=1}^\tau X_s - \frac{\psi\sigma_s^2}{2} \ge \log(N/\delta)} \\
      &\ge \Prob[\bigg]{\sum_{s=1}^\tau X_s \ge \frac{\log(N/\delta)}{\psi} + \frac{\psi V_\tau}{2}},
      &\text{since } V_\tau \ge \sum_{s=1}^\tau \sigma_s^2.
    \end{align*}
    To get the tightest tail bound, we want to minimize the quantity
    on the right side of the inequality; it attains a minimum of
    $\sqrt{2V_\tau \log(1/\delta)}$ at
    $\psi_{\min} \defeq \sqrt{2\log(N/\delta)/V_\tau}$.  Furthermore,
    for any $\psi\ge\psi_{\min}$ that quantity is at most
    $\psi V_\tau$.

    However, since $V_\tau$ is a random quantity, we cannot simply set
    $\psi \defeq \psi_{\min}$.  Instead, we recognize that
    $V_\tau \in [\varepsilon, n\sigma^2]$ almost surely, and we can
    logarithmically cover this range with the $N$ values
    $\varepsilon\gamma_n^{k-1}$ for $k=1,2,\dotsc,N$; at least one of
    these must lie in the interval $[V_\tau/\gamma_n, V_\tau]$.  Then
    we can define the corresponding values
    \begin{align*}
      \psi_k &\defeq \sqrt{\frac{2\log(N/\delta)}{\varepsilon\gamma_n^{k-1}}},
              &\text{for } k=1,2,\dotsc,N;
    \end{align*}
    there must be some value of $k$ for which
    $\psi_k \in [\psi_{\min}, \gamma_n\psi_{\min}]$.  Using a union bound
    over these $N$ values, we get
    \begin{align*}
      \delta
      &\ge \Prob[\bigg]{\exists k\in[N]. \sum_{s=1}^\tau X_s \ge \frac{\log(N/\delta)}{\psi_k} + \frac{\psi_k V_\tau}{2}} \\
      &\ge \Prob[\bigg]{\sum_{s=1}^\tau X_s \ge \gamma_n\psi_{\min}V_\tau} \\
      &= \Prob[\bigg]{\sum_{s=1}^\tau X_s \ge \sqrt{2\gamma_nV_\tau\log(N/\delta)}}.
    \end{align*}
    To complete the proof, define the stopping time $\tau =
    \min\set{n,\tau_n}$ where
    \begin{align*}
      \tau_n &\defeq \min\set[\Big]{t \le n \given \sum_{s=1}^tX_s \ge \sqrt{2\gamma_nV_t\log(N/\delta)}}.
              \qedhere
    \end{align*}
  \end{proof}
\end{lemma}


\end{document}

%%% Local Variables:
%%% mode: latex
%%% TeX-master: t
%%% End:
